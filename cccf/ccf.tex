\documentclass[a4paper]{article}
\usepackage{ctex}
\usepackage[colorlinks,linkcolor=red]{hyperref}
\usepackage[top=2cm, bottom=2cm, left=2.5cm, right=2.5cm]{geometry}

\begin{document}

\title{CCF通讯历年主题列表}
\author{Fan Liang}
\date{\today}
\maketitle


\section{\href{http://history.ccf.org.cn/sites/ccf/jsjtbbd.jsp?contentId=2968168085256}{\textbf{2017年第1期(总第131期)}}}
本期专题是量子计算。量子计算的报道近年来频繁出现在各类媒体上,实用化的专用量子计算机似乎已呼之欲出。量子计算机进展究竟如何,本期的专题文章从量子算法、量子程序设计和量子计算机的实现方法等各个角度做了较全面的介绍,读者可以从中窥视量子计算机的身影。
\subsection{专题}
\href{http://history.ccf.org.cn/resources/1190201776262/2017/01/11/dengmingtang.pdf}{超导和拓扑量子计算简述}

\href{http://history.ccf.org.cn/resources/1190201776262/2017/01/11/lihaio.pdf}{半导体量子点量子计算}

\href{http://history.ccf.org.cn/resources/1190201776262/2017/01/11/1-wujunjie.pdf}{量子计算}

\href{http://history.ccf.org.cn/resources/1190201776262/2017/01/11/yingmingsheng.pdf}{量子程序设计研究的近期进展}

\href{http://history.ccf.org.cn/resources/1190201776262/2017/01/11/wujunjie2.pdf}{光量子计算}

\href{http://history.ccf.org.cn/resources/1190201776262/2017/01/11/Ashley.pdf}{量子算法综述}

\subsection{特邀报告}
\href{http://history.ccf.org.cn/resources/1190201776262/2017/01/11/cncc-zhouzhihua.pdf}{机器学习:发展与未来}

\subsection{专栏}
\href{http://history.ccf.org.cn/resources/1190201776262/2017/01/11/tangpingzhong.pdf}{计算与经济}

\href{http://history.ccf.org.cn/resources/1190201776262/2017/01/11/songruihua.pdf}{会有那么一天,机器人可以写小说吗?}

\href{http://history.ccf.org.cn/resources/1190201776262/2017/01/11/David.pdf}{高性能计算:全球和当地}

\href{http://history.ccf.org.cn/resources/1190201776262/2017/01/11/bigdata.pdf}{2017年中国大数据发展趋势预测及解读}

\href{http://history.ccf.org.cn/resources/1190201776262/2017/01/11/shiyuanchun.pdf}{学科交叉、融合创新}

\href{http://history.ccf.org.cn/resources/1190201776262/2017/01/11/huangtiejun.pdf}{电脑传奇(中篇):智能之争}

\subsection{动态}
\href{http://history.ccf.org.cn/resources/1190201776262/2017/01/11/dongtai.pdf}{ACM普适计算会议UbiComp 2016}

\subsection{学会论坛}
\href{http://history.ccf.org.cn/resources/1190201776262/2017/01/11/Andrew Johnson.pdf}{现代ICT专业团体在澳大利亚的作用}

\href{http://history.ccf.org.cn/resources/1190201776262/2017/01/11/duzide.pdf}{CCF的危机}


\section{\href{http://history.ccf.org.cn/sites/ccf/jsjtbbd.jsp?contentId=2963156880906}{\textbf{2016年第12期(总第130期)}}}
本期专题是CNCC 2016特邀报告。你知道什么技术和产品可以进行金融风控反欺诈吗?作为高校计算机学院负责人,你是不是很关心该如何打造一个顶尖的计算机学院?当代青年学者正在进行什么研究?创业公司正在做什么产品?如果你没能参加10月20~22日在太原举办的CNCC2016,在本期CCCF 中,你可以读到陈纯、维利·扎汶尼普尔、杰克·戴维森、黄铭均、包云岗、王小川、李志飞等几位大牛的精彩文章。
\subsection{特邀报告}
\href{http://history.ccf.org.cn/resources/1190201776262/2016/12/13/4huangmingjun.pdf}{构建“端到端”的数据处理和分析解决方案}

\href{http://history.ccf.org.cn/resources/1190201776262/2016/12/13/2Willy Zwaenepoel.pdf}{如何打造一个顶尖计算机学院}

\href{http://history.ccf.org.cn/resources/1190201776262/2016/12/13/5baoyungang.pdf}{云计算与标签化冯·诺伊曼结构}

\href{http://history.ccf.org.cn/resources/1190201776262/2016/12/13/1chenchun.pdf}{“流立方”流式大数据实时智能处理技术、平台及应用}

\href{http://history.ccf.org.cn/resources/1190201776262/2016/12/13/3Jack Davidson.pdf}{非同寻常的机器联赛:自主网络推理系统的第一步}

\href{http://history.ccf.org.cn/resources/1190201776262/2016/12/13/6wangxiaochuan.pdf}{人工智能技术与商业思考}

\href{http://history.ccf.org.cn/resources/1190201776262/2016/12/13/7lizhifei.pdf}{AI的商业化之路:AI的幻想与现实}

\href{http://history.ccf.org.cn/resources/1190201776262/2016/12/13/8shichunyi.pdf}{CCF人工智能与模式识别专委会建立的回顾}

\subsection{专栏}
\href{http://history.ccf.org.cn/resources/1190201776262/2016/12/13/12shishuming.pdf}{算数、下棋与识文断句——谈数学应用题的人工智能求解}

\href{http://history.ccf.org.cn/resources/1190201776262/2016/12/13/11huangtiejun.pdf}{电脑传奇*(上篇):计算机出世}

\href{http://history.ccf.org.cn/resources/1190201776262/2016/12/13/13David.pdf}{智能手机是真正的革新?}

\href{http://history.ccf.org.cn/resources/1190201776262/2016/12/13/9wangtao.pdf}{VR全景视频关键技术和进展}

\href{http://history.ccf.org.cn/resources/1190201776262/2016/12/13/10nick.pdf}{第五代计算机的教训}

\subsection{动态}
\href{http://history.ccf.org.cn/resources/1190201776262/2016/12/13/15gantian.pdf}{多媒体会议遇上多元化魅力阿姆斯特丹}

\href{http://history.ccf.org.cn/resources/1190201776262/2016/12/13/14lianzhouhui.pdf}{AI让个人手写字库成为可能}

\subsection{译文}
\href{http://history.ccf.org.cn/resources/1190201776262/2016/12/13/16yiwen.pdf}{超越比特币的区块链技术}

\subsection{学会论坛}
\href{http://history.ccf.org.cn/resources/1190201776262/2016/12/13/17maixn.pdf}{《境外NGO法》的施行对中国社团的影响}


\section{\href{http://history.ccf.org.cn/sites/ccf/jsjtbbd.jsp?contentId=2957454485261}{\textbf{2016年第11期(总第129期)}}}
2 0 1 6 中国计算机大会(CNCC 2016)在太原隆重举行。中国计算机大会参会人数今年再创新高,超过5000人。举办各种论坛和活动达到80多场,也创历史新高。众多海内外知名专家、学者、企业家为大会奉献精彩纷呈的报告。多家国际社团兄弟学会到场向大会致辞。中央电视台、人民日报等40多家媒体对大会进行报道,并对大会进行了现场直播。为什么CNCC受到这样广
泛关注?详见本期特别报道。
\subsection{专题}
\href{http://history.ccf.org.cn/resources/1190201776262/2016/11/10/1huanggang.pdf}{云-端融合:一种云计算新模式}

\href{http://history.ccf.org.cn/resources/1190201776262/2016/11/10/caochun.pdf}{云-端融合下的端设备能耗优化}

\href{http://history.ccf.org.cn/resources/1190201776262/2016/11/10/2huanggang.pdf}{云-端融合应用模型与运行平台}

\href{http://history.ccf.org.cn/resources/1190201776262/2016/11/10/wusong.pdf}{面向云-端融合的移动容器云平台}

\href{http://history.ccf.org.cn/resources/1190201776262/2016/11/10/pengxin.pdf}{云-端融合环境下的个人资源开放共享}

\subsection{专栏}
\href{http://history.ccf.org.cn/resources/1190201776262/2016/11/10/YOCSEF.pdf}{“双一流”是否能够推动我国计算机学科进入世界前列?}

\href{http://history.ccf.org.cn/resources/1190201776262/2016/11/10/lihang.pdf}{对于AI,我们应该期待什么?}

\href{http://history.ccf.org.cn/resources/1190201776262/2016/11/10/zhangzhihua.pdf}{机器学习的发展历程及启示}

\href{http://history.ccf.org.cn/resources/1190201776262/2016/11/10/David.pdf}{真正的模拟}

\subsection{视点}
\href{http://history.ccf.org.cn/resources/1190201776262/2016/11/10/Points.pdf}{大数据知识工程基础理论及其应用研究}

\subsection{动态}
\href{http://history.ccf.org.cn/resources/1190201776262/2016/11/10/zhaodongyan.pdf}{“人创造出来的东西是否能超过人自身的进化?”}

\href{http://history.ccf.org.cn/resources/1190201776262/2016/11/10/liuzeyang.pdf}{第39届ACM信息检索大会}

\href{http://history.ccf.org.cn/resources/1190201776262/2016/11/10/sujinshu.pdf}{SIGCOMM 2016与网络技术发展预测}

\href{http://history.ccf.org.cn/resources/1190201776262/2016/11/10/lifei.pdf}{如何成为比特记忆洪流中的“养蜂人”}

\subsection{译文}
\href{http://history.ccf.org.cn/resources/1190201776262/2016/11/10/yiwen.pdf}{人工智能变革:为什么深度学习瞬间改变你的生活}


\section{\href{http://history.ccf.org.cn/sites/ccf/jsjtbbd.jsp?contentId=2952788900178}{\textbf{2016年第10期(总第128期)}}}
本期专题是大数据时代的医疗。信息技术与医疗领域的跨界融合,使得医疗数据快速增长,呈多样化和复杂化。如何利用医疗大数据,使其在医疗信息化进程中发挥应有的作用,体现出数据的价值,促进医疗行业的发展,这也是医学界和IT界关注的热点。本期专题邀请国内几位在医疗信息化领域有所建树的专家撰文,讨论大数据在临床医疗上的应用,也为广大读者展示了智慧医
疗和精准医疗的发展趋势。
\subsection{专题}
\href{http://history.ccf.org.cn/resources/1190201776262/2016/10/14/wuwenjun.pdf}{大数据时代的医疗}

\href{http://history.ccf.org.cn/resources/1190201776262/2016/10/14/liqince.pdf}{虚拟生理心脏系统的构建及其应用}

\href{http://history.ccf.org.cn/resources/1190201776262/2016/10/14/wangxinyan.pdf}{大数据与精准适度医疗}

\href{http://history.ccf.org.cn/resources/1190201776262/2016/10/14/zhuhaogang.pdf}{数据科学下的青光眼诊治}

\href{http://history.ccf.org.cn/resources/1190201776262/2016/10/14/pengshaoliang.pdf}{天河超级计算机上的生物医药大数据研究}

\subsection{专栏}
\href{http://history.ccf.org.cn/resources/1190201776262/2016/10/14/Yocsef.pdf}{电信诈骗,谁之过?}

\href{http://history.ccf.org.cn/resources/1190201776262/2016/10/14/xulingyu.pdf}{汽车自动驾驶路漫漫}

\href{http://history.ccf.org.cn/resources/1190201776262/2016/10/14/baohaifei.pdf}{现代科技发展之旅}

\href{http://history.ccf.org.cn/resources/1190201776262/2016/10/14/yingxingren.pdf}{机器能思考吗?——认知与真实}

\href{http://history.ccf.org.cn/resources/1190201776262/2016/10/14/liuqi.pdf}{生物信息学研究的思考}

\href{http://history.ccf.org.cn/resources/1190201776262/2016/10/14/David.pdf}{软件工厂}

\subsection{视点}
\href{http://history.ccf.org.cn/resources/1190201776262/2016/10/14/taojianhua.pdf}{情感计算研究进展}

\subsection{动态}
\href{http://history.ccf.org.cn/resources/1190201776262/2016/10/14/huangwenjian.pdf}{CSCW 2016——计算机支持的协同工作及社会计算会议}

\href{http://history.ccf.org.cn/resources/1190201776262/2016/10/14/zhangweiguo.pdf}{香农的贡献及其对后世的影响——香农百年诞辰纪念}

\href{http://history.ccf.org.cn/resources/1190201776262/2016/10/14/qianhong.pdf}{“知晓有人的人工智能”——IJCAI 2016}

\subsection{译文}
\href{http://history.ccf.org.cn/resources/1190201776262/2016/10/14/yiwen.pdf}{21世纪的计算生物学:可扩展压缩算法}


\section{\href{http://history.ccf.org.cn/sites/ccf/jsjtbbd.jsp?contentId=2947259303272}{\textbf{2016年第9期(总第127期)}}}
本期专题是网络领域新兴课程的建设。网络技术的发展日新月异,现代工程技术人员只有熟悉并掌握相关技术知识和技能才能适应快速发展的社会。如何培养专业化人才?如何创新、创业?高校信息技术类专业责无旁贷。很多高校已开始着手进行网络领域课程改革, 开设新兴课程。本期专题几位作者介绍了他们在开设新兴课程中的思路、经验和体会,希望能给还在教学改革路上探索的教师以启发。
\subsection{专题}
\href{http://history.ccf.org.cn/resources/1190201776262/2016/09/12/cuiyong.pdf}{开展互联网创新创业课程教学的思考}

\href{http://history.ccf.org.cn/resources/1190201776262/2016/09/12/jinguang.pdf}{无线网络技术:从章节到独立课程}

\href{http://history.ccf.org.cn/resources/1190201776262/2016/09/12/wangxinbing1.pdf}{网络领域新兴课程的建设}

\href{http://history.ccf.org.cn/resources/1190201776262/2016/09/12/wangxinbing2.pdf}{移动互联网时代对大学课程的挑战}

\href{http://history.ccf.org.cn/resources/1190201776262/2016/09/12/xuke.pdf}{赛博新经济:网络与经济交叉学科的教学探索}

\subsection{专栏}
\href{http://history.ccf.org.cn/resources/1190201776262/2016/09/12/wanyun.pdf}{解析软银的安谋收购}

\href{http://history.ccf.org.cn/resources/1190201776262/2016/09/12/yuanchunfeng.pdf}{计算机系统核心教学内容之关联}

\href{http://history.ccf.org.cn/resources/1190201776262/2016/09/12/zhangzheng.pdf}{从类鸟飞行到类脑计算:读《莱特兄弟》}

\href{http://history.ccf.org.cn/resources/1190201776262/2016/09/12/gaowen.pdf}{如何成为优秀计算机学者}

\href{http://history.ccf.org.cn/resources/1190201776262/2016/09/12/chengang.pdf}{形式化数学和证明工程}

\href{http://history.ccf.org.cn/resources/1190201776262/2016/09/12/david.pdf}{对数据的信任}

\href{http://history.ccf.org.cn/resources/1190201776262/2016/09/12/maobo.pdf}{我国计算机学者在CCF推荐A类会议上发表论文情况}

\subsection{动态}
\href{http://history.ccf.org.cn/resources/1190201776262/2016/09/12/tangjie.pdf}{第22届国际数据挖掘与知识发现大会}

\href{http://history.ccf.org.cn/resources/1190201776262/2016/09/12/huanglongbo.pdf}{法国小城的分析评测大会——ACM SIGMETRICS/IFIP Performance 2016}

\href{http://history.ccf.org.cn/resources/1190201776262/2016/09/12/maxin.pdf}{重塑生产力 展望计算机科学新发展——2016微软教育峰会:工业界与学术界的顶级对话}

\subsection{译文}
\href{http://history.ccf.org.cn/resources/1190201776262/2016/09/12/yiwen.pdf}{基于公式的软件调试}

\subsection{学会论坛}
\href{http://history.ccf.org.cn/resources/1190201776262/2016/09/12/lixiaoming.pdf}{开放CWP200T的意义}


\section{\href{http://history.ccf.org.cn/sites/ccf/jsjtbbd.jsp?contentId=2941729697105}{\textbf{2016年第8期(总第126期)}}}
本期专题是大数据时代的城市计算。城市智能化的重要性毋庸置疑。“智慧城市”离我们有多远?在本期“大数据时代的城市计算”专题中,陈宝权教授等学者讲述了目前城市计算面临的四个问题以及解决之道。文章指出,通过各种信息采集设备可以实现对城市环境的数据获取,但要达到城市或家庭的智能化,关键是要建立一个统一的智能化支持环境,将各种信息无缝地整合在一起,形成一个有机的整体。几篇专题文章都有较充实的内容和一定的技术深度。
\subsection{专题}
\href{http://history.ccf.org.cn/resources/1190201776262/2016/08/11/4gaoyunjun.pdf}{度量空间数据管理}

\href{http://history.ccf.org.cn/resources/1190201776262/2016/08/11/1chenbaoquan.pdf}{大数据时代的城市计算}

\href{http://history.ccf.org.cn/resources/1190201776262/2016/08/11/2chenbaoquan.pdf}{大规模城市场景建模与理解}

\href{http://history.ccf.org.cn/resources/1190201776262/2016/08/11/yuxiaohui.pdf}{城市时空大数据的研究与应用现状}

\href{http://history.ccf.org.cn/resources/1190201776262/2016/08/11/5yuanxiaoru.pdf}{城市移动数据知微探秘}

\subsection{专栏}
\href{http://history.ccf.org.cn/resources/1190201776262/2016/08/11/nike.pdf}{自动定理证明兴衰纪(下)}

\href{http://history.ccf.org.cn/resources/1190201776262/2016/08/11/yocsef.pdf}{青年计算机工作者的创新创业之路该如何走}

\href{http://history.ccf.org.cn/resources/1190201776262/2016/08/11/chendaoxu.pdf}{工程教育认证:中国工程教育国际化的有力推手}

\href{http://history.ccf.org.cn/resources/1190201776262/2016/08/11/david.pdf}{态势感知}

\href{http://history.ccf.org.cn/resources/1190201776262/2016/08/11/wugansha.pdf}{自动驾驶:技术、产业和社会变革}

\href{http://history.ccf.org.cn/resources/1190201776262/2016/08/11/zhengweimin.pdf}{从“足够好”到卓越}

\subsection{动态}
\href{http://history.ccf.org.cn/resources/1190201776262/2016/08/11/huangshengjun.pdf}{从多标记学习到主动学习——访2015 CCF优博奖获得者黄圣君博士}

\href{http://history.ccf.org.cn/resources/1190201776262/2016/08/11/meitao.pdf}{绚烂夺目的计算机视觉盛宴——第29届国际计算机视觉与模式识别会议}

\href{http://history.ccf.org.cn/resources/1190201776262/2016/08/11/anbo.pdf}{2016年智能体及多智能体系统会议}

\subsection{译文}
\href{http://history.ccf.org.cn/resources/1190201776262/2016/08/11/yiwen.pdf}{社交机器人的兴起}

\subsection{学会论坛}
\href{http://history.ccf.org.cn/resources/1190201776262/2016/08/11/luoxun.pdf}{地域发展遍天下,四海之内皆同仁——论会员与分部工委的工作宗旨和任务}

\href{http://history.ccf.org.cn/resources/1190201776262/2016/08/11/hushimin.pdf}{如何办好中国计算机学会专委?}


\section{\href{http://history.ccf.org.cn/sites/ccf/jsjtbbd.jsp?contentId=2936476895120}{\textbf{2016年第7期(总第125期)}}}
本期专题是软件定义网络。软件定义网络是近年网络界的研究热点,学术界和工业界对该技术及其未来应用前景给予了较高的评价。本期专题以第63期CCF学科前沿讲习班为基础,邀请国内外专家撰文,对软件定义网络领域的前沿技术研究进行了探讨。他们思路开阔,问题导向鲜明,从
不同的角度一步一步地解决互联网存在的问题。
\subsection{专题}
\href{http://history.ccf.org.cn/resources/1190201776262/2016/07/12/3.pdf}{从OpenDaylight看软件网络趋势与开源力量}

\href{http://history.ccf.org.cn/resources/1190201776262/2016/07/12/2nick.pdf}{用P4对数据平面进行编程}

\href{http://history.ccf.org.cn/resources/1190201776262/2016/07/12/1.pdf}{软件定义网络}

\href{http://history.ccf.org.cn/resources/1190201776262/2016/07/12/5bijun.pdf}{域间软件定义网络互联机制研究进展}

\href{http://history.ccf.org.cn/resources/1190201776262/2016/07/12/4yangyang.pdf}{软件定义网络编程:问题与进展}

\href{http://history.ccf.org.cn/resources/1190201776262/2016/07/12/6zhouwu.pdf}{段路由下的软件定义网络}

\subsection{专栏}
\href{http://history.ccf.org.cn/resources/1190201776262/2016/07/15/p53.pdf}{教育部学科评估应取消排名}

\href{http://history.ccf.org.cn/resources/1190201776262/2016/07/12/8xujun.pdf}{多样化——排序学习发展的新方向}

\href{http://history.ccf.org.cn/resources/1190201776262/2016/07/12/9David1.pdf}{非工程环境中的工程决策}

\href{http://history.ccf.org.cn/resources/1190201776262/2016/07/12/7nick-new.pdf}{自动定理证明兴衰纪(上)}

\subsection{动态}
\href{http://history.ccf.org.cn/resources/1190201776262/2016/07/12/12.pdf}{访CCF杰出教育奖获得者周立柱教授}

\href{http://history.ccf.org.cn/resources/1190201776262/2016/07/12/13.pdf}{第32届国际数据工程大会}

\subsection{译文}
\href{http://history.ccf.org.cn/resources/1190201776262/2016/07/12/14yiwen.pdf}{深层还是浅层?自然语言处理技术正在迎来突破性进展}

\subsection{学会论坛}
\href{http://history.ccf.org.cn/resources/1190201776262/2016/07/12/16.pdf}{CCF应该发出响亮的声音——CCF公共政策委员会工作方案}

\href{http://history.ccf.org.cn/resources/1190201776262/2016/07/12/15.pdf}{有意栽花与无心插柳——自由研讨对学术推动的思考}


\section{\href{http://history.ccf.org.cn/sites/ccf/jsjtbbd.jsp?contentId=2931600070199}{\textbf{2016年第6期(总第124期)}}}
本期专题是数字视频技术与标准。数字化的音频、图像和视频在深刻改变着我们的生活,数字视频编码技术和标准的发展起了决定性作用。中国的制造业在此过程中经历了一场专利之痛。中国自主音视频标准AVS成功开创了一条从技术到标准,再到产业的创新之路,为建立标准自主、专利
可控、芯片自有、运营可靠的数字电视产业链做出了重大贡献。在国家发布“创新驱动战略纲要”之际,本期专题有重要的启示作用。
\subsection{专题}
\href{http://history.ccf.org.cn/resources/1190201776262/2016/06/13/1.pdf}{数字视频技术与标准}

\href{http://history.ccf.org.cn/resources/1190201776262/2016/06/13/5.pdf}{数字视频编码技术在广电的应用}

\href{http://history.ccf.org.cn/resources/1190201776262/2016/06/13/3.pdf}{视频编码标准背后的专利许可问题}

\href{http://history.ccf.org.cn/resources/1190201776262/2016/06/13/4.pdf}{视频编码技术发展与趋势}

\href{http://history.ccf.org.cn/resources/1190201776262/2016/06/13/2.pdf}{数字视频编码技术与标准漫谈}

\subsection{专栏}
\href{http://history.ccf.org.cn/resources/1190201776262/2016/06/13/9.pdf}{我国网络空间安全技能竞赛现状漫谈}

\href{http://history.ccf.org.cn/resources/1190201776262/2016/06/13/12.pdf}{IT职业教育改革发展的几点思考}

\href{http://history.ccf.org.cn/resources/1190201776262/2016/06/13/13.pdf}{“云”之得名}

\href{http://history.ccf.org.cn/resources/1190201776262/2016/06/13/6.pdf}{未来十年计算机科学研究需要的四种思维}

\href{http://history.ccf.org.cn/resources/1190201776262/2016/06/13/7.pdf}{2016密码复兴}

\href{http://history.ccf.org.cn/resources/1190201776262/2016/06/13/11.pdf}{智能织物与服装:人类的“第二层肌肤”}

\href{http://history.ccf.org.cn/resources/1190201776262/2016/06/13/10.pdf}{从“Make It Happen”宏指令想到的}

\href{http://history.ccf.org.cn/resources/1190201776262/2016/06/13/8.pdf}{影响信息安全产品的十大论文}

\subsection{动态}
\href{http://history.ccf.org.cn/resources/1190201776262/2016/06/13/14.pdf}{INFOCOM 2016会议}

\href{http://history.ccf.org.cn/resources/1190201776262/2016/06/13/15.pdf}{机器学习顶级会议NIPS 2015}

\subsection{译文}
\href{http://history.ccf.org.cn/resources/1190201776262/2016/06/13/16.pdf}{ACM为推进《为所有人的计算机科学》倡议而努力的十年}


\section{\href{http://history.ccf.org.cn/sites/ccf/jsjtbbd.jsp?contentId=2926004897743}{\textbf{2016年第5期(总第123期)}}}
本期专题是网络空间安全。2015年6月,国务院学位委员会在工学门类下增设“网络空间安全”一级学科。然而“网络空间安全”至今还没有独立的基础理论体系,如何开展网络安全技术研究,发展网络空间安全学科值得深思。本期专题邀请几位参与“网络空间安全”一级学科论证工作的专家和在学术上有深厚造诣的学者,介绍他们对该领域的认识和深入思考,着重展示网络
空间安全基础、网络安全和应用安全等方向的代表性成果。
\subsection{专题}
\href{http://history.ccf.org.cn/resources/1190201776262/2016/05/12/3.pdf}{保证源和路径真实性的转发验证机制}

\href{http://history.ccf.org.cn/resources/1190201776262/2016/05/12/5.pdf}{无线体域网中的数据安全与隐私保护}

\href{http://history.ccf.org.cn/resources/1190201776262/2016/05/12/2.pdf}{《安全通论》趣谈——从“网络空间安全一级学科”说起}

\href{http://history.ccf.org.cn/resources/1190201776262/2016/05/12/6.pdf}{面向小语种的跨语言网络信息安全研究}

\href{http://history.ccf.org.cn/resources/1190201776262/2016/05/12/1.pdf}{网络空间安全}

\href{http://history.ccf.org.cn/resources/1190201776262/2016/05/12/4.pdf}{云计算中数据安全挑战与研究进展}

\subsection{专栏}
\href{http://history.ccf.org.cn/resources/1190201776262/2016/05/12/10.pdf}{斯坦福大学计算机专业的本科教育}

\href{http://history.ccf.org.cn/resources/1190201776262/2016/05/12/15.pdf}{编程}

\href{http://history.ccf.org.cn/resources/1190201776262/2016/05/12/11.pdf}{专业精神从哪里来?}

\href{http://history.ccf.org.cn/resources/1190201776262/2016/05/12/8.pdf}{统计、计算和未来}

\href{http://history.ccf.org.cn/resources/1190201776262/2016/05/12/13.pdf}{在科研的大海中努力前行——本科生从事科研训练的经验与体会}

\href{http://history.ccf.org.cn/resources/1190201776262/2016/05/12/7.pdf}{量子计算数论}

\href{http://history.ccf.org.cn/resources/1190201776262/2016/05/12/9.pdf}{从引力波看未来重大科学发现的新模式}

\href{http://history.ccf.org.cn/resources/1190201776262/2016/05/12/12.pdf}{指导计算机专业本科生开展科研的体会}

\href{http://history.ccf.org.cn/resources/1190201776262/2016/05/12/14.pdf}{从互联网的昨天看大数据的明天}

\subsection{动态}
\href{http://history.ccf.org.cn/resources/1190201776262/2016/05/12/16.pdf}{只问耕耘,不问收获——访2015 CCF优博获奖者张兰博士}

\href{http://history.ccf.org.cn/resources/1190201776262/2016/05/12/17.pdf}{追随本心,坚持不懈,终有所成——访2015 CCF优博获奖者李泽超博士}

\href{http://history.ccf.org.cn/resources/1190201776262/2016/05/12/18.pdf}{第25届国际万维网大会}

\subsection{译文}
\href{http://history.ccf.org.cn/resources/1190201776262/2016/05/12/19.pdf}{操作系统50年}


\section{\href{http://history.ccf.org.cn/sites/ccf/jsjtbbd.jsp?contentId=2917443956516}{\textbf{2016年第4期(总第122期)}}}
本期专题是数据时代的网络科学。在对复杂网络共性特征研究的基础上,网络科学基于数据的复杂系统模型以一种全新的视角发展成为一门新学科。网络科学不仅是真正的科学,而且将成为物理、生物、社会等众多科学的基础。李国杰主编认为,网络科学,也是21世纪的元科学。本期专题邀请在科研第一线已有出色研究成果的青年学者撰文介绍他们的体会与收获,相信会激励更多
学者关心网络科学。
\subsection{专题}
\href{http://history.ccf.org.cn/resources/1190201776262/2016/04/11/夏.pdf}{熵控网络}

\href{http://history.ccf.org.cn/resources/1190201776262/2016/04/11/唐杰.pdf}{社交网络的三角定律}

\href{http://history.ccf.org.cn/resources/1190201776262/2016/04/11/周涛.pdf}{网络链路预测:概念与前沿}

\href{http://history.ccf.org.cn/resources/1190201776262/2016/04/11/宋朝鸣.pdf}{关于影响力的刻画与预测——透过网络科学视角审视成功}

\href{http://history.ccf.org.cn/resources/1190201776262/2016/04/11/汪小帆.pdf}{数据时代的网络科学}

\subsection{专栏}
\href{http://history.ccf.org.cn/resources/1190201776262/2016/04/11/鲍海飞.pdf}{博士生的出路在哪里}

\href{http://history.ccf.org.cn/resources/1190201776262/2016/04/11/David.pdf}{软件面面观}

\href{http://history.ccf.org.cn/resources/1190201776262/2016/04/11/刘锋.pdf}{从机器人到谷歌大脑——人工智能的6个智能分级}

\href{http://history.ccf.org.cn/resources/1190201776262/2016/04/11/陈云霁.pdf}{什么是类脑计算机?}

\href{http://history.ccf.org.cn/resources/1190201776262/2016/04/11/吴朝晖.pdf}{现代服务业与服务计算:新模型新定义新框架}

\subsection{动态}
\href{http://history.ccf.org.cn/resources/1190201776262/2016/04/11/高文专访.pdf}{要使每一位会员都快乐——访CCF第十一届理事长高文教授}

\href{http://history.ccf.org.cn/resources/1190201776262/2016/04/11/老外专访.pdf}{重要的是你曾经让别人的世界更美好——访ACM图灵奖得主约翰·霍普克洛夫特教授}

\subsection{译文}
\href{http://history.ccf.org.cn/resources/1190201776262/2016/04/11/译文-.pdf}{空间计算}


\section{\href{http://history.ccf.org.cn/sites/ccf/jsjtbbd.jsp?contentId=2914672406466}{\textbf{2016年第3期(总第121期)}}}
本期专题是基于3C融合的天地一体化信息系统。利用信息通信技术及互联网平台,让互联网与各个传统行业进行深度融合,创造出新的发展生态,从而推动生产力的发展,这其中的一项关键技术就是计算、通信和控制的3C融合。只有通过3C技术的有机融合与深度协作,才能实现各个传统行业中大型工程系统所需的实时感知、动态控制和信息服务。本期专题没有介绍国外的进展,而是讲述自己的工作。不妨一读。
\subsection{专题}
\href{http://history.ccf.org.cn/resources/1190201776262/2016/03/14/5.pdf}{空天地一体化信息网络中的移动空基骨干传输网}

\href{http://history.ccf.org.cn/resources/1190201776262/2016/03/14/6.pdf}{基于导航与通信融合的室内定位与位置服务}

\href{http://history.ccf.org.cn/resources/1190201776262/2016/03/14/1.pdf}{基于3C融合的天地一体化信息系统}

\href{http://history.ccf.org.cn/resources/1190201776262/2016/03/14/4.pdf}{空间信息网络体系架构与关键技术}

\href{http://history.ccf.org.cn/resources/1190201776262/2016/03/14/2.pdf}{“计算机即网络”理念与高通量计算}

\href{http://history.ccf.org.cn/resources/1190201776262/2016/03/14/3.pdf}{面向3C融合的空天地一体超级基站系统及关键技术}

\subsection{专栏}
\href{http://history.ccf.org.cn/resources/1190201776262/2016/03/14/9.pdf}{盲人触觉交互的演进及未来趋势分析}

\href{http://history.ccf.org.cn/resources/1190201776262/2016/03/14/8.pdf}{超导RSFQ计算机}

\href{http://history.ccf.org.cn/resources/1190201776262/2016/03/14/10.pdf}{简论人工智能}

\href{http://history.ccf.org.cn/resources/1190201776262/2016/03/14/11.pdf}{迈向自动化中的软件开发}

\href{http://history.ccf.org.cn/resources/1190201776262/2016/03/14/7.pdf}{人工智能的缘起}

\subsection{动态}
\href{http://history.ccf.org.cn/resources/1190201776262/2016/03/14/14.pdf}{立足本土做系统软件“全栈”研究者——访2015 CCF青年科学家奖获得者陈海波博士}

\href{http://history.ccf.org.cn/resources/1190201776262/2016/03/14/13.pdf}{矢志不移,厚积薄发,十年磨一剑——访2015 CCF青年科学家奖获得者山世光博士}

\href{http://history.ccf.org.cn/resources/1190201776262/2016/03/14/15.pdf}{人工智能的春节联欢——AAAI 2016}

\href{http://history.ccf.org.cn/resources/1190201776262/2016/03/14/12.pdf}{放好心态,去见识更多的精彩——访2015 CCF青年科学家奖获得者许畅博士}

\subsection{译文}
\href{http://history.ccf.org.cn/resources/1190201776262/2016/03/14/16.pdf}{计算语言学与深度学习}

\subsection{学会论坛}
\href{http://history.ccf.org.cn/resources/1190201776262/2016/03/14/17.pdf}{运作专委六年有感}


\section{\href{http://history.ccf.org.cn/sites/ccf/jsjtbbd.jsp?contentId=2908932108057}{\textbf{2016年第2期(总第120期)}}}
CCF理事长就职典礼暨2015颁奖大会隆重举行
1月30日,CCF第十一届理事长高文,副理事长吕建、孙凝晖、王巨宏在300多位CCF会员的见证下庄严宣誓就职,这一历史时刻将永远载入CCF史册。2015CCF颁奖大会同时举行。CCF优博奖、青年科学家奖、杰出教育奖、夏培肃奖、杰出贡献奖、卓
越服务奖、终身成就奖七大奖项数十位获奖者,走过红毯,登上领奖台。IEEE-CS、ACM、IPSJ、KIISE等国际学会领导出席颁奖大会,或在现场发言或通过视频向CCF新任理事长表示祝贺。
\subsection{专题}
\href{http://history.ccf.org.cn/resources/1190201776262/2016/02/17/4.pdf}{开源软件的量化分析}

\href{http://history.ccf.org.cn/resources/1190201776262/2016/02/17/6.pdf}{开源软件缺陷管理及自动修复}

\href{http://history.ccf.org.cn/resources/1190201776262/2016/02/17/5.pdf}{开源软件系统缺陷报告管理与分析}

\href{http://history.ccf.org.cn/resources/1190201776262/2016/02/16/7.pdf}{企业视角看到的开源——华为开源5年实践经验}

\href{http://history.ccf.org.cn/resources/1190201776262/2016/02/17/8.pdf}{开源软件和开源社区的反思}

\href{http://history.ccf.org.cn/resources/1190201776262/2016/02/16/3.pdf}{开源软件生态:研究与实践}

\subsection{专栏}
\href{http://history.ccf.org.cn/resources/1190201776262/2016/02/16/9.pdf}{关于校企科研合作的几点认识}

\href{http://history.ccf.org.cn/resources/1190201776262/2016/02/16/10.pdf}{深度学习与人脑模拟}

\href{http://history.ccf.org.cn/resources/1190201776262/2016/02/16/11.pdf}{科研路上探索的快乐程序员}

\href{http://history.ccf.org.cn/resources/1190201776262/2016/02/16/12.pdf}{数字图书馆的长处与短处}

\subsection{译文}
\href{http://history.ccf.org.cn/resources/1190201776262/2016/02/17/17.pdf}{后摩尔定律时代的计算技术}

\subsection{特别报道}
\href{http://history.ccf.org.cn/resources/1190201776262/2016/02/16/2.pdf}{2015 CCF颁奖大会在京隆重举行}

\href{http://history.ccf.org.cn/resources/1190201776262/2016/02/16/1.pdf}{CCF第十一届理事长就职典礼在京隆重举行}


\section{\href{http://history.ccf.org.cn/sites/ccf/jsjtbbd.jsp?contentId=2903940692049}{\textbf{2016年第1期(总第119期)}}}
专题《移动智能终端的软件安全》
移动手机安全吗?智能化汽车有没有安全隐患?有数据表明,在移动设备不断增加的情况下,恶意软件感染事件也以每年20%的速度在增长。它们不但危及设备本身、泄露用户信息,而且还会危及人们的金融安全、生命安全,乃至国家的信息安全等,保障移动网络安全迫在眉睫,并已提升为国家战略。本期专题从几个不同视角介绍了移动智能终端安全、隐私保护的现状和研究热点,相信会给读者带来一些知识和启发。
\subsection{专题}
\href{http://history.ccf.org.cn/resources/1190201776262/2016/01/13/2.pdf}{安卓设备数据存储安全研究进展}

\href{http://history.ccf.org.cn/resources/1190201776262/2016/01/13/3.pdf}{移动终端安全新挑战:感知能力的安全问题与应对}

\href{http://history.ccf.org.cn/resources/1190201776262/2016/01/13/6.pdf}{移动数字取证技术}

\href{http://history.ccf.org.cn/resources/1190201776262/2016/01/13/5.pdf}{移动平台用户隐私保护技术}

\href{http://history.ccf.org.cn/resources/1190201776262/2016/01/13/4.pdf}{汽车智能化的安全思考}

\href{http://history.ccf.org.cn/resources/1190201776262/2016/01/22/导言.pdf}{移动智能终端的软件安全}

\subsection{专栏}
\href{http://history.ccf.org.cn/resources/1190201776262/2016/01/13/15.pdf}{IEEE-CS:2016年科技发展的九大趋势}

\href{http://history.ccf.org.cn/resources/1190201776262/2016/01/13/14.pdf}{学习型机器}

\href{http://history.ccf.org.cn/resources/1190201776262/2016/01/13/8.pdf}{李国杰谈专委工作}

\href{http://history.ccf.org.cn/resources/1190201776262/2016/01/13/7.pdf}{2016年大数据发展趋势预测解读}

\href{http://history.ccf.org.cn/resources/1190201776262/2016/01/13/12.pdf}{人机交互领域的论文写作}

\href{http://history.ccf.org.cn/resources/1190201776262/2016/01/13/13.pdf}{华为研究的畅想:Educated AI}

\href{http://history.ccf.org.cn/resources/1190201776262/2016/01/13/9.pdf}{百度人工智能计算机的探索与实践}

\href{http://history.ccf.org.cn/resources/1190201776262/2016/01/13/10.pdf}{数据科学中的“数据智慧”}

\href{http://history.ccf.org.cn/resources/1190201776262/2016/01/13/11.pdf}{行业组织现象分析}

\subsection{动态}
\href{http://history.ccf.org.cn/resources/1190201776262/2016/01/13/17.pdf}{技术创新与系统的融合——NSDI 2015会议}

\href{http://history.ccf.org.cn/resources/1190201776262/2016/01/13/18.pdf}{新技术 \& 新应用}

\href{http://history.ccf.org.cn/resources/1190201776262/2016/01/13/16.pdf}{存储闸门理论:解决存储墙问题的一种新方案}

\subsection{译文}
\href{http://history.ccf.org.cn/resources/1190201776262/2016/01/13/20.pdf}{大规模系统中的故障}

\subsection{学会论坛}
\href{http://history.ccf.org.cn/resources/1190201776262/2016/01/13/19.pdf}{从开放式选举看学会民主治理的进步}


\section{\href{http://history.ccf.org.cn/sites/ccf/jsjtbbd.jsp?contentId=2897719129971}{\textbf{2015年第12期(总第118期)}}}
特别报道《追思张效祥先生》2015年10月22日,张效祥先生在北京逝世。为了深切缅怀这位中国计算机事业的创始人,中国计算机学会在北京隆重举行追思会。与会者观看了张先生生平记录片,重温张先生的音容笑貌,深情追思张先生高瞻远瞩的胸怀、严谨的学风、谦虚的为人、正直无私的品德。会后,又有多人撰写文章,寄托他们的哀思和缅怀之情。
\subsection{特邀报告}
\href{http://history.ccf.org.cn/resources/1190201776262/2015/12/11/11.pdf}{论骑车穿越美国与实现Postgres的相通之处}

\href{http://history.ccf.org.cn/resources/1190201776262/2015/12/11/12.pdf}{脑认知的形式化}

\href{http://history.ccf.org.cn/resources/1190201776262/2015/12/11/16.pdf}{人工智能时代:聚合智能、自适应智能、隐形智能和增强智能}

\href{http://history.ccf.org.cn/resources/1190201776262/2015/12/11/13.pdf}{量子信息技术前沿进展}

\href{http://history.ccf.org.cn/resources/1190201776262/2015/12/11/15.pdf}{魔术中的数学}

\href{http://history.ccf.org.cn/resources/1190201776262/2015/12/11/18.pdf}{面向下一代数据中心的关键技术研究与实践}

\href{http://history.ccf.org.cn/resources/1190201776262/2015/12/11/17.pdf}{网络环境下的计算可视媒体}

\href{http://history.ccf.org.cn/resources/1190201776262/2015/12/11/14.pdf}{注重社会效益的日本大数据研究}

\subsection{专栏}
\href{http://history.ccf.org.cn/resources/1190201776262/2015/12/11/20.pdf}{研究到产品:距离有多远}

\href{http://history.ccf.org.cn/resources/1190201776262/2015/12/11/21.pdf}{教学之感悟}

\href{http://history.ccf.org.cn/resources/1190201776262/2015/12/11/22.pdf}{青年教师创业之惑}

\href{http://history.ccf.org.cn/resources/1190201776262/2015/12/11/19.pdf}{与学生合作开展研究的体会}

\href{http://history.ccf.org.cn/resources/1190201776262/2015/12/11/23.pdf}{看似不可能完成的任务}

\subsection{动态}
\href{http://history.ccf.org.cn/resources/1190201776262/2015/12/11/25.pdf}{新技术 \& 新应用}

\href{http://history.ccf.org.cn/resources/1190201776262/2015/12/11/24.pdf}{多媒体的视听盛宴——第23届国际多媒体大会}

\subsection{译文}
\href{http://history.ccf.org.cn/resources/1190201776262/2015/12/11/26.pdf}{大规模开放在线课程之殇:在线科学教育需要新的变革}

\subsection{特别报道}
\href{http://history.ccf.org.cn/resources/1190201776262/2015/12/11/10.pdf}{回忆张效祥先生对青少年的关心}

\href{http://history.ccf.org.cn/resources/1190201776262/2015/12/11/1.pdf}{张效祥先生追思会在京举行}

\href{http://history.ccf.org.cn/resources/1190201776262/2015/12/11/4.pdf}{怀念张效祥先生}

\href{http://history.ccf.org.cn/resources/1190201776262/2015/12/11/8.pdf}{良师益友张效祥先生}

\href{http://history.ccf.org.cn/resources/1190201776262/2015/12/11/6.pdf}{张先生是我们学习的榜样}

\href{http://history.ccf.org.cn/resources/1190201776262/2015/12/11/3.pdf}{回忆张先生二三事}

\href{http://history.ccf.org.cn/resources/1190201776262/2015/12/11/7.pdf}{张老的数据库情怀}

\href{http://history.ccf.org.cn/resources/1190201776262/2015/12/11/2.pdf}{“要有自己的计算机事业”}

\href{http://history.ccf.org.cn/resources/1190201776262/2015/12/11/9.pdf}{严谨求实任劳任怨的张效祥先生}

\href{http://history.ccf.org.cn/resources/1190201776262/2015/12/11/5.pdf}{张效祥先生 一路走好}


\section{\href{http://history.ccf.org.cn/sites/ccf/jsjtbbd.jsp?contentId=2894455208267}{\textbf{2015年第11期(总第117期)}}}
特别报道《CNCC 2015盛会》
今年的CNCC超过4000人参会,又创新高;各种论坛、各种活动、各种会议约50个,“科技集市”初见端倪;美国、日本、韩国等更多的国际科技团体参加CNCC,并与CCF会谈合作。CCF第十一次会员代表大会同期召开,产生新一届理事会;新老理事共话CCF未来发展。欲了解更多内容,请看特别报道。
\subsection{专题}
\href{http://history.ccf.org.cn/resources/1190201776262/2015/11/12/7.pdf}{面向地图自动综合的空间相似关系}

\href{http://history.ccf.org.cn/resources/1190201776262/2015/11/12/5.pdf}{中文文本的时空信息获取方法}

\href{http://history.ccf.org.cn/resources/1190201776262/2015/11/12/6.pdf}{空间网络的数据挖掘和应用}

\href{http://history.ccf.org.cn/resources/1190201776262/2015/11/12/3.pdf}{定性空间推理及其应用}

\href{http://history.ccf.org.cn/resources/1190201776262/2015/11/12/4.pdf}{基于空间大数据的社会感知}

\href{http://history.ccf.org.cn/resources/1190201776262/2015/11/12/2.pdf}{空间推理与空间数据分析}

\subsection{专栏}
\href{http://history.ccf.org.cn/resources/1190201776262/2015/11/12/11.pdf}{基于场景驱动的研究方式}

\href{http://history.ccf.org.cn/resources/1190201776262/2015/11/12/8.pdf}{国家科技奖励应从奖项目向奖人转变}

\href{http://history.ccf.org.cn/resources/1190201776262/2015/11/12/9.pdf}{屠呦呦:科技界的新偶像}

\href{http://history.ccf.org.cn/resources/1190201776262/2015/11/12/14.pdf}{弥合鸿沟}

\href{http://history.ccf.org.cn/resources/1190201776262/2015/11/12/12.pdf}{从好的研究方向到强大的研究团队}

\href{http://history.ccf.org.cn/resources/1190201776262/2015/11/12/13.pdf}{招聘安全领域人才的几点观察}

\href{http://history.ccf.org.cn/resources/1190201776262/2015/11/12/10.pdf}{从苹果XCodeGhost事件看互联网软件安全问题}

\subsection{动态}
\href{http://history.ccf.org.cn/resources/1190201776262/2015/11/12/16.pdf}{从2015自然语言处理实证方法会议看发展趋势}

\href{http://history.ccf.org.cn/resources/1190201776262/2015/11/12/17.pdf}{当学术邂逅浪漫——记MobiCom 2015}

\href{http://history.ccf.org.cn/resources/1190201776262/2015/11/12/18.pdf}{新技术 \& 新应用}

\href{http://history.ccf.org.cn/resources/1190201776262/2015/11/12/15.pdf}{40年的数据库之旅——VLDB 2015 会议报告}

\subsection{译文}
\href{http://history.ccf.org.cn/resources/1190201776262/2015/11/12/19.pdf}{增长中的关切:对人工智能的反思和前瞻}

\subsection{特别报道}
\href{http://history.ccf.org.cn/resources/1190201776262/2015/11/12/20.pdf}{CCF隆重颁发三大奖项}

\href{http://history.ccf.org.cn/resources/1190201776262/2015/11/12/23.pdf}{CCF的未来——新老理事建言学会发展}

\href{http://history.ccf.org.cn/resources/1190201776262/2015/11/12/22.pdf}{CCF第十一次会员代表大会召开}

\href{http://history.ccf.org.cn/resources/1190201776262/2015/11/12/21.pdf}{CCF优秀大学生奖颁奖大会在合肥举行}

\href{http://history.ccf.org.cn/resources/1190201776262/2015/11/12/1.pdf}{2015 中国计算机大会隆重举行}


\section{\href{http://history.ccf.org.cn/sites/ccf/jsjtbbd.jsp?contentId=2889262034952}{\textbf{2015年第10期(总第116期)}}}
专题《类脑计算》
无论是在计算方式上,还是在存储方式上,人脑比计算机体系结构都具有明显优势。这促使我们思考,通过对大脑的结构与工作原理进行模仿,是否能创造出更省电、更高效、更智能的计算系统?学术界和工业界都拉开了类脑计算研究的序幕。在本期专题中,我们邀请了多名国内外的知名专家撰文,介绍类脑计算领域的最新进展,讨论该领域面临的机遇和挑战。
\subsection{专题}
\href{http://history.ccf.org.cn/resources/1190201776262/2015/10/12/5.pdf}{机器学习加速器}

\href{http://history.ccf.org.cn/resources/1190201776262/2015/10/12/1.pdf}{类脑计算}

\href{http://history.ccf.org.cn/resources/1190201776262/2015/10/12/4.pdf}{神经拟态认知计算}

\href{http://history.ccf.org.cn/resources/1190201776262/2015/10/12/2.pdf}{神经拟态的类脑计算研究}

\href{http://history.ccf.org.cn/resources/1190201776262/2015/10/12/3.pdf}{基于新型纳米器件的类脑计算系统}

\subsection{专栏}
\href{http://history.ccf.org.cn/resources/1190201776262/2015/10/12/7.pdf}{人机对话浪潮:语音助手、聊天机器人、机器伴侣}

\href{http://history.ccf.org.cn/resources/1190201776262/2015/10/12/10.pdf}{“平民化”——大数据技术发展的新目标}

\href{http://history.ccf.org.cn/resources/1190201776262/2015/10/12/8.pdf}{如何与企业开展合作}

\href{http://history.ccf.org.cn/resources/1190201776262/2015/10/12/9.pdf}{大数据科学:数据、算法和数据驱动思维}

\href{http://history.ccf.org.cn/resources/1190201776262/2015/10/12/6.pdf}{从ACM会议分析我国计算机科学近十年发展情况}

\href{http://history.ccf.org.cn/resources/1190201776262/2015/10/12/11.pdf}{从学术会议中学习}

\subsection{动态}
\href{http://history.ccf.org.cn/resources/1190201776262/2015/10/12/14.pdf}{与工业界紧密联系的第26届USENIX技术年会}

\href{http://history.ccf.org.cn/resources/1190201776262/2015/10/12/13.pdf}{第21届高性能体系结构大会}

\href{http://history.ccf.org.cn/resources/1190201776262/2015/10/12/12.pdf}{KDD 2015——国际数据挖掘与知识发现大会首次澳洲巡演}

\href{http://history.ccf.org.cn/resources/1190201776262/2015/10/12/15.pdf}{新技术 \& 新应用}

\subsection{译文}
\href{http://history.ccf.org.cn/resources/1190201776262/2015/10/12/16.pdf}{神经拟态计算——有新灵魂的机器}


\section{\href{http://history.ccf.org.cn/sites/ccf/jsjtbbd.jsp?contentId=2883909046052}{\textbf{2015年第9期(总第115期)}}}
专题《车联网关键技术与应用》
如果车辆之间实现了互联,我们的驾驶生活会发生哪些变化?车联网给我们描绘了一个美好的场景:利用先进的传感、网络、计算、控制、智能等技术,对道路和交通进行全面感知,保证道路畅通,实现人、汽车和交通设施安全、高效地运转。本期专题邀请国内专家学者基于各自的科研成果,探讨了车联网关键技术在“交通安全”方面的应用和所面临的挑战。
\subsection{专题}
\href{http://history.ccf.org.cn/resources/1190201776262/2015/09/14/1.pdf}{车联网关键技术与应用}

\href{http://history.ccf.org.cn/resources/1190201776262/2015/09/14/3.pdf}{基于智能终端的参与式行驶安全系统}

\href{http://history.ccf.org.cn/resources/1190201776262/2015/09/14/5.pdf}{面向城市生活的车联网应用}

\href{http://history.ccf.org.cn/resources/1190201776262/2015/09/14/2.pdf}{面向主动安全的车联网系统}

\href{http://history.ccf.org.cn/resources/1190201776262/2015/09/14/4.pdf}{车联网V2R技术的研究进展}

\href{http://history.ccf.org.cn/resources/1190201776262/2015/09/14/6.pdf}{基于汽车行驶轨迹的路由方法研究进展}

\subsection{专栏}
\href{http://history.ccf.org.cn/resources/1190201776262/2015/09/14/13.pdf}{如何避免碌碌无为的感觉}

\href{http://history.ccf.org.cn/resources/1190201776262/2015/09/14/10.pdf}{机器人在中国的兴起}

\href{http://history.ccf.org.cn/resources/1190201776262/2015/09/14/14.pdf}{人工智能之争}

\href{http://history.ccf.org.cn/resources/1190201776262/2015/09/14/11.pdf}{产学研结合:机会与挑战}

\href{http://history.ccf.org.cn/resources/1190201776262/2015/09/14/9.pdf}{在快速成长的计算生态系统中开展教育和研究}

\href{http://history.ccf.org.cn/resources/1190201776262/2015/09/14/7.pdf}{变革中的大学计算机学科建设}

\href{http://history.ccf.org.cn/resources/1190201776262/2015/09/14/12.pdf}{论国际合作}

\href{http://history.ccf.org.cn/resources/1190201776262/2015/09/14/8.pdf}{“计算”学科与任意学科x交叉}

\subsection{动态}
\href{http://history.ccf.org.cn/resources/1190201776262/2015/09/14/16.pdf}{人工智能的“探戈”之旅——第24届国际人工智能联合会议}

\href{http://history.ccf.org.cn/resources/1190201776262/2015/09/14/15.pdf}{CCF A类会议报告——计算机网络顶级会议趋势分析}

\href{http://history.ccf.org.cn/resources/1190201776262/2015/09/14/17.pdf}{面向学术界与工业界融合的FAST 2015}

\href{http://history.ccf.org.cn/resources/1190201776262/2015/09/14/18.pdf}{新技术与新应用}

\subsection{译文}
\href{http://history.ccf.org.cn/resources/1190201776262/2015/09/14/19.pdf}{永不失忆的计算机芯片}

\subsection{学会论坛}
\href{http://history.ccf.org.cn/resources/1190201776262/2015/09/14/20.pdf}{专委选举是一份考卷}


\section{\href{http://history.ccf.org.cn/sites/ccf/jsjtbbd.jsp?contentId=2878259748700}{\textbf{2015年第8期(总第114期)}}}
专题《深度学习与媒体计算》:新闻网站、微博、微信、社交网络、图像视频共享网站每天都会产生大量媒体数据。媒体数据的来源渠道广、内容多样化、需求多元化、计算复杂化等特点给媒体计算带来了极大挑战。深度学习技术具有强大的数据抽象能力,它为媒体计算领域的研究带来了哪些机遇?物体识别、图像分类、语音识别等领域又取得了哪些突破?本期专题邀请了学术界和企业界的专家撰文,结合各自领域的最新研究成果,对深度学习和媒体计算的未来发展趋势进行了阐述。
\subsection{专题}
\href{http://history.ccf.org.cn/resources/1190201776262/2015/08/12/4.pdf}{语音信号与信息处理中的深度学习}

\href{http://history.ccf.org.cn/resources/1190201776262/2015/08/12/5.pdf}{深度匹配学习在语言匹配中的应用}

\href{http://history.ccf.org.cn/resources/1190201776262/2015/08/11/1.pdf}{深度学习与媒体计算}

\href{http://history.ccf.org.cn/resources/1190201776262/2015/08/12/3.pdf}{图像识别中的深度学习}

\href{http://history.ccf.org.cn/resources/1190201776262/2015/08/11/2.pdf}{媒体计算新进展与挑战}

\subsection{专栏}
\href{http://history.ccf.org.cn/resources/1190201776262/2015/08/12/6.pdf}{CCF《国际学术会议和期刊目录》得大于失}

\href{http://history.ccf.org.cn/resources/1190201776262/2015/08/12/7.pdf}{我国计算机学科国际期刊论文状况}

\href{http://history.ccf.org.cn/resources/1190201776262/2015/08/12/8.pdf}{为CCF《国际学术会议和期刊目录》点赞}

\href{http://history.ccf.org.cn/resources/1190201776262/2015/08/12/9.pdf}{智能的进化与博弈}

\href{http://history.ccf.org.cn/resources/1190201776262/2015/08/12/10.pdf}{没有高利润的“高技术”不是高技术}

\href{http://history.ccf.org.cn/resources/1190201776262/2015/08/12/11.pdf}{编程人员离不开面对面交流}

\subsection{动态}
\href{http://history.ccf.org.cn/resources/1190201776262/2015/08/12/12.pdf}{CVPR 2015}

\href{http://history.ccf.org.cn/resources/1190201776262/2015/08/12/13.pdf}{从MobiCom看技术发展趋势}

\href{http://history.ccf.org.cn/resources/1190201776262/2015/08/12/14.pdf}{新技术 \& 新应用}

\subsection{译文}
\href{http://history.ccf.org.cn/resources/1190201776262/2015/08/12/15.pdf}{深度学习:成长的烦恼}


\section{\href{http://history.ccf.org.cn/sites/ccf/jsjtbbd.jsp?contentId=2872975896179}{\textbf{2015年第7期(总第113期)}}}
专题《2014 CCF青年科学家奖得主驰骋的领域》CCF青年科学家奖获得者是青年学者中的佼佼者,他们令人羡慕、钦佩。他们到底有多神秘?他们做出了怎样的突出成就?他们所在的领域发展如何?本期专题邀请2014年CCF青年科学家奖获得者陈云霁、李国良、陆品燕结合自己的科研工作对相关学科的发展进行了阐述和展望。他们站在较高的视角俯瞰自己耕耘的科研领地,深入浅出地描述了新的学术方向,相信能开拓读者的视野,给人启迪。
\subsection{专题}
\href{http://history.ccf.org.cn/resources/1190201776262/2015/07/13/4.pdf}{理论计算机}

\href{http://history.ccf.org.cn/resources/1190201776262/2015/07/13/2.pdf}{体系结构研究者眼中的神经网络硬件}

\href{http://history.ccf.org.cn/resources/1190201776262/2015/07/13/3.pdf}{人机协作的群体计算}

\href{http://history.ccf.org.cn/resources/1190201776262/2015/07/13/1.pdf}{2014 CCF青年科学家奖得主驰骋的领域}

\subsection{专栏}
\href{http://history.ccf.org.cn/resources/1190201776262/2015/07/13/8.pdf}{我为什么鼓励你读博士}

\href{http://history.ccf.org.cn/resources/1190201776262/2015/07/13/6.pdf}{发表论文只是研究的一种形式而不是目的}

\href{http://history.ccf.org.cn/resources/1190201776262/2015/07/13/10.pdf}{机器翻译,让语言交流无障碍}

\href{http://history.ccf.org.cn/resources/1190201776262/2015/07/13/5.pdf}{中国计算机学会发布《国际学术会议和期刊目录》得失谈}

\href{http://history.ccf.org.cn/resources/1190201776262/2015/07/13/11.pdf}{“我们相信电子商务”}

\href{http://history.ccf.org.cn/resources/1190201776262/2015/07/13/7.pdf}{打造学术绿洲 推进科教兴国}

\href{http://history.ccf.org.cn/resources/1190201776262/2015/07/13/9.pdf}{让硕士生活更加精彩}

\subsection{动态}
\href{http://history.ccf.org.cn/resources/1190201776262/2015/07/13/14.pdf}{第24届国际万维网大会精彩纷呈}

\href{http://history.ccf.org.cn/resources/1190201776262/2015/07/13/15.pdf}{新技术 \& 新应用}

\href{http://history.ccf.org.cn/resources/1190201776262/2015/07/13/13.pdf}{SIGMOD 2015}

\href{http://history.ccf.org.cn/resources/1190201776262/2015/07/13/12.pdf}{兴趣是最好的原动力——访香港中文大学吕自成教授}

\subsection{译文}
\href{http://history.ccf.org.cn/resources/1190201776262/2015/07/13/16.pdf}{不借助参考标准的社会媒体研究评价}

\subsection{学会论坛}
\href{http://history.ccf.org.cn/resources/1190201776262/2015/07/13/17.pdf}{开放式选举是会员治理的重要体现}


\section{\href{http://history.ccf.org.cn/sites/ccf/jsjtbbd.jsp?contentId=2867635215085}{\textbf{2015年第5期(总第112期)}}}
专题《未来互联网体系结构探讨》:未来互联网如何构建?这是世界计算机领域科技人员十分关注的重大问题。本期专题对未来互联网体系结构进行了探讨,李国杰主编也在开篇谈了未来互联网向何处去。互联网是渗透到各行各业的庞大的基础设施,发展得如何,未来互联网的研究考验着我们今天判断宏观市场和洞察未来的眼力。或许吴建平、罗洪斌教授等人的文章会给我们一些启示?
\subsection{专题}
\href{http://history.ccf.org.cn/resources/1190201776262/2015/06/12/1.pdf}{未来互联网向何处去?}

\href{http://history.ccf.org.cn/resources/1190201776262/2015/06/12/4.pdf}{绿色节能的可重构网络体系结构}

\href{http://history.ccf.org.cn/resources/1190201776262/2015/06/12/5.pdf}{面向服务的信息中心网络体系结构}

\href{http://history.ccf.org.cn/resources/1190201776262/2015/06/12/2.pdf}{ADN:地址驱动的网络体系结构}

面向应用的可演进互联网体系结构

\href{http://history.ccf.org.cn/resources/1190201776262/2015/06/12/3.pdf}{前后向兼容的未来互联网体系结构}

\subsection{专栏}
\href{http://history.ccf.org.cn/resources/1190201776262/2015/06/12/10.pdf}{“急于求成”?}

\href{http://history.ccf.org.cn/resources/1190201776262/2015/06/12/7.pdf}{奇点理论:是关于人工智能的科学理论?还是垃圾科学?}

\href{http://history.ccf.org.cn/resources/1190201776262/2015/06/12/8.pdf}{关于读博,关于成为一个专家}

\href{http://history.ccf.org.cn/resources/1190201776262/2015/06/12/9.pdf}{“互联网+” = 新一代ICT+创新2.0}

\subsection{动态}
\href{http://history.ccf.org.cn/resources/1190201776262/2015/06/12/13.pdf}{ICDE 2015给我们带来了什么}

\href{http://history.ccf.org.cn/resources/1190201776262/2015/06/12/14.pdf}{全新面孔的INFOCOM}

\href{http://history.ccf.org.cn/resources/1190201776262/2015/06/12/15.pdf}{新技术 \& 新应用}

\href{http://history.ccf.org.cn/resources/1190201776262/2015/06/12/11.pdf}{成功很少青睐一个孤独的人——访2014 CCF海外杰出贡献奖获得者张可昭教授}

\href{http://history.ccf.org.cn/resources/1190201776262/2015/06/12/12.pdf}{由数据库界图灵奖得主说发展}

\subsection{译文}
\href{http://history.ccf.org.cn/resources/1190201776262/2015/06/12/16.pdf}{戈登·摩尔:他的名字意味着发展——86岁有远见的工程师回望50岁的摩尔定律}


\section{\href{http://history.ccf.org.cn/sites/ccf/jsjtbbd.jsp?contentId=2867635215085}{\textbf{2015年第6期(总第112期)}}}
专题《未来互联网体系结构探讨》:未来互联网如何构建?这是世界计算机领域科技人员十分关注的重大问题。本期专题对未来互联网体系结构进行了探讨,李国杰主编也在开篇谈了未来互联网向何处去。互联网是渗透到各行各业的庞大的基础设施,发展得如何,未来互联网的研究考验着我们今天判断宏观市场和洞察未来的眼力。或许吴建平、罗洪斌教授等人的文章会给我们一些启示?
\subsection{专题}
\href{http://history.ccf.org.cn/resources/1190201776262/2015/06/12/1.pdf}{未来互联网向何处去?}

\href{http://history.ccf.org.cn/resources/1190201776262/2015/06/12/4.pdf}{绿色节能的可重构网络体系结构}

\href{http://history.ccf.org.cn/resources/1190201776262/2015/06/12/5.pdf}{面向服务的信息中心网络体系结构}

\href{http://history.ccf.org.cn/resources/1190201776262/2015/06/12/2.pdf}{ADN:地址驱动的网络体系结构}

\href{http://history.ccf.org.cn/resources/1190201776262/2015/06/23/6.pdf}{面向应用的可演进互联网体系结构}

\href{http://history.ccf.org.cn/resources/1190201776262/2015/06/12/3.pdf}{前后向兼容的未来互联网体系结构}

\subsection{专栏}
\href{http://history.ccf.org.cn/resources/1190201776262/2015/06/12/10.pdf}{“急于求成”?}

\href{http://history.ccf.org.cn/resources/1190201776262/2015/06/12/7.pdf}{奇点理论:是关于人工智能的科学理论?还是垃圾科学?}

\href{http://history.ccf.org.cn/resources/1190201776262/2015/06/12/8.pdf}{关于读博,关于成为一个专家}

\href{http://history.ccf.org.cn/resources/1190201776262/2015/06/12/9.pdf}{“互联网+” = 新一代ICT+创新2.0}

\subsection{动态}
\href{http://history.ccf.org.cn/resources/1190201776262/2015/06/12/13.pdf}{ICDE 2015给我们带来了什么}

\href{http://history.ccf.org.cn/resources/1190201776262/2015/06/12/14.pdf}{全新面孔的INFOCOM}

\href{http://history.ccf.org.cn/resources/1190201776262/2015/06/12/15.pdf}{新技术 \& 新应用}

\href{http://history.ccf.org.cn/resources/1190201776262/2015/06/12/11.pdf}{成功很少青睐一个孤独的人——访2014 CCF海外杰出贡献奖获得者张可昭教授}

\href{http://history.ccf.org.cn/resources/1190201776262/2015/06/12/12.pdf}{由数据库界图灵奖得主说发展}

\subsection{译文}
\href{http://history.ccf.org.cn/resources/1190201776262/2015/06/12/16.pdf}{戈登·摩尔:他的名字意味着发展——86岁有远见的工程师回望50岁的摩尔定律}


\section{\href{http://history.ccf.org.cn/sites/ccf/jsjtbbd.jsp?contentId=2860516747201}{\textbf{2015年第5期(总第111期)}}}
专题《展望新智能时代》:人工智能始终处于不断向前推进的计算机技术的前沿,互联网的普及和大数据的兴起再次将人工智能技术推向新的高峰。“谷歌大脑”、“欧洲大脑”、“美国大脑”等人工智能概念相继提出,我国也正在酝酿启动“中国脑计划”。然而,尽管人工智能是当前的热门,本期作者杨强教授认为现在仍处于从1到N的时代。王飞跃、余凯、胡郁、白硕、汤兴等也撰文,讲了他们理解的人工智能、发展历程和未来前景。
\subsection{专题}
\href{http://history.ccf.org.cn/resources/1190201776262/2015/05/12/6.pdf}{自然语言处理与人工智能}

\href{http://history.ccf.org.cn/resources/1190201776262/2015/05/12/7.pdf}{爱奇艺大脑——视频进化}

\href{http://history.ccf.org.cn/resources/1190201776262/2015/05/12/1.pdf}{展望新智能时代}

\href{http://history.ccf.org.cn/resources/1190201776262/2015/05/12/3.pdf}{有温度的人工智能}

\href{http://history.ccf.org.cn/resources/1190201776262/2015/05/12/4.pdf}{后图灵时代:一个从1到N的时代}

\href{http://history.ccf.org.cn/resources/1190201776262/2015/05/12/5.pdf}{从感知智能到认知智能}

\href{http://history.ccf.org.cn/resources/1190201776262/2015/05/12/2.pdf}{X5.0:平行时代的平行智能体系}

\subsection{专栏}
\href{http://history.ccf.org.cn/resources/1190201776262/2015/05/12/8.pdf}{数据中心成本与利用率现状分析}

\href{http://history.ccf.org.cn/resources/1190201776262/2015/05/13/13.pdf}{“类脑计算”}

\href{http://history.ccf.org.cn/resources/1190201776262/2015/05/12/10.pdf}{工业界 vs. 学术界:一个年轻员工的视角}

\href{http://history.ccf.org.cn/resources/1190201776262/2015/05/13/11.pdf}{问 路}

\href{http://history.ccf.org.cn/resources/1190201776262/2015/05/12/9.pdf}{OCR:慧眼读世界}

\href{http://history.ccf.org.cn/resources/1190201776262/2015/05/13/12.pdf}{“芯片限售对我国超算的影响”}

\subsection{视点}
\href{http://history.ccf.org.cn/resources/1190201776262/2015/05/13/14.pdf}{物理-网络空间中的群体情绪认知与推演}

\subsection{动态}
\href{http://history.ccf.org.cn/resources/1190201776262/2015/05/13/16.pdf}{坚持大学的使命,与时俱进培养优秀人才——访“2014 CCF杰出教育奖”获得者李晓明教授}

\href{http://history.ccf.org.cn/resources/1190201776262/2015/05/13/17.pdf}{第33届ACM人机交互大会}

\href{http://history.ccf.org.cn/resources/1190201776262/2015/05/13/18.pdf}{新技术 \& 新应用}

\href{http://history.ccf.org.cn/resources/1190201776262/2015/05/13/15.pdf}{工科出身、有理科情结的哲学思考者 ——访“2014 CCF青年科学家奖”获得者陆品燕博士}

\subsection{译文}
\href{http://history.ccf.org.cn/resources/1190201776262/2015/05/13/19.pdf}{计算研究者的新责任}

\subsection{学会论坛}
\href{http://history.ccf.org.cn/resources/1190201776262/2015/05/13/20.pdf}{计算研究者的新责任}


\section{\href{http://history.ccf.org.cn/sites/ccf/jsjtbbd.jsp?contentId=2857793951402}{\textbf{2015年第4期(总第110期)}}}
计算机视觉既是科学领域中富有挑战性的理论研究,也是工程领域中的重要应用,它已成为“谷歌大脑”、“百度大脑”等研究计划中的核心项目。国内科研单位也投入巨大力量开展相关研究。为推动国内计算机视觉学科发展,提升我国计算机视觉研究的国际影响力,CCF成立了“计算机视觉专业组”。在本期专题中,计算机视觉专业组特别邀请多位著名专家从不同角度撰文,介绍计算机视觉前沿与深度学习研究方面的最新进展。
\subsection{专题}
\href{http://history.ccf.org.cn/resources/1190201776262/2015/04/13/2.pdf}{从统一子空间分析到联合深度学习:人脸识别的十年历程}

\href{http://history.ccf.org.cn/resources/1190201776262/2015/04/13/3.pdf}{深度学习在人脸分析与识别中的应用}

\href{http://history.ccf.org.cn/resources/1190201776262/2015/04/13/6.pdf}{基于深度学习的图像识别进展:百度的若干实践}

\href{http://history.ccf.org.cn/resources/1190201776262/2015/04/13/1.pdf}{计算机视觉前沿与深度学习}

\href{http://history.ccf.org.cn/resources/1190201776262/2015/04/13/4.pdf}{信号与数据处理中的低秩模型}

\href{http://history.ccf.org.cn/resources/1190201776262/2015/04/13/5.pdf}{视觉局部特征的表达学习}

\subsection{专栏}
\href{http://history.ccf.org.cn/resources/1190201776262/2015/04/13/10.pdf}{产业互联网,寻找产业界的BAT}

\href{http://history.ccf.org.cn/resources/1190201776262/2015/04/13/7.pdf}{可穿戴技术的机会与挑战}

\href{http://history.ccf.org.cn/resources/1190201776262/2015/04/13/11.pdf}{模仿人生}

\href{http://history.ccf.org.cn/resources/1190201776262/2015/04/13/8.pdf}{图灵奖得主CCF论文发表情况分析}

\href{http://history.ccf.org.cn/resources/1190201776262/2015/04/13/9.pdf}{我们的科研之路何去何从?}

\subsection{动态}
\href{http://history.ccf.org.cn/resources/1190201776262/2015/04/13/12.pdf}{科研的一万小时—— 访“2014 CCF青年科学家奖”获得者李国良博士}

\href{http://history.ccf.org.cn/resources/1190201776262/2015/04/13/13.pdf}{访多媒体领域创始人之一拉梅什·杰恩教授}

\href{http://history.ccf.org.cn/resources/1190201776262/2015/04/13/14.pdf}{第三十五届IEEE实时系统会议}

\href{http://history.ccf.org.cn/resources/1190201776262/2015/04/13/15.pdf}{第二十九届AAAI人工智能会议}

\subsection{译文}
\href{http://history.ccf.org.cn/resources/1190201776262/2015/04/13/16.pdf}{对话深度学习专家雅恩·乐昆:让深度学习摆脱束缚}


\section{\href{http://history.ccf.org.cn/sites/ccf/jsjtbbd.jsp?contentId=2851766225495}{\textbf{2015年第3期(总第109期)}}}
什么是奖励的本质?什么奖励会异化?奖励本质是让受奖者得到应有的荣誉,让同行或同类产生敬意;而奖励的异化会使奖励与利益挂钩、评选主体错位、程序失当,产生不公正等。奖励反映的是一个组织或国家的价值取向,反映的是文化。本期专栏刊登了CCF海外理事张晓东教授、CCF秘书长杜子德的文章《奖励的本质和奖励的异化》。
\subsection{专题}
\href{http://history.ccf.org.cn/resources/1190201776262/2015/03/12/1.pdf}{多智能自然语言处理}

\href{http://history.ccf.org.cn/resources/1190201776262/2015/03/12/2.pdf}{深度学习在自然语言处理中的应用}

\href{http://history.ccf.org.cn/resources/1190201776262/2015/03/12/5.pdf}{基于社会媒体的预测技术}

\href{http://history.ccf.org.cn/resources/1190201776262/2015/03/12/4.pdf}{拥抱社会智能}

\href{http://history.ccf.org.cn/resources/1190201776262/2015/03/12/3.pdf}{从问答系统看知识智能}

\subsection{专栏}
\href{http://history.ccf.org.cn/resources/1190201776262/2015/03/12/8.pdf}{关于图灵奖获得者的一些统计}

\href{http://history.ccf.org.cn/resources/1190201776262/2015/03/12/10.pdf}{黑客精神与开放架构:个人电脑40年}

\href{http://history.ccf.org.cn/resources/1190201776262/2015/03/12/6.pdf}{奖励的本质和奖励的异化}

\href{http://history.ccf.org.cn/resources/1190201776262/2015/03/12/9.pdf}{计算机跨界研究之心得体会}

\href{http://history.ccf.org.cn/resources/1190201776262/2015/03/12/11.pdf}{建设“发展 - 安全一体”的国家网络安全战略}

\href{http://history.ccf.org.cn/resources/1190201776262/2015/03/12/12.pdf}{谈一谈CMU导师和学生的互动方式}

\href{http://history.ccf.org.cn/resources/1190201776262/2015/03/12/7.pdf}{漫谈错误的技术预测}

\href{http://history.ccf.org.cn/resources/1190201776262/2015/03/12/13.pdf}{收益率}

\subsection{动态}
\href{http://history.ccf.org.cn/resources/1190201776262/2015/03/12/14.pdf}{选择一个让我激动、又有能力作出贡献的方向——访“2014 CCF青年科学家奖”获得者陈云霁博士}

\href{http://history.ccf.org.cn/resources/1190201776262/2015/03/12/16.pdf}{第八届互联网搜索与数据挖掘国际会议}

\href{http://history.ccf.org.cn/resources/1190201776262/2015/03/12/15.pdf}{自然语言处理的发展趋势——访卡内基梅隆大学爱德华·霍威教授}

\subsection{译文}
\href{http://history.ccf.org.cn/resources/1190201776262/2015/03/12/17.pdf}{交互式搜索意图理解:超越传统搜索的信息发现}


\section{\href{http://history.ccf.org.cn/sites/ccf/jsjtbbd.jsp?contentId=2845154181957}{\textbf{2015年第2期(总第108期)}}}
1月31日,CCF在北京召开2014颁奖大会。会上隆重颁发了CCF终身成就奖、CCF计算机企业家奖、CCF杰出女计算机工作者奖、CCF青年科学家奖、CCF优秀博士学位论文奖、CCF杰出教育奖、CCF杰出贡献奖、CCF卓越服务奖等。来自全国各地的CCF理事、CCF各工作委员会、各专业委员会、CCCF编委、YOCSEF委员以及科研机构、企业、媒体和相关学会代表300多人应邀出席。CCF秘书长杜子德主持大会,CCF理事长郑纬民在大会上致辞。
\subsection{专题}
\href{http://history.ccf.org.cn/resources/1190201776262/2015/02/12/5.pdf}{产品设计流程的可信表示和系统}

\href{http://history.ccf.org.cn/resources/1190201776262/2015/02/12/6.pdf}{基于CAD/CAE一体化的复杂产品快速设计计算方法}

\href{http://history.ccf.org.cn/resources/1190201776262/2015/02/12/2.pdf}{现代产品设计大型应用软件的可信性问题}

\href{http://history.ccf.org.cn/resources/1190201776262/2015/02/12/3.pdf}{产品设计的精度问题和求解}

\href{http://history.ccf.org.cn/resources/1190201776262/2015/02/12/4.pdf}{面向CAD/CAE一体化的产品自适应离散造型方法}

\subsection{专栏}
\href{http://history.ccf.org.cn/resources/1190201776262/2015/02/12/8.pdf}{鼓励创业,破解“青椒”困局}

\href{http://history.ccf.org.cn/resources/1190201776262/2015/02/12/9.pdf}{机器人做实验}

\href{http://history.ccf.org.cn/resources/1190201776262/2015/02/12/11.pdf}{共同的价值观}

\href{http://history.ccf.org.cn/resources/1190201776262/2015/02/12/7.pdf}{计算机教育,从CC 2001到CS 2013,以及未来}

\href{http://history.ccf.org.cn/resources/1190201776262/2015/02/12/10.pdf}{什么是“渔”?}

\subsection{动态}
\href{http://history.ccf.org.cn/resources/1190201776262/2015/02/12/12.pdf}{2014神经信息处理系统国际会议}

\subsection{译文}
\href{http://history.ccf.org.cn/resources/1190201776262/2015/02/12/13.pdf}{指令系统应该免费:RISC-V的案例}

\subsection{特别报道}
\href{http://history.ccf.org.cn/resources/1190201776262/2015/02/12/1.pdf}{2014 CCF颁奖大会在京举行}


\section{\href{http://history.ccf.org.cn/sites/ccf/jsjtbbd.jsp?contentId=2842044881493}{\textbf{2015年第1期(总第107期)}}}
通过提高路由交换设备的开放性和可编程能力来满足网络创新需求,是解决网络僵化问题的新思路。开放式可编程的技术路线也给网络攻击者提供了更多可乘之机。本期专题邀请中国工程院院士、高等院校学者、华为和中兴通讯公司的技术人员撰写文章,介绍路由交换平台与关键技术在支撑网络服务创新和信息安全方面的最新研究进展,希望他们能给您带来一场别样的视觉盛宴。
\subsection{专题}
\href{http://history.ccf.org.cn/resources/1190201776262/2015/01/13/5.pdf}{基于路由交换范式构建安全可信网络}

\href{http://history.ccf.org.cn/resources/1190201776262/2015/01/13/6.pdf}{从协议无感知转发到OpenFlow 2.0}

\href{http://history.ccf.org.cn/resources/1190201776262/2015/01/13/2.pdf}{拟态计算与拟态安全防御}

\href{http://history.ccf.org.cn/resources/1190201776262/2015/01/12/1.pdf}{面向服务创新与信息安全的路由交换技术}

\href{http://history.ccf.org.cn/resources/1190201776262/2015/01/13/3.pdf}{网络虚拟化:概念、应用与挑战}

\href{http://history.ccf.org.cn/resources/1190201776262/2015/01/12/4.pdf}{可编程网络实验平台研究进展}

\href{http://history.ccf.org.cn/resources/1190201776262/2015/01/12/7.pdf}{柔性重构路由交换平台}

\subsection{专栏}
\href{http://history.ccf.org.cn/resources/1190201776262/2015/01/12/10.pdf}{拥抱数据共享与代码开源的新时代}

\href{http://history.ccf.org.cn/resources/1190201776262/2015/01/12/11.pdf}{计算机中的“魅影”}

\href{http://history.ccf.org.cn/resources/1190201776262/2015/01/12/8.pdf}{2015年大数据发展趋势预测}

\href{http://history.ccf.org.cn/resources/1190201776262/2015/01/12/9.pdf}{当前科研成果转化中的种种误区}

\subsection{视点}
\href{http://history.ccf.org.cn/resources/1190201776262/2015/01/13/12.pdf}{从软件研究者的视角认识“软件定义”}

\subsection{动态}
\href{http://history.ccf.org.cn/resources/1190201776262/2015/01/12/13.pdf}{SIGGRAPH Asia 2014}

\href{http://history.ccf.org.cn/resources/1190201776262/2015/01/12/14.pdf}{IEEE VIS 2014大会}

\subsection{译文}
\href{http://history.ccf.org.cn/resources/1190201776262/2015/01/13/15.pdf}{一种新的软件工程}


\section{\href{http://history.ccf.org.cn/sites/ccf/jsjtbbd.jsp?contentId=2836201103891}{\textbf{2014年第12期(总第106期)}}}
你可否聆听过图灵奖得主伊凡·苏泽兰的报告?可否聆听过方滨兴院士的报告?可否聆听过雷军的报告?如果你参加CNCC,即可现场聆听。10月23~25日,CNCC 2014特邀了多位学术界与企业界的“大牛”作报告。如果你没有聆听到他们的现场报告,在此辑CCCF中,你可读到他们的文章。
\subsection{专题}
\href{http://history.ccf.org.cn/resources/1190201776262/2014/12/11/2.pdf}{异构时代:挑战和机遇}

\href{http://history.ccf.org.cn/resources/1190201776262/2014/12/11/4.pdf}{小米成功密码}

\href{http://history.ccf.org.cn/resources/1190201776262/2014/12/11/6.pdf}{互联网时代的大数据技术}

\href{http://history.ccf.org.cn/resources/1190201776262/2014/12/11/7.pdf}{可信平台的新范式}

\href{http://history.ccf.org.cn/resources/1190201776262/2014/12/11/1.pdf}{计算机图形学的发展历程}

\href{http://history.ccf.org.cn/resources/1190201776262/2014/12/11/3.pdf}{从网络主权角度谈自主根域名体系}

\href{http://history.ccf.org.cn/resources/1190201776262/2014/12/11/5.pdf}{从大数据到认知计算}

\subsection{专栏}
\href{http://history.ccf.org.cn/resources/1190201776262/2014/12/11/10.pdf}{关于会议的“协议”}

\href{http://history.ccf.org.cn/resources/1190201776262/2014/12/11/8.pdf}{智能手机应用的能耗与性能问题诊断}

\href{http://history.ccf.org.cn/resources/1190201776262/2014/12/11/9.pdf}{研究生要具备什么能力}

\subsection{视点}
\href{http://history.ccf.org.cn/resources/1190201776262/2014/12/11/11.pdf}{大数据:系统遇上机器学习}

\href{http://history.ccf.org.cn/resources/1190201776262/2014/12/11/12.pdf}{普适计算的新趋势:柔性电子时代}

\subsection{动态}
\href{http://history.ccf.org.cn/resources/1190201776262/2014/12/11/13.pdf}{专访C++发明人Bjarne Stroustrup}

\href{http://history.ccf.org.cn/resources/1190201776262/2014/12/11/14.pdf}{跨越真实和虚拟世界的边界——走近SIGGRAPH 2014大会}

\href{http://history.ccf.org.cn/resources/1190201776262/2014/12/11/15.pdf}{ACM Multimedia 2014}

\href{http://history.ccf.org.cn/resources/1190201776262/2014/12/11/16.pdf}{ACM CIKM 2014}

\subsection{译文}
\href{http://history.ccf.org.cn/resources/1190201776262/2014/12/11/18.pdf}{机器学习大家迈克尔·乔丹谈大数据可能只是一场空欢喜等}

\subsection{学会论坛}
\href{http://history.ccf.org.cn/resources/1190201776262/2014/12/12/17.pdf}{学会的核心是认同}


\section{\href{http://history.ccf.org.cn/sites/ccf/jsjtbbd.jsp?contentId=2831644844360}{\textbf{2014年第11期(总第105期)}}}
CNCC 2014 盛大召开。2014 中国计算机大会(CNCC)在郑州隆重举行。来自国内外学术界、企业界、政界、新闻界逾3000人参加了大会。大会邀请了图灵奖获得者、著名企业家、著名学者等嘉宾作报告;隆重颁发了CCF王选奖、海外杰出贡献奖、科技奖、优秀大学生奖;举办了11个专题论坛;举行了约40个会议、活动、座谈会;有逾50个单位参加科技成果展⋯⋯本刊对这场年度盛会进行了特别报道。
\subsection{专题}
\href{http://history.ccf.org.cn/resources/1190201776262/2014/11/13/1.pdf}{面向大数据的存储与处理}

\href{http://history.ccf.org.cn/resources/1190201776262/2014/11/13/3.pdf}{忆阻器:融合大数据计算与存储的“算珠”}

\href{http://history.ccf.org.cn/resources/1190201776262/2014/11/13/2.pdf}{大数据存储与处理关键技术}

\href{http://history.ccf.org.cn/resources/1190201776262/2014/11/13/5.pdf}{面向大数据的在线学习算法}

\href{http://history.ccf.org.cn/resources/1190201776262/2014/11/13/4.pdf}{OpenKN——网络大数据时代的知识计算引擎}

\subsection{专栏}
\href{http://history.ccf.org.cn/resources/1190201776262/2014/11/13/7.pdf}{浅谈剑桥大学的博士生培养机制和文化特点}

\href{http://history.ccf.org.cn/resources/1190201776262/2014/11/13/9.pdf}{Wi-Fi雷达:从RSSI到CSI}

\href{http://history.ccf.org.cn/resources/1190201776262/2014/11/13/11.pdf}{4G时代的穿戴式终端与智慧健康}

\href{http://history.ccf.org.cn/resources/1190201776262/2014/11/13/10.pdf}{并行思考}

\href{http://history.ccf.org.cn/resources/1190201776262/2014/11/13/6.pdf}{科技评价漫谈}

\href{http://history.ccf.org.cn/resources/1190201776262/2014/11/13/8.pdf}{我们需要什么样的机器人}

\subsection{视点}
\href{http://history.ccf.org.cn/resources/1190201776262/2014/11/13/12.pdf}{解密接近人脑的智能学习机器——深度学习及并行化实现}

\subsection{动态}
\href{http://history.ccf.org.cn/resources/1190201776262/2014/11/13/13.pdf}{ACM MobiCom 2014}

\subsection{译文}
\href{http://history.ccf.org.cn/resources/1190201776262/2014/11/13/14.pdf}{机器学习大家迈克尔·乔丹谈大数据等重大技术探索的迷思}

\subsection{特别报道}
\href{http://history.ccf.org.cn/resources/1190201776262/2014/11/13/0.pdf}{CNCC 2014在郑州举行}


\section{\href{http://history.ccf.org.cn/sites/ccf/jsjtbbd.jsp?contentId=2826288041117}{\textbf{2014年第10期(总第104期)}}}
特别报道:深切怀念夏培肃先生。8月27日11时10分,中国科学院资深院士、中国计算机事业的创始人、中科院计算所研究员夏培肃先生与世长辞。夏先生的家人、朋友、同事、学生都无比悲痛,纷纷撰文怀念夏先生。李国杰、郑纬民、杜子德、唐志敏、侯紫峰、胡伟武、祝明发在文章中回忆了做夏先生学生期间或与夏先生接触过程中的点点滴滴,体现了夏先生对工作的精益求精以及对学生无微不至的关怀与照顾。夏先生永远活在我们心中!
\subsection{专题}
\href{http://history.ccf.org.cn/resources/1190201776262/2014/10/13/13.pdf}{金融定价、交易与资产配置的算法视角}

\href{http://history.ccf.org.cn/resources/1190201776262/2014/10/13/8.pdf}{计算金融学和算法交易}

\href{http://history.ccf.org.cn/resources/1190201776262/2014/10/13/9.pdf}{大数据驱动下的金融模拟交易平台}

\href{http://history.ccf.org.cn/resources/1190201776262/2014/10/13/10.pdf}{算法交易系统基本架构与学习算法}

\href{http://history.ccf.org.cn/resources/1190201776262/2014/10/13/11.pdf}{投资选择之在线学习方法}

\href{http://history.ccf.org.cn/resources/1190201776262/2014/10/13/12.pdf}{互联网时代金融市场预测之挑战}

\subsection{专栏}
\href{http://history.ccf.org.cn/resources/1190201776262/2014/10/13/15.pdf}{今人不见古时月,古月依旧照今人}

\href{http://history.ccf.org.cn/resources/1190201776262/2014/10/13/16.pdf}{切实负起责任}

\href{http://history.ccf.org.cn/resources/1190201776262/2014/10/13/14.pdf}{技术人员创业的种种误区}

\subsection{动态}
\href{http://history.ccf.org.cn/resources/1190201776262/2014/10/13/22.pdf}{SIGGRAPH Asia 2014将在深圳举行}

\href{http://history.ccf.org.cn/resources/1190201776262/2014/10/13/19.pdf}{SIGMOD 2014}

\href{http://history.ccf.org.cn/resources/1190201776262/2014/10/14/nami.pdf}{纳米网络:从底层出发}

\href{http://history.ccf.org.cn/resources/1190201776262/2014/10/13/20.pdf}{书评——周鸿祎自述:我的互联网方法论}

\href{http://history.ccf.org.cn/resources/1190201776262/2014/10/13/17.pdf}{SIGKDD二十周年庆典}

\href{http://history.ccf.org.cn/resources/1190201776262/2014/10/13/21.pdf}{书评——周宏桥:跨界引爆创新}

\subsection{译文}
\href{http://history.ccf.org.cn/resources/1190201776262/2014/10/13/23.pdf}{慢搜索}

\subsection{特别报道}
\href{http://history.ccf.org.cn/resources/1190201776262/2014/10/13/2.pdf}{回忆夏先生的一件事}

\href{http://history.ccf.org.cn/resources/1190201776262/2014/10/13/4.pdf}{严谨的导师 慈蔼的长者——怀念我的恩师夏培肃先生}

\href{http://history.ccf.org.cn/resources/1190201776262/2014/10/13/5.pdf}{精益求精 高风亮节——忆恩师夏培肃先生}

\href{http://history.ccf.org.cn/resources/1190201776262/2014/10/13/7.pdf}{永远怀念我的职业生涯领路人夏培肃先生}

\href{http://history.ccf.org.cn/resources/1190201776262/2014/10/13/1.pdf}{深切怀念夏培肃老师}

\href{http://history.ccf.org.cn/resources/1190201776262/2014/10/13/3.pdf}{怀念夏培肃老师}

\href{http://history.ccf.org.cn/resources/1190201776262/2014/10/13/6.pdf}{学高为师 身正为范——怀念我的导师夏培肃先生}


\section{\href{http://history.ccf.org.cn/sites/ccf/jsjtbbd.jsp?contentId=2820931243508}{\textbf{2014年第9期(总第103期)}}}
本期专题是智能体和多智能体系统研究。1950年,阿兰·图灵提出了著名的“图灵测试”。自此,“智能体”的概念便开始兴起。经过几十年的发展,智能体和多智能体系统的研究热度依然不减,而且越来越有生命力,它反映了自然界和人类社会的一些本质。该领域最近有哪些新的研究成果?其研究热点、面临的挑战有哪些?发展趋势又如何?本期专题邀请多位专家撰
写稿件,向大家进行介绍。也希望有更多的人参加讨论。
\subsection{专题}
\href{http://history.ccf.org.cn/resources/1190201776262/2014/09/12/1.pdf}{智能体和多智能体系统研究}

\href{http://history.ccf.org.cn/resources/1190201776262/2014/09/12/2.pdf}{多智能体系统研究的历史、现状及挑战}

\href{http://history.ccf.org.cn/resources/1190201776262/2014/09/12/3.pdf}{面向智能体软件工程}

\href{http://history.ccf.org.cn/resources/1190201776262/2014/09/15/6.pdf}{基于多智能体的社会网络研究}

\href{http://history.ccf.org.cn/resources/1190201776262/2014/09/15/4.pdf}{基于智能体的在线演化机制及支撑环境}

\href{http://history.ccf.org.cn/resources/1190201776262/2014/09/15/5.pdf}{网络理论与多智能体系统研究}

\href{http://history.ccf.org.cn/resources/1190201776262/2014/09/15/7.pdf}{扑克游戏中的不完美信息博弈}

\subsection{专栏}
\href{http://history.ccf.org.cn/resources/1190201776262/2014/09/15/9.pdf}{科技创新能力是高技术企业的核心竞争力}

\href{http://history.ccf.org.cn/resources/1190201776262/2014/09/15/11.pdf}{全球课堂}

\href{http://history.ccf.org.cn/resources/1190201776262/2014/09/15/8.pdf}{联想华为启示录——论中国IT企业发展路在何方}

\href{http://history.ccf.org.cn/resources/1190201776262/2014/09/15/10.pdf}{把编程语言称作“计算机语言”之惑}

\subsection{视点}
\href{http://history.ccf.org.cn/resources/1190201776262/2014/09/15/12.pdf}{大数据时代的网络数据可视分析}

\subsection{动态}
\href{http://history.ccf.org.cn/resources/1190201776262/2014/09/15/13.pdf}{科研不是比赛——与邢波(Eric Xing)面对面聊科研}

\href{http://history.ccf.org.cn/resources/1190201776262/2014/09/15/14.pdf}{漫谈2014年人机交互大会}

\href{http://history.ccf.org.cn/resources/1190201776262/2014/09/15/15.pdf}{第37届国际信息检索年会}

\href{http://history.ccf.org.cn/resources/1190201776262/2014/09/15/16.pdf}{第28届AAAI人工智能会议}

\subsection{译文}
\href{http://history.ccf.org.cn/resources/1190201776262/2014/09/15/17.pdf}{命名数据网}


\section{\href{http://history.ccf.org.cn/sites/ccf/jsjtbbd.jsp?contentId=2815747243183}{\textbf{2014年第8期(总第102期)}}}
本期专题是计算机系统互连网络。在计算机系统中,负责数据搬运的互连网络是关键部件。根据位置的不同,互连网络又分为不同的层次。不同层次的互连网络之间究竟具有怎样错综复杂的联系与不同的显著特征?为了使读者对不同类型的互连网络有一个初步的认识,特邀编辑付斌章组织了本期专题,并邀请多位专家学者从系统角度出发,对不同层次的互连网络进行了介绍和分析。
\subsection{专题}
\href{http://history.ccf.org.cn/resources/1190201776262/2014/08/13/3.pdf}{光互连在未来E级系统中的应用}

\href{http://history.ccf.org.cn/resources/1190201776262/2014/08/13/4.pdf}{数据中心网络带宽保证技术}

\href{http://history.ccf.org.cn/resources/1190201776262/2014/08/13/0.pdf}{计算机系统互连网络}

\href{http://history.ccf.org.cn/resources/1190201776262/2014/08/13/2.pdf}{超大规模互连网络的通信模型及接口}

\href{http://history.ccf.org.cn/resources/1190201776262/2014/08/13/1.pdf}{跨层次优化片上网络设计}

\href{http://history.ccf.org.cn/resources/1190201776262/2014/08/13/5.pdf}{Torus网络的高效路由技术}

\subsection{专栏}
\href{http://history.ccf.org.cn/resources/1190201776262/2014/08/13/12.pdf}{做MOOC,教MOOC,学MOOC}

\href{http://history.ccf.org.cn/resources/1190201776262/2014/08/13/11.pdf}{中国网络安全立法之路在哪里?}

\href{http://history.ccf.org.cn/resources/1190201776262/2014/08/13/8.pdf}{Kaggle:从大数据走向大数据分析}

\href{http://history.ccf.org.cn/resources/1190201776262/2014/08/13/10.pdf}{互联网金融:是机遇还是泡沫}

\href{http://history.ccf.org.cn/resources/1190201776262/2014/08/13/7.pdf}{规则系统的移植性太差吗?}

\href{http://history.ccf.org.cn/resources/1190201776262/2014/08/13/9.pdf}{如何理解产业界}

\href{http://history.ccf.org.cn/resources/1190201776262/2014/08/13/6.pdf}{云存储价格战背后的科研缺失}

\subsection{视点}
\href{http://history.ccf.org.cn/resources/1190201776262/2014/08/13/13.pdf}{实时竞价:大数据驱动的计算广告新模态}

\subsection{动态}
\href{http://history.ccf.org.cn/resources/1190201776262/2014/08/13/15.pdf}{国际机器学习会议(ICML2014)在北京举行}

\href{http://history.ccf.org.cn/resources/1190201776262/2014/08/13/16.pdf}{第52届计算语言学国际会议}

\href{http://history.ccf.org.cn/resources/1190201776262/2014/08/13/14.pdf}{问高德纳的二十个问题}

\subsection{译文}
\href{http://history.ccf.org.cn/resources/1190201776262/2014/08/13/17.pdf}{结构性挑战和调整的必要性}


\section{\href{http://history.ccf.org.cn/sites/ccf/jsjtbbd.jsp?contentId=2810563245054}{\textbf{2014年第7期(总第101期)}}}
本期专题是从多媒体到跨媒体文本、图像、视频等不同类型媒体及其与之相关的社会属性信息混合在一起,形成了跨媒体形式。为了有效表达、分析和处理跨平台、跨模态网络媒体数据,需要研究这些数据的关联性语义结构一致性描述、属性获取、传播机制、时空推演等一系列关键问题。本期专题邀请多位专家对这些问题进行了阐述。
\subsection{专题}
\href{http://history.ccf.org.cn/resources/1190201776262/2014/07/14/1.pdf}{从多媒体到跨媒体}

\href{http://history.ccf.org.cn/resources/1190201776262/2014/07/14/2.pdf}{跨媒体时代的知识表达——感知、关联及一致性表示}

\href{http://history.ccf.org.cn/resources/1190201776262/2014/07/14/5.pdf}{数字墨水技术:数字世界中的神来之笔}

\href{http://history.ccf.org.cn/resources/1190201776262/2014/07/15/6.pdf}{社交多媒体计算的时效问题与隐私保护}

\href{http://history.ccf.org.cn/resources/1190201776262/2014/07/14/3.pdf}{跨媒体检索与排序}

\href{http://history.ccf.org.cn/resources/1190201776262/2014/07/14/4.pdf}{面向跨媒体数据的因果推理}

\subsection{专栏}
\href{http://history.ccf.org.cn/resources/1190201776262/2014/07/14/10.pdf}{RSA会议回顾与应用安全}

\href{http://history.ccf.org.cn/resources/1190201776262/2014/07/14/11.pdf}{信息安全大变局}

\href{http://history.ccf.org.cn/resources/1190201776262/2014/07/14/9.pdf}{建议创设CCF汇刊}

\href{http://history.ccf.org.cn/resources/1190201776262/2014/07/14/13.pdf}{大数据时代的机器学习}

\href{http://history.ccf.org.cn/resources/1190201776262/2014/07/14/8.pdf}{大数据下的数据管理领域研究体会}

\href{http://history.ccf.org.cn/resources/1190201776262/2014/07/14/7.pdf}{百度基础架构技术发展之路}

\href{http://history.ccf.org.cn/resources/1190201776262/2014/07/14/12.pdf}{软件工程:两个方向的挑战}

\subsection{动态}
\href{http://history.ccf.org.cn/resources/1190201776262/2014/07/14/15.pdf}{第23届国际万维网大会侧记}

\href{http://history.ccf.org.cn/resources/1190201776262/2014/07/14/14.pdf}{教育、科学、因果之美——访2011年度图灵奖得主、加州大学洛杉矶分校居迪亚·珀尔教授}

\subsection{译文}
\href{http://history.ccf.org.cn/resources/1190201776262/2014/07/14/16.pdf}{不好的研究中心是怎样炼成的}


\section{\href{http://history.ccf.org.cn/sites/ccf/jsjtbbd.jsp?contentId=2804860852126}{\textbf{2014年第6期(总第100期)}}}
本期特别报道:发出中国计算机科技人员自己的声音——祝贺CCCF出版发行100期。从2005年3月的创刊号到今年6月的第100期,九年来,CCCF 始终遵循为读者办刊的原则,受到广大会员和计算机同行的欢迎。现在CCCF 已拥有1500多位作者、两万多名读者。为纪念CCCF 出版100期,本期特别邀请国内外著名专家撰文,谈最新的计算机技术、谈科技创新。特别是李凯教授的文章,谈到教育和科研领域一些深层次的问题,令人深思。
\subsection{专题}
\href{http://history.ccf.org.cn/resources/1190201776262/2014/06/13/10.pdf}{移动社交网络中的位置隐私保护}

\href{http://history.ccf.org.cn/resources/1190201776262/2014/06/13/5.pdf}{社交网络中的隐私保护}

\href{http://history.ccf.org.cn/resources/1190201776262/2014/06/13/6.pdf}{冰与火:社交网络与个人隐私保护}

\href{http://history.ccf.org.cn/resources/1190201776262/2014/06/13/8.pdf}{数据分享中的差分隐私保护}

\href{http://history.ccf.org.cn/resources/1190201776262/2014/06/13/7.pdf}{微博博主的特征与行为大数据挖掘}

\href{http://history.ccf.org.cn/resources/1190201776262/2014/06/13/9.pdf}{社交网络匿名与隐私保护}

\subsection{专栏}
\href{http://history.ccf.org.cn/resources/1190201776262/2014/06/13/11.pdf}{对计算机体系结构研究的一点认识}

\subsection{动态}
\href{http://history.ccf.org.cn/resources/1190201776262/2014/06/13/12.pdf}{坚持到底,扬起人生的风帆——访华中科技大学蒋洪波教授}

\href{http://history.ccf.org.cn/resources/1190201776262/2014/06/13/13.pdf}{2014 人机交互大会}

\subsection{译文}
\href{http://history.ccf.org.cn/resources/1190201776262/2014/06/13/14.pdf}{谷歌流感趋势的启示:大数据分析中的陷阱}

\subsection{特别报道}
\href{http://history.ccf.org.cn/resources/1190201776262/2014/06/11/1.pdf}{促进中国高科技科研创新的想法}

\href{http://history.ccf.org.cn/resources/1190201776262/2014/06/11/0.pdf}{发出中国计算机科技人员自己的声音——祝贺CCCF 出版发行100期}

\href{http://history.ccf.org.cn/resources/1190201776262/2014/06/11/2.pdf}{学术经典·研发方向·教育论坛}

\href{http://history.ccf.org.cn/resources/1190201776262/2014/06/11/3.pdf}{C-AMAT:大数据时代的数据存取模型}

\href{http://history.ccf.org.cn/resources/1190201776262/2014/06/11/4.pdf}{无源感知网络}


\section{\href{http://history.ccf.org.cn/sites/ccf/jsjtbbd.jsp?contentId=2799676848706}{\textbf{2014年第5期(总第99期)}}}
本期专题是大数据时代的用户理解。普适计算的核心特征是“围绕人的计算”。从大数据中
更好地理解用户和为用户服务是普适计算一个新的重要研究方向。本期专题“大数据时代的用户理解”是典型的具有交叉性质的选题,讨论的实际内容超出了一般人理解的“普适计算”,既涉及用户数据的传感测量、行为理解、情感计算,也涉及对特定类型的用户数据的分析与挖掘。读者可以通过这组专题文章对大数据时代的用户理解相关前沿有更全面深入的了解和认识。
\subsection{专题}
\href{http://history.ccf.org.cn/resources/1190201776262/2014/05/12/3.pdf}{社交网络数据的情感计算}

\href{http://history.ccf.org.cn/resources/1190201776262/2014/05/12/4.pdf}{精彩纷呈的网络搜索日志挖掘}

\href{http://history.ccf.org.cn/resources/1190201776262/2014/05/12/5.pdf}{地理标注照片挖掘}

\href{http://history.ccf.org.cn/resources/1190201776262/2014/05/12/0.pdf}{大数据时代的用户理解}

\href{http://history.ccf.org.cn/resources/1190201776262/2014/05/12/1.pdf}{多源数据融合的用户行为感知与识别}

\href{http://history.ccf.org.cn/resources/1190201776262/2014/05/12/2.pdf}{基于大规模行为数据的用户理解}

\subsection{专栏}
\href{http://history.ccf.org.cn/resources/1190201776262/2014/05/12/11.pdf}{我们失去XP,中国将会怎样?}

\href{http://history.ccf.org.cn/resources/1190201776262/2014/05/20/6-1.pdf}{浅谈产业界与学术界的合作研究}

\href{http://history.ccf.org.cn/resources/1190201776262/2014/05/12/7.pdf}{大数据的存储渊源}

\href{http://history.ccf.org.cn/resources/1190201776262/2014/05/12/8.pdf}{体系结构研究者的人工智能之梦}

\href{http://history.ccf.org.cn/resources/1190201776262/2014/05/12/9.pdf}{对产业互联网战略的战术思考——争夺企业信息入口的战争}

\href{http://history.ccf.org.cn/resources/1190201776262/2014/05/12/10.pdf}{反病毒方法的现状、挑战与改进(下)}

\subsection{动态}
\href{http://history.ccf.org.cn/resources/1190201776262/2014/05/12/12.pdf}{仰望星空,脚踏实地——访2013中组部“青年千人”计划入选者黄隆波博士}

\href{http://history.ccf.org.cn/resources/1190201776262/2014/05/12/13.pdf}{记2013年图灵奖得主、微软科学家莱斯利·兰伯特}

\subsection{译文}
\href{http://history.ccf.org.cn/resources/1190201776262/2014/05/12/14.pdf}{重新审视MOOCs及政策建议}


\section{\href{http://history.ccf.org.cn/sites/ccf/jsjtbbd.jsp?contentId=2794320049848}{\textbf{2014年第4期(总第98期)}}}
本期专题是面向大数据的计算机体系结构。人类社会每天产生的数据量激增,迫使数据组织与处理技术、基础设施构建等方面要有所革新和突破。本期专题邀请企业界和学术界多位专家学者围绕大数据这一主题,从广义计算机体系结构角度,分别从不同侧面阐述该领域的研究现状及发展趋势。他们能给我们带来怎样的真知灼见?让我们一饱眼福!
\subsection{专题}
\href{http://history.ccf.org.cn/resources/1190201776262/2014/04/11/3.pdf}{基于新型非易失存储的存储结构}

\href{http://history.ccf.org.cn/resources/1190201776262/2014/04/11/5.pdf}{大数据处理的模式——系统结构、方法及发展趋势}

\href{http://history.ccf.org.cn/resources/1190201776262/2014/04/11/1.pdf}{异构并行的大数据处理器结构}

\href{http://history.ccf.org.cn/resources/1190201776262/2014/04/11/2.pdf}{大数据下处理器体系结构探讨}

\href{http://history.ccf.org.cn/resources/1190201776262/2014/04/11/4.pdf}{计算与存储融合体系结构}

\href{http://history.ccf.org.cn/resources/1190201776262/2014/04/11/0.pdf}{面向大数据的计算机体系结构}

\subsection{专栏}
\href{http://history.ccf.org.cn/resources/1190201776262/2014/04/11/11.pdf}{新产业周期下基础软件发展}

\href{http://history.ccf.org.cn/resources/1190201776262/2014/04/11/14.pdf}{ASPLOS会议}

\href{http://history.ccf.org.cn/resources/1190201776262/2014/04/11/13.pdf}{全新的开始}

\href{http://history.ccf.org.cn/resources/1190201776262/2014/04/11/12.pdf}{WPS前世、今生与未来}

\href{http://history.ccf.org.cn/resources/1190201776262/2014/04/11/8-1.pdf}{互联网在未来将给我们带来什么}

\href{http://history.ccf.org.cn/resources/1190201776262/2014/04/11/10.pdf}{文化是制约基础软件发展的重要因素}

\href{http://history.ccf.org.cn/resources/1190201776262/2014/04/11/9.pdf}{中科红旗为何倒下?——论国产基础软件的发展之路}

\href{http://history.ccf.org.cn/resources/1190201776262/2014/04/11/15.pdf}{ACM多媒体国际会议}

\href{http://history.ccf.org.cn/resources/1190201776262/2014/04/11/7.pdf}{从XP停服的博弈分析到中央网安小组方略解读}

\href{http://history.ccf.org.cn/resources/1190201776262/2014/04/11/6.pdf}{反病毒的现状、挑战与改进(上)*}

\subsection{视点}
\href{http://history.ccf.org.cn/resources/1190201776262/2014/04/11/16.pdf}{中文树库须加强谓词-论元结构描写}

\subsection{动态}
\href{http://history.ccf.org.cn/resources/1190201776262/2014/04/11/17.pdf}{2014 ACM网络搜索与数据挖掘国际会议}

\subsection{译文}
\href{http://history.ccf.org.cn/resources/1190201776262/2014/04/11/18.pdf}{大数据在政府方面的应用}


\section{\href{http://history.ccf.org.cn/sites/ccf/jsjtbbd.jsp?contentId=2788790450856}{\textbf{2014年第3期(总第97期)}}}
本期专题是大数据与软件工程。当前,计算技术学科各子领域均以大数据为处理对象。数据的获取与存储、组织与管理、分析与应用等环节都离不开软件的支持。在大数据时代,软件工程面临哪些机遇与挑战呢?本期专题邀请了国内外数名专家学者撰稿,分别从不同视角探讨大数据与软件工程的关系,相信会给读者带来参考和启发。
\subsection{专题}
\href{http://history.ccf.org.cn/resources/1190201776262/2014/03/11/1.pdf}{大数据时代的软件工程关键技术}

\href{http://history.ccf.org.cn/resources/1190201776262/2014/03/11/3.pdf}{面向移动应用大数据的软件工程研究}

\href{http://history.ccf.org.cn/resources/1190201776262/2014/03/11/4.pdf}{软件解析学——要点与实践}

\href{http://history.ccf.org.cn/resources/1190201776262/2014/03/11/6.pdf}{面向软件工程教育的大数据与在线系统}

\href{http://history.ccf.org.cn/resources/1190201776262/2014/03/10/0.pdf}{大数据与软件工程}

\href{http://history.ccf.org.cn/resources/1190201776262/2014/03/11/5.pdf}{基于大数据的软件工程新思维}

\href{http://history.ccf.org.cn/resources/1190201776262/2014/03/11/2.pdf}{一种基于大数据的软件设计模型}

\subsection{专栏}
\href{http://history.ccf.org.cn/resources/1190201776262/2014/03/11/7.pdf}{科研创新的基础:思维方式}

\href{http://history.ccf.org.cn/resources/1190201776262/2014/03/11/8.pdf}{占卜、推理和博弈}

\href{http://history.ccf.org.cn/resources/1190201776262/2014/03/11/9.pdf}{易忽略却重要的论文写作琐事}

\href{http://history.ccf.org.cn/resources/1190201776262/2014/03/11/10.pdf}{IEEE VIS会议}

\subsection{视点}
\href{http://history.ccf.org.cn/resources/1190201776262/2014/03/11/11.pdf}{大规模知识图谱技术}

\subsection{动态}
\href{http://history.ccf.org.cn/resources/1190201776262/2014/03/11/12.pdf}{创新,有时是不经意间开放的花朵——访2013 CCF青年科学家奖获得者朱军博士}

\href{http://history.ccf.org.cn/resources/1190201776262/2014/03/11/13.pdf}{ICCV 2013计算机视觉国际会议}

\subsection{译文}
\href{http://history.ccf.org.cn/resources/1190201776262/2014/03/11/14.pdf}{基于系统理论的一种防危性和安全性的综合方法}

\subsection{学会论坛}
\href{http://history.ccf.org.cn/resources/1190201776262/2014/03/11/15.pdf}{学会·会员·平台}


\section{\href{http://history.ccf.org.cn/sites/ccf/jsjtbbd.jsp?contentId=2785161647298}{\textbf{2014年第2期(总第96期)}}}
2014年1月18日是个值得关注的日子。CCF在这一天隆重地颁发了2013年度的各种奖项,还
召开了十届五次常务理事会。这次颁奖大会,不但颁发了终身成就奖、优秀博士学位论文奖、青年科学家奖等多个奖项,还首次颁发了计算机企业家奖。柳传志获得计算机企业家奖,并出席了颁奖大会。同时出席颁奖大会的还有王选奖获得者李彦宏先生等。
\subsection{专题}
\href{http://history.ccf.org.cn/resources/1190201776262/2014/02/17/5.pdf}{互联网数据的情感认知计算}

\href{http://history.ccf.org.cn/resources/1190201776262/2014/02/17/6.pdf}{视觉感知与缩放图像质量评估}

\href{http://history.ccf.org.cn/resources/1190201776262/2014/02/17/01.pdf}{认知与计算}

\href{http://history.ccf.org.cn/resources/1190201776262/2014/02/17/2.pdf}{感知、记忆和判断的认知计算模型}

\href{http://history.ccf.org.cn/resources/1190201776262/2014/02/17/3.pdf}{深度网络和认知计算}

\href{http://history.ccf.org.cn/resources/1190201776262/2014/02/17/4.pdf}{运动状态下人类对视觉信息的速度感知和决策的认知建模}

\subsection{专栏}
\href{http://history.ccf.org.cn/resources/1190201776262/2014/02/17/8.pdf}{物联网工程专业的建设与规范研究}

\href{http://history.ccf.org.cn/resources/1190201776262/2014/02/17/7.pdf}{从工程教育认证角度看中国的计算机教育}

\href{http://history.ccf.org.cn/resources/1190201776262/2014/02/17/9.pdf}{专业认证与专业建设}

\href{http://history.ccf.org.cn/resources/1190201776262/2014/02/17/10.pdf}{实践工程教育专业认证:体会与收获}

\href{http://history.ccf.org.cn/resources/1190201776262/2014/02/17/11.pdf}{“市场算法”}

\subsection{视点}
\href{http://history.ccf.org.cn/resources/1190201776262/2014/02/17/12.pdf}{高性能计算中基于算法的容错技术}

\subsection{动态}
\href{http://history.ccf.org.cn/resources/1190201776262/2014/02/17/14.pdf}{大数据会议专家观点分享——2013 CCF中国大数据技术大会和学术大会介绍}

\href{http://history.ccf.org.cn/resources/1190201776262/2014/02/17/13.pdf}{第26届神经信息处理系统国际会议}

\subsection{译文}
\href{http://history.ccf.org.cn/resources/1190201776262/2014/02/17/15.pdf}{事实上,图灵并没有发明计算机}

\subsection{学会论坛}
\href{http://history.ccf.org.cn/resources/1190201776262/2014/02/17/16.pdf}{专委改革任重道远}

\subsection{特别报道}
\href{http://history.ccf.org.cn/resources/1190201776262/2014/02/17/0.pdf}{2013 CCF颁奖大会}


\section{\href{http://history.ccf.org.cn/sites/ccf/jsjtbbd.jsp?contentId=2779113649284}{\textbf{2014年第1期(总第95期)}}}
本期专题是高性能科学与工程计算。“高性能科学与工程计算”专题,邀请了核物理数值模拟、创新数值算法研究、地球系统模式研究、计算生物学和材料物性计算研究等领域国内的著名专家撰文,几乎囊括了国内最活跃的高性能计算应用。我国高性能计算机的研制已走到世界前列,但应用还相当落后。这些专家介绍的应用,是否会给广大读者以启迪呢?在此,也特别感谢本期专题的特邀编辑莫则尧研究员和肖侬教授的工作。
\subsection{专题}
\href{http://history.ccf.org.cn/resources/1190201776262/2014/01/15/4.pdf}{高性能计算在计算生物学中的应用}

\href{http://history.ccf.org.cn/resources/1190201776262/2014/01/15/5.pdf}{材料大规模计算的需求与挑战}

\href{http://history.ccf.org.cn/resources/1190201776262/2014/01/15/0.pdf}{高性能计算应用与领域编程框架}

\href{http://history.ccf.org.cn/resources/1190201776262/2014/01/15/1.pdf}{面向高性能科学与工程计算的领域编程框架研究}

\href{http://history.ccf.org.cn/resources/1190201776262/2014/01/15/3.pdf}{地球系统模式与超级计算机}

\href{http://history.ccf.org.cn/resources/1190201776262/2014/01/15/2.pdf}{E级高性能科学计算的基础算法研究}

\subsection{专栏}
\href{http://history.ccf.org.cn/resources/1190201776262/2014/01/15/10.pdf}{ACM SIGGRAPH会议}

\href{http://history.ccf.org.cn/resources/1190201776262/2014/01/15/7.pdf}{拥抱互联网金融}

\href{http://history.ccf.org.cn/resources/1190201776262/2014/01/15/6.pdf}{2014年大数据发展趋势预测}

\href{http://history.ccf.org.cn/resources/1190201776262/2014/01/15/8.pdf}{青年学者成长难在哪里?}

\href{http://history.ccf.org.cn/resources/1190201776262/2014/01/15/11.pdf}{ACM SIGMOD会议}

\href{http://history.ccf.org.cn/resources/1190201776262/2014/01/15/9.pdf}{“协会”=“工会”?}

\subsection{视点}
\href{http://history.ccf.org.cn/resources/1190201776262/2014/01/15/12.pdf}{片网——图像表示的一种新技术}

\href{http://history.ccf.org.cn/resources/1190201776262/2014/01/15/13.pdf}{物联网感知大数据的存储与处理}

\href{http://history.ccf.org.cn/resources/1190201776262/2014/01/15/14.pdf}{我所理解的“软件定义的网络”}

\subsection{动态}
\href{http://history.ccf.org.cn/resources/1190201776262/2014/01/15/16.pdf}{穿戴式计算}

\href{http://history.ccf.org.cn/resources/1190201776262/2014/01/15/15.pdf}{关于统计学的科研、发展和教育}

\subsection{译文}
\href{http://history.ccf.org.cn/resources/1190201776262/2014/01/15/18.pdf}{经济学家:“小数据”成就了美国2012大选预测}

\href{http://history.ccf.org.cn/resources/1190201776262/2014/01/15/17.pdf}{人工智能会导致失业危机?}


\section{\href{http://history.ccf.org.cn/sites/ccf/jsjtbbd.jsp?contentId=2774275247149}{\textbf{2013年第12期(总第94期)}}}
本期专题是CCF CNCC 2013特邀报告。2013年10月24~26日,由CCF主办的2013中国计算机大会(CNCC)在长沙举行。大会特别邀请了国内外著名专家学者温腾·瑟夫、斯坦利·霍尔特、高光荣、张宏江、李晓明、杨学军以及主题论坛讲者李国杰、柴天佑、鄂维南、梅宏为大会作报告。报告非常精彩。现将部分大会报告整理成文呈现给大家。
\subsection{专题}
\href{http://history.ccf.org.cn/resources/1190201776262/2013/12/16/1.pdf}{互联网的起源与未来}

\href{http://history.ccf.org.cn/resources/1190201776262/2013/12/16/2.pdf}{为什么需要数据科学}

\href{http://history.ccf.org.cn/resources/1190201776262/2013/12/16/3.pdf}{大数据的流动之美——数据流与大数据:挑战与机遇}

\href{http://history.ccf.org.cn/resources/1190201776262/2013/12/16/9.pdf}{对大数据时代软件技术面临挑战的若干认识和思考}

\href{http://history.ccf.org.cn/resources/1190201776262/2013/12/16/4.pdf}{关于大数据的观察和思考}

\href{http://history.ccf.org.cn/resources/1190201776262/2013/12/16/5.pdf}{慕课:是橱窗?还是店堂?}

\href{http://history.ccf.org.cn/resources/1190201776262/2013/12/16/6.pdf}{E级计算的挑战与思考}

\href{http://history.ccf.org.cn/resources/1190201776262/2013/12/16/7.pdf}{大数据对计算机系统的挑战}

\href{http://history.ccf.org.cn/resources/1190201776262/2013/12/16/8.pdf}{工业“大数据”为实现现代工业系统的多目标运行优化开辟了新途径}

\subsection{专栏}
\href{http://history.ccf.org.cn/resources/1190201776262/2013/12/16/10.pdf}{英文论文写作随感}

\href{http://history.ccf.org.cn/resources/1190201776262/2013/12/16/11.pdf}{MobilityFirst:以移动支持为中心的未来互联网架构}

\href{http://history.ccf.org.cn/resources/1190201776262/2013/12/16/12.pdf}{岳麓山悠思}

\href{http://history.ccf.org.cn/resources/1190201776262/2013/12/16/13.pdf}{CCF YOCSEF报告会、论坛综述:经费使用,科研人员为什么成了“贼”?}

\subsection{动态}
\href{http://history.ccf.org.cn/resources/1190201776262/2013/12/16/14.pdf}{CCF NLPCC 2013之实践}

\href{http://history.ccf.org.cn/resources/1190201776262/2013/12/16/15.pdf}{微软亚洲研究院在“计算”中庆祝15岁生日}

\subsection{译文}
\href{http://history.ccf.org.cn/resources/1190201776262/2013/12/16/16.pdf}{钟摆摆得太远}


\section{\href{http://history.ccf.org.cn/sites/ccf/jsjtbbd.jsp?contentId=2768636233516}{\textbf{2013年第11期(总第93期)}}}
本期专题是三维呈现。近年来,三维显示受到了前所未有的关注,相关技术的发展突飞猛进。我们距离三维显示的普及还有多远?三维显示的应用前景如何?三维技术目前有哪些问题亟需突破?特邀编辑王涌天教授邀请数位专家,对目前三维显示技术面临的机遇、挑战和未来发展方向进行了全方位解读。
\subsection{专题}
\href{http://history.ccf.org.cn/resources/1190201776262/2013/11/15/0.pdf}{三维呈现}

\href{http://history.ccf.org.cn/resources/1190201776262/2013/11/15/1.pdf}{裸视三维显示技术}

\href{http://history.ccf.org.cn/resources/1190201776262/2013/11/15/2.pdf}{可探入光场三维显示}

\href{http://history.ccf.org.cn/resources/1190201776262/2013/11/15/3.pdf}{一种真三维显示技术:计算全息术}

\href{http://history.ccf.org.cn/resources/1190201776262/2013/11/15/4.pdf}{立体显示的观看体验评价}

\href{http://history.ccf.org.cn/resources/1190201776262/2013/11/15/5.pdf}{三维立体显示技术在医学诊疗中的应用}

\subsection{专栏}
\href{http://history.ccf.org.cn/resources/1190201776262/2013/11/15/7.pdf}{跟随你的心}

\href{http://history.ccf.org.cn/resources/1190201776262/2013/11/15/8.pdf}{一次“进程迁移”}

\href{http://history.ccf.org.cn/resources/1190201776262/2013/11/15/6.pdf}{浅谈本科生科研能力培养}

\subsection{视点}
\href{http://history.ccf.org.cn/resources/1190201776262/2013/11/15/9.pdf}{异构计算需要新的操作系统抽象}

\subsection{动态}
\href{http://history.ccf.org.cn/resources/1190201776262/2013/11/15/10.pdf}{ACM普适计算联合会议}

\href{http://history.ccf.org.cn/resources/1190201776262/2013/11/15/11.pdf}{大数据时代的自然语言处理——评《统计自然语言处理(第2版)》}

\subsection{译文}
\href{http://history.ccf.org.cn/resources/1190201776262/2013/11/15/12.pdf}{计算机科学研究的发展趋势}


\section{\href{http://history.ccf.org.cn/sites/ccf/jsjtbbd.jsp?contentId=2763391724116}{\textbf{2013年第10期(总第92期)}}}
本期专题是计算经济学。计算机技术和经济学的紧密结合,催生了一门交叉学科——计算经济学的发展。它在市场均衡、优化设计、拍卖机制、商业模式设计等方面起着重要的作用,有效地解决了从单一学科角度无法解决的问题。特邀编辑陆品燕教授邀请了数位专家撰文,从各自专业角度出发,结合该领域近年来的发展和未来趋势进行了探讨,读之,令人难忘。
\subsection{专题}
\href{http://history.ccf.org.cn/resources/1190201776262/2013/10/14/2.pdf}{计算社会选择理论简介}

\href{http://history.ccf.org.cn/resources/1190201776262/2013/10/14/3.pdf}{稳定匹配算法的研究和演变}

\href{http://history.ccf.org.cn/resources/1190201776262/2013/10/14/5.pdf}{社交网络·互联网·市场·人之算法博弈论}

\href{http://history.ccf.org.cn/resources/1190201776262/2013/10/18/1.pdf}{计算经济学}

\href{http://history.ccf.org.cn/resources/1190201776262/2013/10/14/4.pdf}{计算经济学与最优机制设计问题}

\subsection{专栏}
\href{http://history.ccf.org.cn/resources/1190201776262/2013/10/14/6.pdf}{神经网络计算机的涅槃}

\href{http://history.ccf.org.cn/resources/1190201776262/2013/10/14/7.pdf}{计算机系统会议论文是如何评审的}

\href{http://history.ccf.org.cn/resources/1190201776262/2013/10/14/9.pdf}{美国MOOC考察见闻}

\href{http://history.ccf.org.cn/resources/1190201776262/2013/10/14/8.pdf}{万维网的必然与偶然}

\href{http://history.ccf.org.cn/resources/1190201776262/2013/10/14/10.pdf}{从技术和研究角度看MOOC}

\subsection{动态}
\href{http://history.ccf.org.cn/resources/1190201776262/2013/10/14/13.pdf}{国际超大型数据库会议}

\href{http://history.ccf.org.cn/resources/1190201776262/2013/10/14/14.pdf}{第27届AAAI人工智能会议}

\href{http://history.ccf.org.cn/resources/1190201776262/2013/10/14/11.pdf}{国际数据挖掘与知识发现大会}

\href{http://history.ccf.org.cn/resources/1190201776262/2013/10/14/12.pdf}{ACM数据通信国际会议}

\subsection{译文}
\href{http://history.ccf.org.cn/resources/1190201776262/2013/10/18/15.pdf}{会议和期刊的融合}


\section{\href{http://history.ccf.org.cn/sites/ccf/jsjtbbd.jsp?contentId=2758898948684}{\textbf{2013年第9期(总第91期)}}}
本期专题是科研中的大数据处理。本期专题从国际科技数据委员会中国全国委员编写的“大数据时代的科研活动”研究报告中精选出5篇报告,分别由高能物理、空间科学、生物医药、生物多样性、对地观测等领域的专家撰文,为计算机领域的科技人员展现出一幅基础研究中大数据处理的壮观画面。这些文章既可为读者开拓眼界,同时也激励着他们进入这一片广阔的天地。
\subsection{专题}
\href{http://history.ccf.org.cn/resources/1190201776262/2013/09/18/5.pdf}{大数据时代的对地观测科学研究}

\href{http://history.ccf.org.cn/resources/1190201776262/2013/09/18/3.pdf}{大数据时代的生物医学}

\href{http://history.ccf.org.cn/resources/1190201776262/2013/09/18/4.pdf}{生物多样性信息学研究及应用}

\href{http://history.ccf.org.cn/resources/1190201776262/2013/09/18/2.pdf}{空间科学数据应用环境研究}

\href{http://history.ccf.org.cn/resources/1190201776262/2013/09/18/0.pdf}{科研中的大数据处理}

\href{http://history.ccf.org.cn/resources/1190201776262/2013/09/18/1.pdf}{高能物理数据及挑战}

\subsection{专栏}
\href{http://history.ccf.org.cn/resources/1190201776262/2013/09/18/12.pdf}{IEEE-CS是一个榜样——赴IEEE-CS访问、学习报告}

\href{http://history.ccf.org.cn/resources/1190201776262/2013/09/18/10.pdf}{跨越“朦胧期”的云计算}

\href{http://history.ccf.org.cn/resources/1190201776262/2013/09/18/6.pdf}{兵临城下——信息安全的新挑战}

\href{http://history.ccf.org.cn/resources/1190201776262/2013/09/18/7.pdf}{工业控制系统的安全性研究}

\href{http://history.ccf.org.cn/resources/1190201776262/2013/09/18/9.pdf}{系统能力培养初探}

\href{http://history.ccf.org.cn/resources/1190201776262/2013/09/18/11.pdf}{光一的“宇宙中心”}

\href{http://history.ccf.org.cn/resources/1190201776262/2013/09/18/8.pdf}{信息安全学科教育之路——从信息安全学科的“解剖学”课程开始}

\subsection{动态}
\href{http://history.ccf.org.cn/resources/1190201776262/2013/09/18/14.pdf}{国际机器学习大会}

\href{http://history.ccf.org.cn/resources/1190201776262/2013/09/18/13.pdf}{MobiSys 2013国际会议介绍——关注车辆通信系统设计降低能耗及定位追踪}

\subsection{译文}
\href{http://history.ccf.org.cn/resources/1190201776262/2013/09/18/15.pdf}{计算机科学各领域产出与影响力之辩}

\subsection{学会论坛}
\href{http://history.ccf.org.cn/resources/1190201776262/2013/09/18/16.pdf}{专委会特色活动:全国“电脑鼠走迷宫”竞赛}


\section{\href{http://history.ccf.org.cn/sites/ccf/jsjtbbd.jsp?contentId=2753023765862}{\textbf{2013年第8期(总第90期)}}}
本期专题是城市计算。空气污染、交通拥堵、能耗增加等问题正逐步困扰生活在城市中的人们。如何将计算机科学与城市中的大数据相结合,为城市的规划及发展提供技术及数据支持、为人们提供高品质的城市生活,是城市发展面临的一个挑战和机遇。本期专题特邀编辑郑宇教授邀请了数位专家,就城市计算的概念、基本框架、具体应用和面临的问题等内容进行了分析。
\subsection{专题}
\href{http://history.ccf.org.cn/resources/1190201776262/2013/08/16/5.pdf}{当出租车轨迹挖掘遇见智能交通}

\href{http://history.ccf.org.cn/resources/1190201776262/2013/08/15/1.pdf}{城市计算}

\href{http://history.ccf.org.cn/resources/1190201776262/2013/08/16/4.pdf}{大数据下的灾难行为分析和城市应急管理}

\href{http://history.ccf.org.cn/resources/1190201776262/2013/08/16/6.pdf}{城市交通的可视分析研究}

\href{http://history.ccf.org.cn/resources/1190201776262/2013/08/16/2.pdf}{城市计算与大数据}

\href{http://history.ccf.org.cn/resources/1190201776262/2013/08/16/3.pdf}{从数字脚印到城市计算}

\subsection{专栏}
\href{http://history.ccf.org.cn/resources/1190201776262/2013/08/16/8.pdf}{主流的傲慢与偏见:规则系统与机器学习}

\href{http://history.ccf.org.cn/resources/1190201776262/2013/08/16/9.pdf}{学会聪明地犯错误}

\href{http://history.ccf.org.cn/resources/1190201776262/2013/08/16/10.pdf}{一蹴而就还是锲而不舍}

\href{http://history.ccf.org.cn/resources/1190201776262/2013/08/16/7.pdf}{大数据融合实战}

\subsection{动态}
\href{http://history.ccf.org.cn/resources/1190201776262/2013/08/16/12.pdf}{自动和半自动知识提取}

\href{http://history.ccf.org.cn/resources/1190201776262/2013/08/16/13.pdf}{从图谱搜索看搜索技术的发展趋势}

\href{http://history.ccf.org.cn/resources/1190201776262/2013/08/16/11.pdf}{时空数据管理领域的国际会议}

\subsection{译文}
\href{http://history.ccf.org.cn/resources/1190201776262/2013/08/16/14.pdf}{经验数字}


\section{\href{http://history.ccf.org.cn/sites/ccf/jsjtbbd.jsp?contentId=2747666950372}{\textbf{2013年第7期(总第89期)}}}
本期专题是信息—物理融合系统。信息—物理融合系统(CPS)自提出以来,短短数年间获得国内外大量专家、学者的关注,被视为继计算机、互联网之后的又一重要里程碑,是国际信息技术竞争力新的制高点之一,具有重大战略意义。本期专题邀请了CPS领域几位年轻学者,介绍了CPS的基本概念、理论研究情况并进行了案例分析。该专题由何积丰院士、李宣东教授组织。
\subsection{专题}
\href{http://history.ccf.org.cn/resources/1190201776262/2013/07/19/1.pdf}{CPS研究热点概述}

\href{http://history.ccf.org.cn/resources/1190201776262/2013/07/19/2.pdf}{CPS行为建模及其仿真验证}

\href{http://history.ccf.org.cn/resources/1190201776262/2013/07/19/3.pdf}{形式化验证:从混成系统到CPS}

\href{http://history.ccf.org.cn/resources/1190201776262/2013/07/19/4.pdf}{CPS研究案例分析}

\href{http://history.ccf.org.cn/resources/1190201776262/2013/07/19/0.pdf}{信息—物理融合系统}

\subsection{专栏}
\href{http://history.ccf.org.cn/resources/1190201776262/2013/07/19/10.pdf}{米康革命与后PC时代}

\href{http://history.ccf.org.cn/resources/1190201776262/2013/07/19/11.pdf}{技术在“弃旧从新”中发展}

\href{http://history.ccf.org.cn/resources/1190201776262/2013/07/19/6.pdf}{科学研究中多思考本质问题}

\href{http://history.ccf.org.cn/resources/1190201776262/2013/07/19/8.pdf}{计算思维:计算机基础教学改革的第三个里程碑?}

\href{http://history.ccf.org.cn/resources/1190201776262/2013/07/19/9.pdf}{为建立自主可控的信息产业体系而努力奋斗}

\href{http://history.ccf.org.cn/resources/1190201776262/2013/07/15/5.pdf}{斯诺登效应的前因解读——Cyber空间相关的博弈思考}

\href{http://history.ccf.org.cn/resources/1190201776262/2013/07/19/7.pdf}{切换到数据视角,以数据为中心}

\subsection{动态}
\href{http://history.ccf.org.cn/resources/1190201776262/2013/07/19/14.pdf}{CCF YOCSEF特别论坛综述}

\href{http://history.ccf.org.cn/resources/1190201776262/2013/07/19/12.pdf}{深度学习——机器学习领域的新热点}

\href{http://history.ccf.org.cn/resources/1190201776262/2013/07/19/13.pdf}{智能体及多智能体系统国际会议}

\subsection{译文}
\href{http://history.ccf.org.cn/resources/1190201776262/2013/07/19/15.pdf}{网络犯罪、网络武器、网络战争:空气中是否弥漫了太多这样的事情?}

\subsection{学会论坛}
\href{http://history.ccf.org.cn/resources/1190201776262/2013/07/19/16.pdf}{从服务产业到面向产业——学术与产业的结合}

\href{http://history.ccf.org.cn/resources/1190201776262/2013/07/19/17.pdf}{探索学术交流的新方式——CCF软件工程专业委员会}


\section{\href{http://history.ccf.org.cn/sites/ccf/jsjtbbd.jsp?contentId=2742828525686}{\textbf{2013年第6期(总第88期)}}}
本期专题是软件系统需要自适应能力。运行环境的不确定性要求软件需要具备自适应能力,以应对其执行过程中运行场景和交互环境的变化。软件自适应的需求从何而来?如何描述软件的自适应需求并对其进行建模?软件在运行中需要引入何种机制才能体现自适应性?针对这些热点问题,本期专题特邀编辑金芝、吕建教授邀请数位知名专家撰文,与读者分享。
\subsection{专题}
\href{http://history.ccf.org.cn/resources/1190201776262/2013/06/17/4.pdf}{社会技术系统的自适应技术}

\href{http://history.ccf.org.cn/resources/1190201776262/2013/06/17/2.pdf}{软件可信性与自适应软件随想}

\href{http://history.ccf.org.cn/resources/1190201776262/2013/06/17/3.pdf}{软件服务的自适应与演化需求建模}

\href{http://history.ccf.org.cn/resources/1190201776262/2013/06/17/1.pdf}{软件系统需要自适应能力——为什么和如何拥有?}

\href{http://history.ccf.org.cn/resources/1190201776262/2013/06/17/5.pdf}{需求驱动的软件自适应方法}

\subsection{专栏}
\href{http://history.ccf.org.cn/resources/1190201776262/2013/06/17/9.pdf}{分层管理的魅力}

\href{http://history.ccf.org.cn/resources/1190201776262/2013/06/17/7.pdf}{MOOC:一个学生的体验与思考}

\href{http://history.ccf.org.cn/resources/1190201776262/2013/06/17/8.pdf}{自行创建类jQuery的JavaScript库}

\href{http://history.ccf.org.cn/resources/1190201776262/2013/06/17/6.pdf}{从大数据中挖掘什么?}

\subsection{动态}
\href{http://history.ccf.org.cn/resources/1190201776262/2013/06/18/11.pdf}{国际顶级舞台上的中国HPC形象——CCF高专委参加SC展览经验谈}

\href{http://history.ccf.org.cn/resources/1190201776262/2013/06/18/13.pdf}{命名数据网络——下一代互联网技术初探}

\href{http://history.ccf.org.cn/resources/1190201776262/2013/06/18/14.pdf}{软件定义网络:如果我们重新设计互联网}

\href{http://history.ccf.org.cn/resources/1190201776262/2013/06/18/10.pdf}{兴趣是科学研究的源动力——访2012 CCF青年科学家奖获得者王新兵}

\href{http://history.ccf.org.cn/resources/1190201776262/2013/06/18/15.pdf}{CCF YOCSEF“在线教育是改良还是革命”综述}

\href{http://history.ccf.org.cn/resources/1190201776262/2013/06/18/12.pdf}{第22届国际万维网大会}

\subsection{译文}
\href{http://history.ccf.org.cn/resources/1190201776262/2013/06/18/16.pdf}{计算机科学中的科学}

\subsection{学会论坛}
\href{http://history.ccf.org.cn/resources/1190201776262/2013/06/18/17.pdf}{学术与产业 研究与创新}

\href{http://history.ccf.org.cn/resources/1190201776262/2013/06/18/18.pdf}{做有用的学问——从学术到企业的一个支点}


\section{\href{http://history.ccf.org.cn/sites/ccf/jsjtbbd.jsp?contentId=2736780528206}{\textbf{2013年第5期(总第87期)}}}
本期专题是互联网时代的中文言语信息处理。言语是人类社会及现代语言学发展的产物。进入互联网时代,言语从不同角度对传统文化产生着影响,研究新时代下的言语信息处理有着重要意义。目前该领域的研究现状如何、面临哪些挑战?多言语机器翻译的发展趋势如何?言语信息服务对社会有何影响?针对这些关注度较高的问题,本期特邀编辑党建武、胡清华两位教授邀请了数位知名专家撰文,与读者分享。
\subsection{专题}
\href{http://history.ccf.org.cn/resources/1190201776262/2013/05/13/3.pdf}{从语音识别到言语识别}

\href{http://history.ccf.org.cn/resources/1190201776262/2013/05/13/4.pdf}{面向互联网的多言语机器翻译}

\href{http://history.ccf.org.cn/resources/1190201776262/2013/05/13/1.pdf}{互联网时代的中文言语信息处理}

\href{http://history.ccf.org.cn/resources/1190201776262/2013/05/13/2.pdf}{言语链:言语生成、感知及其交互}

\href{http://history.ccf.org.cn/resources/1190201776262/2013/05/13/6.pdf}{互联网时代言语信息服务}

\href{http://history.ccf.org.cn/resources/1190201776262/2013/05/13/5.pdf}{面向互联网的言语篇章深度理解}

\subsection{专栏}
\href{http://history.ccf.org.cn/resources/1190201776262/2013/05/13/7.pdf}{共享文化大数据的新机制}

\href{http://history.ccf.org.cn/resources/1190201776262/2013/05/13/9.pdf}{思想来自交流}

\href{http://history.ccf.org.cn/resources/1190201776262/2013/05/13/8.pdf}{网络个人隐私保护的难题——用户权利边界在哪里?}

\subsection{动态}
\href{http://history.ccf.org.cn/resources/1190201776262/2013/05/13/10.pdf}{CCF 2012海外杰出贡献奖获得者华云生教授}

\href{http://history.ccf.org.cn/resources/1190201776262/2013/05/13/11.pdf}{多样化驱动知识安全}

\subsection{译文}
\href{http://history.ccf.org.cn/resources/1190201776262/2013/05/13/12.pdf}{对斯坦福MOOC的思考}

\subsection{学会论坛}
\href{http://history.ccf.org.cn/resources/1190201776262/2013/05/13/20.pdf}{特色办会 注重品质——中国数据库学术会议经验介绍}


\section{\href{http://history.ccf.org.cn/sites/ccf/jsjtbbd.jsp?contentId=2732460526829}{\textbf{2013年第4期(总第86期)}}}
本期专题是互联网上的计算机语言。在互联网应用蓬勃发展的同时,互联网上的计算机语言也在发生着变化。互联网编程语言在高性能计算、兼容性、开放性和安全性等方面面临挑战。本期专题邀请数位在数据表示与查询、编程语言等领域有所建树的专家撰文,审视互联网上计算机语言的变革,探讨计算机语言面临的挑战,以期推动互联网时代计算机语言的进一步发展。
\subsection{专题}
\href{http://history.ccf.org.cn/resources/1190201776262/2013/04/18/1.pdf}{互联网上的计算机语言}

\href{http://history.ccf.org.cn/resources/1190201776262/2013/04/18/5.pdf}{互联网编程语言}

\href{http://history.ccf.org.cn/resources/1190201776262/2013/04/18/4.pdf}{语义网上的知识表示与查询语言}

\href{http://history.ccf.org.cn/resources/1190201776262/2013/04/18/2.pdf}{万维网数据表示与查询处理语言}

\href{http://history.ccf.org.cn/resources/1190201776262/2013/04/18/3.pdf}{万维网服务描述和业务流程定义语言}

\subsection{专栏}
\href{http://history.ccf.org.cn/resources/1190201776262/2013/04/18/10.pdf}{对未来的投资}

\href{http://history.ccf.org.cn/resources/1190201776262/2013/04/18/9.pdf}{绿色编译研究现状}

\href{http://history.ccf.org.cn/resources/1190201776262/2013/04/18/6.pdf}{从术语变化看高性能计算机的发展}

\href{http://history.ccf.org.cn/resources/1190201776262/2013/04/18/7.pdf}{内存计算:大数据处理的机遇与挑战}

\href{http://history.ccf.org.cn/resources/1190201776262/2013/04/18/8.pdf}{空间信息网络的若干关键技术}

\subsection{动态}
\href{http://history.ccf.org.cn/resources/1190201776262/2013/04/18/11.pdf}{“云物人海涌来,普适泛在何去”——记CCF PCC 2012主题论坛}

\href{http://history.ccf.org.cn/resources/1190201776262/2013/04/18/14.pdf}{CCF推荐国际学术会议和期刊目录(第三版)发布}

\href{http://history.ccf.org.cn/resources/1190201776262/2013/04/18/12.pdf}{下一代搜索引擎的焦点:知识图谱}

\href{http://history.ccf.org.cn/resources/1190201776262/2013/04/18/13.pdf}{国际业务过程管理大会}

\subsection{译文}
\href{http://history.ccf.org.cn/resources/1190201776262/2013/04/18/15.pdf}{打开共享社会媒体数据之门}

\subsection{学会论坛}
\href{http://history.ccf.org.cn/resources/1190201776262/2013/04/18/16.pdf}{团结一致 精心运作 扩大影响——CCF中文信息技术专业委员会}


\section{\href{http://history.ccf.org.cn/sites/ccf/jsjtbbd.jsp?contentId=2726585359042}{\textbf{2013年第3期(总第85期)}}}
本期专题是推荐技术与互联网广告。互联网的发展将我们带入了一个大数据信息化的时代。如何从海量信息中挖掘出用户感兴趣的信息,并把这些信息推送给用户?本期专题针对此问题,邀请了学术界和工业界的专家撰文,向读者展示了推荐系统技术和计算广告技术在基础理论和算法方面受到的挑战,以及具有重大应用价值的研究方向。
\subsection{专题}
\href{http://history.ccf.org.cn/resources/1190201776262/2013/03/15/1.pdf}{推荐技术与互联网广告}

\href{http://history.ccf.org.cn/resources/1190201776262/2013/03/15/2.pdf}{机器学习和数据挖掘在个性化推荐系统中的应用}

\href{http://history.ccf.org.cn/resources/1190201776262/2013/03/15/3.pdf}{场景引擎:下一代推荐系统的核心模块}

\href{http://history.ccf.org.cn/resources/1190201776262/2013/03/15/4.pdf}{移动情境感知的个性化推荐技术}

\href{http://history.ccf.org.cn/resources/1190201776262/2013/03/15/5.pdf}{方兴未艾的计算广告学}

\href{http://history.ccf.org.cn/resources/1190201776262/2013/03/15/6.pdf}{移动互联网广告的机遇和挑战}

\subsection{专栏}
\href{http://history.ccf.org.cn/resources/1190201776262/2013/03/15/7.pdf}{浅谈科研流程及其中的师生合作}

\href{http://history.ccf.org.cn/resources/1190201776262/2013/03/15/8.pdf}{唤醒环境的善念}

\href{http://history.ccf.org.cn/resources/1190201776262/2013/03/15/9.pdf}{《现代软件工程》教学心得}

\href{http://history.ccf.org.cn/resources/1190201776262/2013/03/15/10.pdf}{中国比你想象的要近}

\subsection{动态}
\href{http://history.ccf.org.cn/resources/1190201776262/2013/03/15/11.pdf}{科学与工程,原本一家人——访CCF青年科学家奖获得者陈小武教授}

\href{http://history.ccf.org.cn/resources/1190201776262/2013/03/15/12.pdf}{2013 ACM网络搜索与数据挖掘国际会议}

\href{http://history.ccf.org.cn/resources/1190201776262/2013/03/15/13.pdf}{中外专家畅谈在线教育现状及未来前景}

\href{http://history.ccf.org.cn/resources/1190201776262/2013/03/15/15.pdf}{CCF推荐国际学术会议和期刊目录(第三版)发布}

\subsection{译文}
\href{http://history.ccf.org.cn/resources/1190201776262/2013/03/15/14.pdf}{我最爱的十篇“实用理论”论文}


\section{\href{http://history.ccf.org.cn/sites/ccf/jsjtbbd.jsp?contentId=2722092546435}{\textbf{2013年第2期(总第84期)}}}
2012 CCF颁奖大会于2013年1月26日在京隆重举行。会上颁发了CCF终身成就奖、CCF青年科学奖、CCF优秀博士学位论文奖以及CCF杰出教育奖、杰出贡献奖、卓越服务奖、优秀工作组奖,还表彰了2012年度的优秀专委和CCF会员发展优秀单位和个人。还为2012 CCF会士颁发了会士证书。本期的专题是并行程序的调试、测试与模型检验。随着多核系统的普及和互联网
产生的大量数据对分布式系统的需求,软件系统的主流正从过去的单机串行软件转变为并行与分布式软件。由于并行与分布式软件的执行具有不确定性,因此对并行与分布式程序的调试成为更具挑战性的问题。本期专题栏目邀请了多位专家,从不同角度就并行程序的调试、测试与模型检验展开了讨论。
\subsection{专题}
\href{http://history.ccf.org.cn/resources/1190201776262/2013/02/17/1.pdf}{并行程序的调试、测试与模型检验}

\href{http://history.ccf.org.cn/resources/1190201776262/2013/02/17/2.pdf}{如何应对多线程程序并发错误}

\href{http://history.ccf.org.cn/resources/1190201776262/2013/02/17/3.pdf}{确定性重放技术}

\href{http://history.ccf.org.cn/resources/1190201776262/2013/02/17/4.pdf}{确定性执行技术}

\href{http://history.ccf.org.cn/resources/1190201776262/2013/02/17/5.pdf}{并行软件测试技术浅析}

\href{http://history.ccf.org.cn/resources/1190201776262/2013/02/17/6.pdf}{大规模并行程序的软件调试方法浅析}

\href{http://history.ccf.org.cn/resources/1190201776262/2013/02/17/7.pdf}{模型检测在分布式系统中的应用}

\subsection{专栏}
\href{http://history.ccf.org.cn/resources/1190201776262/2013/02/17/8.pdf}{关于我国百亿亿级计算发展的思考}

\href{http://history.ccf.org.cn/resources/1190201776262/2013/02/17/9.pdf}{普适计算2018:发展趋势}

\href{http://history.ccf.org.cn/resources/1190201776262/2013/02/17/10.pdf}{为什么人前进的路总是被自己挡住}

\href{http://history.ccf.org.cn/resources/1190201776262/2013/02/17/11.pdf}{浅谈如何提高学生的“系统思维”能力}

\subsection{动态}
\href{http://history.ccf.org.cn/resources/1190201776262/2013/02/17/12.pdf}{神经信息处理系统国际会议}

\subsection{译文}
\href{http://history.ccf.org.cn/resources/1190201776262/2013/02/17/13.pdf}{质感科学的发展前景}

\subsection{特别报道}
\href{http://history.ccf.org.cn/resources/1190201776262/2013/02/22/0.pdf}{2012中国计算机学会颁奖大会在北京举行}


\section{\href{http://history.ccf.org.cn/sites/ccf/jsjtbbd.jsp?contentId=2716908557315}{\textbf{2013年第1期(总第83期)}}}
本期专题是CCF CNCC 2012特邀报告。2012年10月18~20日,由中国计算机学会主办的2012
中国计算机大会(CCF CNCC2012)在大连举行。大会特别邀请了多位国内外知名学者和专家就当今计算技术研究热点和发展趋势作了报告。本期专题刊登了部分特邀讲者的报告,以飨读者。
\subsection{专题}
\href{http://history.ccf.org.cn/resources/1190201776262/2013/01/18/6.pdf}{新一代自然用户界面——Kinect引领人机交互未来}

\href{http://history.ccf.org.cn/resources/1190201776262/2013/01/18/1.pdf}{图灵未竟的事业}

\href{http://history.ccf.org.cn/resources/1190201776262/2013/01/18/2.pdf}{云中高性能计算}

\href{http://history.ccf.org.cn/resources/1190201776262/2013/01/18/7.pdf}{华为在信息技术领域的战略思考}

\href{http://history.ccf.org.cn/resources/1190201776262/2013/01/18/3.pdf}{视频会议中基于延迟感应的丢包隐藏}

\href{http://history.ccf.org.cn/resources/1190201776262/2013/01/18/5.pdf}{移动互联网时代的语音技术}

\href{http://history.ccf.org.cn/resources/1190201776262/2013/01/18/4.pdf}{网络视频服务的关键问题}

\subsection{专栏}
\href{http://history.ccf.org.cn/resources/1190201776262/2013/01/21/10.pdf}{计算机科学,大一学生怎样来爱你?}

\href{http://history.ccf.org.cn/resources/1190201776262/2013/01/18/9.pdf}{什么是好的教育?}

\href{http://history.ccf.org.cn/resources/1190201776262/2013/01/18/8.pdf}{斯坦福与北大计算机课程的改革实践}

\href{http://history.ccf.org.cn/resources/1190201776262/2013/01/21/11.pdf}{安全博弈论}

\subsection{动态}
\href{http://history.ccf.org.cn/resources/1190201776262/2013/01/21/12.pdf}{第二十一届国际信息与知识管理会议}

\subsection{译文}
\href{http://history.ccf.org.cn/resources/1190201776262/2013/01/21/13.pdf}{为金字塔底层进行IT创新}


\section{\href{http://history.ccf.org.cn/sites/ccf/jsjtbbd.jsp?contentId=2711378931600}{\textbf{2012年第12期(总第82期)}}}
本期专题是多媒体信息检索。伴随着便携式数码设备的流行以及媒体压缩、存储和通信技术的进步,多媒体数据呈现爆炸式增长并全方位渗透到人们的生活中。如何有效地对多媒体信息进行检索,一直是多媒体以及信息检索研究领域的热点问题。本期专题围绕多媒体信息检索中的机遇和挑战,邀请了数位在该领域有所建树的专家撰文,针对多媒体信息检索中最受关注的问题以及相应的解决方法进行了探讨,内容涵盖了移动视觉搜索、海量多媒体数据索引、草图搜索、三维对象检索和多媒体等技术。
\subsection{专题}
\href{http://history.ccf.org.cn/resources/1190201776262/2012/12/17/6.pdf}{多媒体问答——多模态智能检索初探}

\href{http://history.ccf.org.cn/resources/1190201776262/2012/12/17/4.pdf}{草图搜索的魅力与挑战}

\href{http://history.ccf.org.cn/resources/1190201776262/2012/12/17/2.pdf}{移动视觉搜索技术瓶颈与挑战}

\href{http://history.ccf.org.cn/resources/1190201776262/2012/12/17/5.pdf}{基于视图的三维对象检索}

\href{http://history.ccf.org.cn/resources/1190201776262/2012/12/17/3.pdf}{海量多媒体数据哈希索引技术}

\href{http://history.ccf.org.cn/resources/1190201776262/2012/12/17/1.pdf}{多媒体信息检索}

\subsection{专栏}
\href{http://history.ccf.org.cn/resources/1190201776262/2012/12/17/9.pdf}{大数据时代:计算机专业教育探讨}

\href{http://history.ccf.org.cn/resources/1190201776262/2012/12/17/12.pdf}{大数据算法}

\href{http://history.ccf.org.cn/resources/1190201776262/2012/12/17/11.pdf}{互联网计算背景下的集成科学}

\href{http://history.ccf.org.cn/resources/1190201776262/2012/12/17/8.pdf}{从零星工程到微博技术:迈向计算社会}

\href{http://history.ccf.org.cn/resources/1190201776262/2012/12/17/10.pdf}{网络思维:互联网时代新思维}

\href{http://history.ccf.org.cn/resources/1190201776262/2012/12/17/7.pdf}{大数据热点问题与2013年发展趋势分析}

\subsection{动态}
\href{http://history.ccf.org.cn/resources/1190201776262/2012/12/17/13.pdf}{“内容已死”与“内容万岁”——记2012 ACM多媒体会议主题论坛}

\href{http://history.ccf.org.cn/resources/1190201776262/2012/12/17/14.pdf}{ACM SIGGRAPH Asia 2012在新加坡举行}

\subsection{译文}
\href{http://history.ccf.org.cn/resources/1190201776262/2012/12/17/15.pdf}{21世纪计算机体系结构}


\section{\href{http://history.ccf.org.cn/sites/ccf/jsjtbbd.jsp?contentId=2706022159974}{\textbf{2012年第11期(总第81期)}}}
本期专题是图数据的管理与挖掘。现有应用处理的很多数据都具有图数据的特性,这种趋势随着应用及数据的日趋复杂变得愈加明显。作为非关系型数据库重要分支的图数据库应运而生。对图数据的研究不仅要注重算法层面,还要特别关注对系统的研究。本期专题邀请了工业界和学术界的研究人员就图数据的管理和挖掘这一主题发表看法,围绕系统和算法两个方面展开讨论。
\subsection{专题}
\href{http://history.ccf.org.cn/resources/1190201776262/2012/11/16/6.pdf}{海量RDF数据管理}

\href{http://history.ccf.org.cn/resources/1190201776262/2012/11/16/3.pdf}{基于哈希存储器的大图生成器}

\href{http://history.ccf.org.cn/resources/1190201776262/2012/11/16/4.pdf}{图匹配问题的应用和研究}

\href{http://history.ccf.org.cn/resources/1190201776262/2012/11/16/2.pdf}{大图的分布式存储}

\href{http://history.ccf.org.cn/resources/1190201776262/2012/11/16/5.pdf}{图查询:社会计算时代的新型搜索}

\href{http://history.ccf.org.cn/resources/1190201776262/2012/11/16/1.pdf}{图数据的管理与挖掘}

\subsection{专栏}
\href{http://history.ccf.org.cn/resources/1190201776262/2012/11/16/8.pdf}{给互联网装上“梯子”——留守儿童万家团圆项目}

\href{http://history.ccf.org.cn/resources/1190201776262/2012/11/16/9.pdf}{信息鸿沟 科技跨越}

\href{http://history.ccf.org.cn/resources/1190201776262/2012/11/16/7.pdf}{世纪图灵纪念}

\href{http://history.ccf.org.cn/resources/1190201776262/2012/11/16/12.pdf}{顶天立地,做大文章}

\href{http://history.ccf.org.cn/resources/1190201776262/2012/11/16/10.pdf}{前十载粗刀钝剑,再十年干将莫邪}

\href{http://history.ccf.org.cn/resources/1190201776262/2012/11/16/11.pdf}{踏踏实实做好系统研究}

\subsection{动态}
\href{http://history.ccf.org.cn/resources/1190201776262/2012/11/16/15.pdf}{获得ACM SIGIR 2012最佳学生论文奖}

\href{http://history.ccf.org.cn/resources/1190201776262/2012/11/16/13.pdf}{第29届CCF数据库学术会议}

\href{http://history.ccf.org.cn/resources/1190201776262/2012/12/26/201211-兰艳艳.pdf}{ACM信息检索国际会议}

\subsection{译文}
\href{http://history.ccf.org.cn/resources/1190201776262/2012/11/16/16.pdf}{机器学习那些事}


\section{\href{http://history.ccf.org.cn/sites/ccf/jsjtbbd.jsp?contentId=2700838126422}{\textbf{2012年第10期(总第80期)}}}
本期专题内容为存储关键技术。数据作为信息的载体历来受到人们的重视。如今爆炸式增长的数据已经给大规模数据中心带来了前所未有的挑战,传统的存储模式已不能满足EB级数据存储的可靠性、安全性、能耗等方面的需求。本期专题栏目邀请了国内外从事存储领域的专家学者,重点阐述了大规模数据中心的存储关键技术。希望能让读者对目前大规模数据中心的存储背景和未来发展有所了解。
\subsection{专题}
\href{http://history.ccf.org.cn/resources/1190201776262/2012/10/17/2.pdf}{大规模数据中心的数据存储可靠性}

\href{http://history.ccf.org.cn/resources/1190201776262/2012/10/17/5.pdf}{面向EB级存储的机群文件系统}

\href{http://history.ccf.org.cn/resources/1190201776262/2012/10/17/4.pdf}{集群重复数据删除与大数据保护}

\href{http://history.ccf.org.cn/resources/1190201776262/2012/10/17/6.pdf}{大规模数据中心的存储安全访问控制}

\href{http://history.ccf.org.cn/resources/1190201776262/2012/10/17/1.pdf}{大规模数据中心的存储关键技术}

\href{http://history.ccf.org.cn/resources/1190201776262/2012/10/17/3.pdf}{固态存储技术的发展与展望}

\subsection{专栏}
\href{http://history.ccf.org.cn/resources/1190201776262/2012/10/17/7.pdf}{群智感知计算}

\href{http://history.ccf.org.cn/resources/1190201776262/2012/10/17/8.pdf}{技术公益——技术人的社会责任}

\href{http://history.ccf.org.cn/resources/1190201776262/2012/10/17/11.pdf}{“政产学研用”人才培养模式}

\href{http://history.ccf.org.cn/resources/1190201776262/2012/10/17/10.pdf}{最后一课——我的计算机教育实验}

\href{http://history.ccf.org.cn/resources/1190201776262/2012/10/17/9.pdf}{大学计算机基础教学改革的困惑与跃升}

\subsection{动态}
\href{http://history.ccf.org.cn/resources/1190201776262/2012/10/17/12.pdf}{KDD’12概况}

\subsection{译文}
\href{http://history.ccf.org.cn/resources/1190201776262/2012/10/17/13.pdf}{计算机科学中文献引用的探讨}


\section{\href{http://history.ccf.org.cn/sites/ccf/jsjtbbd.jsp?contentId=2694962949020}{\textbf{2012年第9期(总第79期)}}}
本期专题主题是学术界谈“大数据”。当人们享受社交网络带来的方便与快捷时,大数据时代的进化就发生在我们身边。大数据不仅仅是数据数量的差别,也不仅仅是相关信息技术的开发和应对,而可能是一次人类对客观世界认知飞跃的前奏。本期专题邀请了李国杰院士、周晓方教授等国内外知名学者撰文,介绍了当前大数据面临的挑战,希望能引起读者的关注和讨论。
\subsection{专题}
\href{http://history.ccf.org.cn/resources/1190201776262/2012/09/13/2.pdf}{大数据研究的科学价值}

\href{http://history.ccf.org.cn/resources/1190201776262/2012/09/13/5.pdf}{大规模科学可视化}

\href{http://history.ccf.org.cn/resources/1190201776262/2012/09/13/6.pdf}{超大规模数据可视分析十大挑战}

\href{http://history.ccf.org.cn/resources/1190201776262/2012/09/13/4.pdf}{大数据科学与工程的挑战与思考}

\href{http://history.ccf.org.cn/resources/1190201776262/2012/09/13/1.pdf}{学术界谈“大数据”}

\href{http://history.ccf.org.cn/resources/1190201776262/2012/09/13/3.pdf}{从数据管理视角看大数据挑战}

\subsection{专栏}
\href{http://history.ccf.org.cn/resources/1190201776262/2012/09/13/8.pdf}{科研生活:培养超视距的心灵}

\href{http://history.ccf.org.cn/resources/1190201776262/2012/09/13/10.pdf}{超级计算中心的核心应用}

\href{http://history.ccf.org.cn/resources/1190201776262/2012/09/13/7.pdf}{谁推动了信息产业发展?——读美国科学院报告《信息产业创新》有感}

\href{http://history.ccf.org.cn/resources/1190201776262/2012/09/13/12.pdf}{用“咖啡”的精神做学问}

\href{http://history.ccf.org.cn/resources/1190201776262/2012/09/13/13.pdf}{理实交融 踏实做事}

\href{http://history.ccf.org.cn/resources/1190201776262/2012/09/13/11.pdf}{我的普适计算科研经历与感受}

\href{http://history.ccf.org.cn/resources/1190201776262/2012/09/13/9.pdf}{我的手机我做主——从安卓程序用户隐私泄露问题说起}

\subsection{动态}
\href{http://history.ccf.org.cn/resources/1190201776262/2012/09/13/14.pdf}{AAAI人工智能会议}

\subsection{译文}
\href{http://history.ccf.org.cn/resources/1190201776262/2012/09/13/15.pdf}{实物用户界面:你可以触摸的技术}

\href{http://history.ccf.org.cn/resources/1190201776262/2012/09/13/16.pdf}{谷歌的混合研究方法}


\section{\href{http://history.ccf.org.cn/sites/ccf/jsjtbbd.jsp?contentId=2689951732598}{\textbf{2012年第8期(总第78期)}}}
本期专题内容是计算摄影学。计算摄影学是一门将计算机视觉、数字信号处理、图形学等深度交叉的新兴学科,其研究范围宽泛。自20世纪末,计算摄影学发展迅速,逐渐成为一个热门的研究领域。本期专题文章,以全光函数的不同组合为示例,从不同的侧面介绍了计算摄影学的研究思路、实现手段及相应的成果
\subsection{专题}
\href{http://history.ccf.org.cn/resources/1190201776262/2012/08/15/3.pdf}{光场理论、采集及应用}

\href{http://history.ccf.org.cn/resources/1190201776262/2012/08/15/4.pdf}{计算摄影学——时间维度的拓展}

\href{http://history.ccf.org.cn/resources/1190201776262/2012/08/15/5.pdf}{计算光谱成像方法}

\href{http://history.ccf.org.cn/resources/1190201776262/2012/08/15/1.pdf}{计算摄影学}

\href{http://history.ccf.org.cn/resources/1190201776262/2012/08/15/6.pdf}{基于图像的深度计算}

\href{http://history.ccf.org.cn/resources/1190201776262/2012/08/15/2.pdf}{透过全光函数看计算摄影发展}

\subsection{专栏}
\href{http://history.ccf.org.cn/resources/1190201776262/2012/08/15/10.pdf}{指令系统与国产处理器}

\href{http://history.ccf.org.cn/resources/1190201776262/2012/08/15/11.pdf}{移动互联网环境下基础软件之路}

\href{http://history.ccf.org.cn/resources/1190201776262/2012/08/15/7.pdf}{科学研究的道路与目标——在芝加哥IEEE院士庆祝晚宴上的发言}

\href{http://history.ccf.org.cn/resources/1190201776262/2012/08/15/8.pdf}{获得ACM MobiCom 2011最佳论文奖}

\href{http://history.ccf.org.cn/resources/1190201776262/2012/08/15/9.pdf}{跨界服务:现代服务业的创新服务模式}

\href{http://history.ccf.org.cn/resources/1190201776262/2012/08/15/12.pdf}{命名数据网络与互联网内容分发}

\subsection{动态}
\href{http://history.ccf.org.cn/resources/1190201776262/2012/08/15/14.pdf}{与我国计算机事业缘伴终生——专访夏培肃先生}

\href{http://history.ccf.org.cn/resources/1190201776262/2012/08/17/15.pdf}{诺基亚移动数据挖掘竞赛}

\href{http://history.ccf.org.cn/resources/1190201776262/2012/08/15/16.pdf}{让虚拟世界真实可触——2012 IEEE触/力觉研讨会介绍}

\subsection{译文}
\href{http://history.ccf.org.cn/resources/1190201776262/2012/08/15/17.pdf}{认知计算}


\section{\href{http://history.ccf.org.cn/sites/ccf/jsjtbbd.jsp?contentId=2684767749074}{\textbf{2012年第7期(总第77期)}}}
本期专题为“云计算环境安全”。在云计算广泛应用的同时,云服务提供商扮演着数据掌控者的角色,用户的安全和隐私可能被侵犯……面对安全挑战,研究者们通过使用虚拟机隔离、加密、监控与恢复、可信计算等技术对安全进行增强。
\subsection{专题}
\href{http://history.ccf.org.cn/resources/1190201776262/2012/07/16/1.pdf}{云计算环境的安全与隐私性保护}

\href{http://history.ccf.org.cn/resources/1190201776262/2012/07/16/3.pdf}{公有云中的安全研究}

\href{http://history.ccf.org.cn/resources/1190201776262/2012/07/16/5.pdf}{可信云计算环境的构建}

\href{http://history.ccf.org.cn/resources/1190201776262/2012/07/16/6.pdf}{云计算在信息安全中的应用}

\href{http://history.ccf.org.cn/resources/1190201776262/2012/07/16/7.pdf}{云环境下安全模式的进化与演绎}

\href{http://history.ccf.org.cn/resources/1190201776262/2012/07/16/4.pdf}{加密数据云存储及其隐私保护}

\href{http://history.ccf.org.cn/resources/1190201776262/2012/07/16/2.pdf}{云计算下的系统安全监控与恢复}

\subsection{专栏}
\href{http://history.ccf.org.cn/resources/1190201776262/2012/07/16/8.pdf}{个性化推荐的十大挑战}

\href{http://history.ccf.org.cn/resources/1190201776262/2012/07/16/11.pdf}{戈登奖简介}

\href{http://history.ccf.org.cn/resources/1190201776262/2012/07/16/9.pdf}{面向容灾云的计算容错机制研究}

\href{http://history.ccf.org.cn/resources/1190201776262/2012/07/16/10.pdf}{从IBM红杉系统夺冠看第三十九届TOP500排行榜}

\href{http://history.ccf.org.cn/resources/1190201776262/2012/07/16/12.pdf}{计算机网络病毒的传播模式分析}

\href{http://history.ccf.org.cn/resources/1190201776262/2012/07/16/13.pdf}{“震网”、“火焰”恶意代码警示——信息物理系统安全问题与挑战}

\subsection{动态}
\href{http://history.ccf.org.cn/resources/1190201776262/2012/07/16/14.pdf}{2012超大数据库国际会议亚洲分会}

\subsection{学会论坛}
\href{http://history.ccf.org.cn/resources/1190201776262/2012/07/16/16.pdf}{ACM举行阿兰·图灵诞辰100周年纪念会}


\section{\href{http://history.ccf.org.cn/sites/ccf/jsjtbbd.jsp?contentId=2679929345796}{\textbf{2012年第6期(总第76期)}}}
本期专题是“数据密集业务的挑战和机遇——‘大数据’在工业界”。随着各行业和部门产生的数据量的急剧增长,对数据处理分析能力的要求不断提高。如何面对大数据的挑战,发现新的机遇,是工业界和学术界共同关注的问题。本期专题重点介绍了工业部门在密集数据方面的应用、实践和需求;邀请了中兴通讯、淘宝、百分点公司、华大基因、SAP、盛大在线等企业撰文,分享来自业务第一线的看法。本期专题作者主要来自企业的科研人员,是一次有意义的尝试。
\subsection{专题}
\href{http://history.ccf.org.cn/resources/1190201776262/2012/06/18/2.pdf}{海量数据技术在电信业务内应用}

\href{http://history.ccf.org.cn/resources/1190201776262/2012/06/18/3.pdf}{大数据的魔力}

\href{http://history.ccf.org.cn/resources/1190201776262/2012/06/18/7.pdf}{基于云计算的数据密集业务应用}

\href{http://history.ccf.org.cn/resources/1190201776262/2012/06/18/1.pdf}{数据密集业务的挑战和机遇——“大数据”在工业界}

\href{http://history.ccf.org.cn/resources/1190201776262/2012/06/18/4.pdf}{推荐引擎:信息暗海的领航员}

\href{http://history.ccf.org.cn/resources/1190201776262/2012/06/18/6.pdf}{应对生命科学的大数据挑战}

\href{http://history.ccf.org.cn/resources/1190201776262/2012/06/18/5.pdf}{有容乃大——大规模数据云端存储}

\subsection{专栏}
\href{http://history.ccf.org.cn/resources/1190201776262/2012/06/18/8.pdf}{普林斯顿大学招聘之学术报告系列}

\href{http://history.ccf.org.cn/resources/1190201776262/2012/06/18/10.pdf}{计算机改变了学习和记忆方式}

\href{http://history.ccf.org.cn/resources/1190201776262/2012/06/18/9.pdf}{聪明的傻瓜系列}

\href{http://history.ccf.org.cn/resources/1190201776262/2012/06/18/11.pdf}{我是一个普通清华人}

\href{http://history.ccf.org.cn/resources/1190201776262/2012/06/18/12.pdf}{国际互联网治理热点问题综述}

\href{http://history.ccf.org.cn/resources/1190201776262/2012/06/18/13.pdf}{社会学中的社会网络}

\href{http://history.ccf.org.cn/resources/1190201776262/2012/06/18/14.pdf}{从日志数据中挖掘过程知识}

\subsection{动态}
\href{http://history.ccf.org.cn/resources/1190201776262/2012/06/18/16.pdf}{国际数据工程会议}

\href{http://history.ccf.org.cn/resources/1190201776262/2012/06/18/15.pdf}{研究自己感兴趣并希望解决的问题——访图灵奖获得者查尔斯·泰克}

\subsection{译文}
\href{http://history.ccf.org.cn/resources/1190201776262/2012/06/18/17.pdf}{跨越软件教育鸿沟}

\subsection{特别报道}
\href{http://history.ccf.org.cn/resources/1190201776262/2012/06/18/19.pdf}{对社会和他人的贡献是检验成功的重要标准}

\href{http://history.ccf.org.cn/resources/1190201776262/2012/06/18/18.pdf}{青年科技工作者的价值与成功——2012 CCF青年精英大会侧记}


\section{\href{http://history.ccf.org.cn/sites/ccf/jsjtbbd.jsp?contentId=2674572570143}{\textbf{2012年第5期(总第75期)}}}
本期专题是移动社交网络。近期,移动社交网络成为了研究热点。伴随着SoLoMo概念的风靡,国内外创业公司也开始关注这个领域。到底什么是移动社交网络?与传统社交网络有什么不同?又带来了哪些新的研究课题?为此,本期专题邀请了多位专家对移动社交网络及其未来发展进行了探讨。文章内容十分丰富,不仅选题是目前的热点,而且作者的覆盖面也很广。
\subsection{专题}
\href{http://history.ccf.org.cn/resources/1190201776262/2012/05/18/6.pdf}{移动轨迹数据分析与智慧城市}

\href{http://history.ccf.org.cn/resources/1190201776262/2012/05/18/7.pdf}{社交媒体中的时空轨迹模式挖掘}

\href{http://history.ccf.org.cn/resources/1190201776262/2012/05/18/1.pdf}{移动社交网络}

\href{http://history.ccf.org.cn/resources/1190201776262/2012/05/18/4.pdf}{移动社交网络中的用户行为预测模型}

\href{http://history.ccf.org.cn/resources/1190201776262/2012/05/18/5.pdf}{移动社交网络与用户位置}

\href{http://history.ccf.org.cn/resources/1190201776262/2012/05/18/2.pdf}{瞬时社交网络——从在线社交网络到移动社交网络}

\href{http://history.ccf.org.cn/resources/1190201776262/2012/05/18/3.pdf}{移动社交网络中的感知计算模型、平台与实践}

\subsection{专栏}
\href{http://history.ccf.org.cn/resources/1190201776262/2012/05/18/8.pdf}{保研学生问题多}

\href{http://history.ccf.org.cn/resources/1190201776262/2012/05/18/12.pdf}{心理生理计算——一种基于交叉学科的计算模式}

\href{http://history.ccf.org.cn/resources/1190201776262/2012/05/18/9.pdf}{系统设计黄金法则——简单之美}

\href{http://history.ccf.org.cn/resources/1190201776262/2012/05/18/10.pdf}{运行时系统:多核/众核时代的新挑战}

\href{http://history.ccf.org.cn/resources/1190201776262/2012/05/18/11.pdf}{自然和谐的人机交互}

\href{http://history.ccf.org.cn/resources/1190201776262/2012/05/18/13.pdf}{JavaScript或成主导的编程语言}

\subsection{动态}
\href{http://history.ccf.org.cn/resources/1190201776262/2012/05/18/14.pdf}{用算法解决现实问题——访图灵奖获得者罗伯特·塔扬教授}

\href{http://history.ccf.org.cn/resources/1190201776262/2012/05/18/15.pdf}{IEEE 2011可视化周}

\subsection{译文}
\href{http://history.ccf.org.cn/resources/1190201776262/2012/05/18/16.pdf}{桌面:用于普适计算的交互式水平显示}

\subsection{学会论坛}
\href{http://history.ccf.org.cn/resources/1190201776262/2012/05/18/17.pdf}{重访欧洲ACM  法布里奇奥(Fabrizio Gagliardi)}

\href{http://history.ccf.org.cn/resources/1190201776262/2012/05/18/18.pdf}{CCF代表团访问法德}


\section{\href{http://history.ccf.org.cn/sites/ccf/jsjtbbd.jsp?contentId=2669042924594}{\textbf{2012年第4期(总第74期)}}}
本期专题是“社会媒体”。如何研究面向社会属性的媒体信息处理方法,如何揭示信息属性对社交网络的形成及作用,如何发现社会媒体在信息属性和社会属性融合后的内在机理,这些都亟待社会媒体的研究人员进行研究和解决。为此,本期专题组织了6篇文章,作者来自国内几家主要研究单位,介绍了当前的研究动态以及未来将面临的挑战和机遇。
中国科学院自动化研究所彭思龙研究员的专栏文章“工程和学术的完美结合:产品化”,谈的体会发人深省。他自己办过公司,他的经验对正在办公司或做产品的科技人员有借鉴意义。安天实验室肖新光的“互联网用户泛隐私安全热点问题回顾与浅析”一文,观点十分鲜明,有棱有角,对互联网用户泛隐私安全的警示发人深省。...
\subsection{专题}
\href{http://history.ccf.org.cn/resources/1190201776262/2012/04/16/7.pdf}{社会媒体云计算}

\href{http://history.ccf.org.cn/resources/1190201776262/2012/04/16/3.pdf}{社会媒体内容分析}

\href{http://history.ccf.org.cn/resources/1190201776262/2012/04/16/6.pdf}{面向社会媒体的舆情分析}

\href{http://history.ccf.org.cn/resources/1190201776262/2012/04/16/1.pdf}{社会媒体}

\href{http://history.ccf.org.cn/resources/1190201776262/2012/04/16/5.pdf}{网络化社交媒体的传播复杂性}

\href{http://history.ccf.org.cn/resources/1190201776262/2012/04/16/2.pdf}{社会多媒体计算}

\href{http://history.ccf.org.cn/resources/1190201776262/2012/04/16/4.pdf}{社会网络与图匹配查询}

\subsection{专栏}
\href{http://history.ccf.org.cn/resources/1190201776262/2012/04/16/9.pdf}{工程和学术的完美结合:产品化}

\href{http://history.ccf.org.cn/resources/1190201776262/2012/04/16/11.pdf}{可信编译器构造方法研究}

\href{http://history.ccf.org.cn/resources/1190201776262/2012/04/16/8.pdf}{智能驱动的下一代安全}

\href{http://history.ccf.org.cn/resources/1190201776262/2012/04/16/10.pdf}{互联网用户泛隐私安全热点问题回顾与浅析}

\subsection{动态}
\href{http://history.ccf.org.cn/resources/1190201776262/2012/04/16/13.pdf}{第九十一届美国交通研究学会}

\href{http://history.ccf.org.cn/resources/1190201776262/2012/04/16/12.pdf}{徐家福教授访谈录}

\subsection{译文}
\href{http://history.ccf.org.cn/resources/1190201776262/2012/04/16/14.pdf}{网络中的社团、模块与大规模结构的研究}

\subsection{学会论坛}
\href{http://history.ccf.org.cn/resources/1190201776262/2012/04/16/16.pdf}{日本工程教育认证的几个特点}

\href{http://history.ccf.org.cn/resources/1190201776262/2012/04/16/15.pdf}{2012年会员发展战略}

\subsection{特别报道}
\href{http://history.ccf.org.cn/resources/1190201776262/2012/04/16/18.pdf}{见证中国IDC之路}

\href{http://history.ccf.org.cn/resources/1190201776262/2012/04/16/17.pdf}{创新数据中心}


\section{\href{http://history.ccf.org.cn/sites/ccf/jsjtbbd.jsp?contentId=2663704737281}{\textbf{2012年第3期(总第73期)}}}
万物相连万物生。物联网和嵌入式系统,我们真的透彻理解了吗?物联网强调的广泛而全面的互联互通,不仅人与人要交流,物与物也要互通。只有有了互联互通,物联网的感知才更透彻更具洞察力;只有有了更透彻的感知,才有更综合更深入的智能。物联网的真正成功,实际上是系统的真正成功。本期专题,借助中国计算机学会(CCF)和ACM联合提供的第26期学科前沿讲习班(ADL)这个平台,邀请图灵奖获得者斯法克斯(Joseph Sifakis)教授等国外的5名学者,根据他们即将在3月30日讲授的内容写了4篇文章,系统地介绍了物联网与嵌入式系统,探索物联网的未知……
\subsection{专题}
\href{http://history.ccf.org.cn/resources/1190201776262/2012/03/16/1.pdf}{万物相联万物生}

\href{http://history.ccf.org.cn/resources/1190201776262/2012/03/16/4.pdf}{驯服弗兰肯斯坦博士:设计网络嵌入式系统}

\href{http://history.ccf.org.cn/resources/1190201776262/2012/03/16/3.pdf}{网络嵌入式系统可靠架构搭建}

\href{http://history.ccf.org.cn/resources/1190201776262/2012/03/16/5.pdf}{信息物理系统的网络挑战}

\href{http://history.ccf.org.cn/resources/1190201776262/2012/03/16/2.pdf}{计算机科学的愿景——系统发展观}

\subsection{专栏}
\href{http://history.ccf.org.cn/resources/1190201776262/2012/03/16/9.pdf}{下一代互联网与IPv6过渡}

\href{http://history.ccf.org.cn/resources/1190201776262/2012/03/16/8.pdf}{中国高性能计算机TOP100十周年}

\href{http://history.ccf.org.cn/resources/1190201776262/2012/03/16/7.pdf}{工程和学术的完美结合}

\href{http://history.ccf.org.cn/resources/1190201776262/2012/03/16/6.pdf}{从CSDN数据泄露谈网络身份管理}

\subsection{动态}
\href{http://history.ccf.org.cn/resources/1190201776262/2012/03/16/12.pdf}{中国计算机学会推荐国际学术会议(四)}

\href{http://history.ccf.org.cn/resources/1190201776262/2012/03/16/11.pdf}{国际会议介绍:ACM多媒体国际会议}

\href{http://history.ccf.org.cn/resources/1190201776262/2012/03/16/15.pdf}{CCF海外理事李凯当选美国工程院院士}

\href{http://history.ccf.org.cn/resources/1190201776262/2012/03/16/10.pdf}{人物专访:杨芙清教授访谈录}

\subsection{译文}
\href{http://history.ccf.org.cn/resources/1190201776262/2012/03/16/13.pdf}{美国能源部中的数据密集型科学:案例研究与未来挑战}

\subsection{学会论坛}
\href{http://history.ccf.org.cn/resources/1190201776262/2012/03/16/14.pdf}{中国计算机学会理事会条例}


\section{\href{http://history.ccf.org.cn/sites/ccf/jsjtbbd.jsp?contentId=2658329348582}{\textbf{2012年第2期(总第72期)}}}
2012年1月7日,2011 CCF颁奖大会在北京隆重举行。会上颁发了CCF终身成就奖、CCF青年科学奖、CCF优秀博士学位论文奖以及CCF杰出贡献奖、C卓越服务奖、优秀工作委员会奖和CCF会员发展优秀单位和个人奖。同时,还举行了CCF第十届理事会理事长就职仪式。
本期的专题是“嵌入式系统软件前沿”。嵌入式系统的早期定义是指对仪器、机器和工厂运作进行监控的设备,通常表现为针对特定应用的计算机系统。经过发展,现在嵌入式系统有了与互联、融合、云计算等相结合的特点,嵌入式软件的比重也越来越大。本期专题文章,从五个不同方面阐述了嵌入式系统,包括软硬件协同设计、嵌入式软件测试、验证和调试、可高信系统。
\subsection{专题}
\href{http://history.ccf.org.cn/resources/1190201776262/2012/02/14/6.pdf}{嵌入式系统调试浅谈}

\href{http://history.ccf.org.cn/resources/1190201776262/2012/02/14/7.pdf}{安全关键系统的嵌入式软件}

\href{http://history.ccf.org.cn/resources/1190201776262/2012/02/14/2.pdf}{嵌入式系统软件前沿}

\href{http://history.ccf.org.cn/resources/1190201776262/2012/02/14/3.pdf}{片上系统芯片的软硬件协同设计}

\href{http://history.ccf.org.cn/resources/1190201776262/2012/02/14/4.pdf}{嵌入式软件自动分析和验证技术}

\href{http://history.ccf.org.cn/resources/1190201776262/2012/02/14/5.pdf}{嵌入式系统软件测试}

\subsection{专栏}
\href{http://history.ccf.org.cn/resources/1190201776262/2012/02/14/8.pdf}{背后的背后}

\href{http://history.ccf.org.cn/resources/1190201776262/2012/02/14/9.pdf}{社会计算还是社会化计算}

\href{http://history.ccf.org.cn/resources/1190201776262/2012/02/14/10.pdf}{文件系统的发展脉络}

\subsection{动态}
\href{http://history.ccf.org.cn/resources/1190201776262/2012/02/14/13.pdf}{IEEE“今日计算”技术热点}

\href{http://history.ccf.org.cn/resources/1190201776262/2012/02/14/11.pdf}{人物专访:英特尔中国研究院院长方之熙}

\href{http://history.ccf.org.cn/resources/1190201776262/2012/03/19/12.pdf}{ACM 地理信息系统国际会议}

\subsection{译文}
\href{http://history.ccf.org.cn/resources/1190201776262/2012/02/14/14.pdf}{研究型大学里的教学型教师}

\subsection{特别报道}
\href{http://history.ccf.org.cn/resources/1190201776262/2012/02/14/1.pdf}{中国计算机学会举行2011颁奖大会}


\section{\href{http://history.ccf.org.cn/sites/ccf/jsjtbbd.jsp?contentId=2653836525553}{\textbf{2012年第1期(总第71期)}}}
本期专题为2011中国计算机大会特邀报告。中国计算机大会是一个高端的学术会议,它重点探讨计算及及信息科学技术领域最新进展和宏观发展趋势,展示计算领域学术界、企业界最重要的学术、技术成果。大会邀请到了世界上计算领域内资深的专家作大会特邀报告,有CCF理事长、中国工程院院士李国杰,2007年ACM图灵奖获得者Joseph Sifakis,原美国国家科学基金会副主席兼信息学部主任Peter Freeman,ACM前主席,英国南安普顿大学教授Wendy Hall,深圳大学教授、中国科学院院士陈国良,香港科技大学校长陈繁昌,中国宽带产业基金董事长田溯宁,华为技术有限公司副总裁李三琦,中科院深圳先进技术研究院院长樊建平,美国天普大学计算机系主任吴杰,加拿大多伦...
\subsection{专题}
\href{http://history.ccf.org.cn/resources/1190201776262/2012/01/19/1.pdf}{切实加强前瞻性研究}

\href{http://history.ccf.org.cn/resources/1190201776262/2012/01/19/2.pdf}{计算机科学的远景——从系统角度探讨}

\href{http://history.ccf.org.cn/resources/1190201776262/2012/01/19/3.pdf}{计算的未来}

\href{http://history.ccf.org.cn/resources/1190201776262/2012/01/19/4.pdf}{迈向智能万维网}

\href{http://history.ccf.org.cn/resources/1190201776262/2012/01/19/6.pdf}{计算思维}

\href{http://history.ccf.org.cn/resources/1190201776262/2012/01/19/7.pdf}{华为云解决方案和技术挑战}

\href{http://history.ccf.org.cn/resources/1190201776262/2012/01/19/9.pdf}{云计算创新与中国机遇}

\href{http://history.ccf.org.cn/resources/1190201776262/2012/01/19/11.pdf}{云体系结构下的业务流程管理}

\href{http://history.ccf.org.cn/resources/1190201776262/2012/01/19/8.pdf}{关于计算机与信息科学教育的思考}

\href{http://history.ccf.org.cn/resources/1190201776262/2012/01/19/5.pdf}{中国计算机科学与技术之我见}

\href{http://history.ccf.org.cn/resources/1190201776262/2012/01/19/10.pdf}{深圳先进院可持续发展之路}

\subsection{专栏}
\href{http://history.ccf.org.cn/resources/1190201776262/2012/01/19/14.pdf}{如何提升计算机学科的专业价值}

\href{http://history.ccf.org.cn/resources/1190201776262/2012/01/19/12.pdf}{办好计算机系的三要素}

\href{http://history.ccf.org.cn/resources/1190201776262/2012/01/19/13.pdf}{混合智能:人工智能的新方向}

\subsection{动态}
\href{http://history.ccf.org.cn/resources/1190201776262/2012/01/19/15.pdf}{国际会议介绍:2011 ACM普适计算国际会议}

\href{http://history.ccf.org.cn/resources/1190201776262/2012/01/19/16.pdf}{国内会议介绍:汉字字形计算会议}

\href{http://history.ccf.org.cn/resources/1190201776262/2012/01/19/17.pdf}{中国计算机学会推荐国际学术会议(三)}

\subsection{译文}
\href{http://history.ccf.org.cn/resources/1190201776262/2012/07/23/18.pdf}{中国计算机技术会超越美国……,也许,某天……}

\subsection{学会论坛}
\href{http://history.ccf.org.cn/resources/1190201776262/2012/01/19/19.pdf}{中国计算机学会会员条例}


\section{\href{http://history.ccf.org.cn/sites/ccf/jsjtbbd.jsp?contentId=2651625261838}{\textbf{2011年第12期(总第70期)}}}
本期专题内容是“社会计算”。如今海量的社会数据催生了社会计算这门新兴的学科,利用计算技术和社会科学的交叉,帮助人们认识社会规律、相互沟通与协作,利用群体智慧解决问题。本期专题组织了5篇文章,内容覆盖了社会计算的多个主要方向,深入浅出地描绘了社会计算当前研究动态,介绍了国内研究成果。
\subsection{专题}
\href{http://history.ccf.org.cn/resources/1190201776262/2012/01/06/6.pdf}{从语言计算到社会计算}

\href{http://history.ccf.org.cn/resources/1190201776262/2012/01/06/1.pdf}{社会计算}

\href{http://history.ccf.org.cn/resources/1190201776262/2012/01/06/2.pdf}{社会计算的研究现状与未来}

\href{http://history.ccf.org.cn/resources/1190201776262/2012/01/06/3.pdf}{社会信息网络中的社区分析}

\href{http://history.ccf.org.cn/resources/1190201776262/2012/01/06/4.pdf}{社区媒体信息传播机制}

\href{http://history.ccf.org.cn/resources/1190201776262/2012/01/06/5.pdf}{用社会化方法计算社会}

\subsection{专栏}
\href{http://history.ccf.org.cn/resources/1190201776262/2012/01/06/9.pdf}{云计算安全挑战与实践}

\href{http://history.ccf.org.cn/resources/1190201776262/2012/01/06/7.pdf}{移动互联网时代的位置服务}

\href{http://history.ccf.org.cn/resources/1190201776262/2012/01/06/8.pdf}{机会物联——兼谈物联网的社会性}

\subsection{动态}
\href{http://history.ccf.org.cn/resources/1190201776262/2012/01/06/10.pdf}{国际会议介绍:ACM SIGKDD数据挖掘会议}

\subsection{译文}
\href{http://history.ccf.org.cn/resources/1190201776262/2012/01/06/11.pdf}{面向服务的架构成熟度}

\subsection{特别报道}
\href{http://history.ccf.org.cn/resources/1190201776262/2012/01/06/14.pdf}{中国计算机学会监事会条例}

\href{http://history.ccf.org.cn/resources/1190201776262/2012/01/06/15.pdf}{2011年目录}

\href{http://history.ccf.org.cn/resources/1190201776262/2012/01/06/13.pdf}{中国计算机学会章程}

\href{http://history.ccf.org.cn/resources/1190201776262/2012/01/06/12.pdf}{会员代表大会的历史性贡献}


\section{\href{http://history.ccf.org.cn/sites/ccf/jsjtbbd.jsp?contentId=2643122951271}{\textbf{2011年第11期(总第69期)}}}
本期专题是《人机交互》。自然用户界面使人们不必再受鼠标和键盘限制,可以通过手势、言语、笔迹等不同手段输入信息,甚至是通过非主动方式,如生理、脑电信号输入信息,人们可以融合多个通道信息进行人机交互。这些新型人机交互技术给传统人机界面带来了彻底改变,使计算机和用户间的界线变得模糊。然而,从一个界面时代的发展及至成熟而言,目前关于自然用户界面的研究仍然处在刚刚起步的阶段,在基础研究方面的成果非常有限。本期专题邀请了多位人机交互领域的专家,围绕人机交互的基本概念、关键技术和方法撰文,从不同角度进行介绍和探讨。
\subsection{专题}
\href{http://history.ccf.org.cn/resources/1190201776262/2011/12/01/1.pdf}{人机交互}

\href{http://history.ccf.org.cn/resources/1190201776262/2011/12/01/2.pdf}{自然用户界面的基础研究}

\href{http://history.ccf.org.cn/resources/1190201776262/2011/12/01/4.pdf}{为什么需要超大交互表面?}

\href{http://history.ccf.org.cn/resources/1190201776262/2011/12/01/5.pdf}{多设备共享的智能交互}

\href{http://history.ccf.org.cn/resources/1190201776262/2011/12/01/6.pdf}{多模态融合的人机对话系统}

\href{http://history.ccf.org.cn/resources/1190201776262/2011/12/01/3.pdf}{自然用户界面自然在哪儿?}

\subsection{专栏}
\href{http://history.ccf.org.cn/resources/1190201776262/2011/12/01/11.pdf}{下一代e-Learning系统}

\href{http://history.ccf.org.cn/resources/1190201776262/2011/12/01/10.pdf}{基于增强现实的人机交互}

\href{http://history.ccf.org.cn/resources/1190201776262/2011/12/01/9.pdf}{乔布斯与人机交互}

\href{http://history.ccf.org.cn/resources/1190201776262/2011/12/01/7.pdf}{乔布斯对我们有什么意义?}

\href{http://history.ccf.org.cn/resources/1190201776262/2011/12/01/8.pdf}{乔布斯的“失败”意义}

\subsection{动态}
\href{http://history.ccf.org.cn/resources/1190201776262/2011/12/01/13.pdf}{ACM人工智能专委会}

\href{http://history.ccf.org.cn/resources/1190201776262/2011/12/01/12.pdf}{CCF海外杰出贡献奖获得者李凯教授}

\href{http://history.ccf.org.cn/resources/1190201776262/2011/12/01/14.pdf}{中国计算机学会推荐国际学术会议(二)}

\subsection{译文}
\href{http://history.ccf.org.cn/resources/1190201776262/2011/12/01/17.pdf}{自然用户界面不自然}

\href{http://history.ccf.org.cn/resources/1190201776262/2011/12/01/18.pdf}{手势界面:可用性的倒退}

\subsection{特别报道}
\href{http://history.ccf.org.cn/resources/1190201776262/2011/12/01/16.pdf}{ACM SIGGRAPH ASIA 2011将在香港举行}

\href{http://history.ccf.org.cn/resources/1190201776262/2011/12/01/15.pdf}{“云计算”时代的海量数据存储与管理}


\section{\href{http://history.ccf.org.cn/sites/ccf/jsjtbbd.jsp?contentId=2637593328494}{\textbf{2011年第10期(总第68期)}}}
本期专题是《计算系统虚拟化——评测和应用》,将继续为读者展示国家“973计划”项目“计算系统虚拟化基础理论与方法研究”在虚拟计算系统的性能评测和应用示范、验证等方面的研究进展和成果。虚拟化与云计算的时代已经来临。虚拟化技术是构建云基础架构不可或缺的关键技术之一,但是仍存在着诸多问题,需要花时间进行研究。希望本期专题的文章为读者对虚拟化和云计算的研究带来一些启发。
\subsection{专题}
\href{http://history.ccf.org.cn/resources/1190201776262/2011/10/17/3.pdf}{面向高效能计算机的虚拟化技术}

\href{http://history.ccf.org.cn/resources/1190201776262/2011/10/17/5.pdf}{多机虚拟执行技术的应用研究}

\href{http://history.ccf.org.cn/resources/1190201776262/2011/10/17/1.pdf}{计算系统虚拟化——评测和应用}

\href{http://history.ccf.org.cn/resources/1190201776262/2011/10/17/2.pdf}{虚拟计算系统性能评测}

\href{http://history.ccf.org.cn/resources/1190201776262/2011/10/17/4.pdf}{云仿真中虚拟化技术的应用}

\subsection{专栏}
\href{http://history.ccf.org.cn/resources/1190201776262/2011/10/17/6.pdf}{脚踏实地,不慕虚荣——与CCF YOCSEF成员座谈}

\href{http://history.ccf.org.cn/resources/1190201776262/2011/10/17/9.pdf}{HPC Cloud——新兴的高性能计算模式}

\href{http://history.ccf.org.cn/resources/1190201776262/2011/10/17/7.pdf}{科研中的布局}

\href{http://history.ccf.org.cn/resources/1190201776262/2011/10/17/8.pdf}{多尺度用户界面}

\subsection{动态}
\href{http://history.ccf.org.cn/resources/1190201776262/2011/10/17/12.pdf}{中国计算机学会推荐国际学术刊物(三)}

\href{http://history.ccf.org.cn/resources/1190201776262/2011/10/17/10.pdf}{国际会议介绍:ACM人机交互国际会议}

\href{http://history.ccf.org.cn/resources/1190201776262/2011/10/17/11.pdf}{国际会议介绍:ACM信息检索国际会议}

\href{http://history.ccf.org.cn/resources/1190201776262/2011/10/17/13.pdf}{中国计算机学会推荐国际学术会议(一)}

\subsection{译文}
\href{http://history.ccf.org.cn/resources/1190201776262/2011/10/17/14.pdf}{搜索需要一场变革 奥伦·艾齐厄尼}

\subsection{学会论坛}
\href{http://history.ccf.org.cn/resources/1190201776262/2011/10/17/17.pdf}{我为什么要参加竞选}

\href{http://history.ccf.org.cn/resources/1190201776262/2011/10/17/16.pdf}{换届选举体现学会的“三民主义”}

\href{http://history.ccf.org.cn/resources/1190201776262/2011/10/17/18.pdf}{“五会主义”}

\href{http://history.ccf.org.cn/resources/1190201776262/2011/10/17/19.pdf}{服务是行动}

\subsection{特别报道}
\href{http://history.ccf.org.cn/resources/1190201776262/2011/10/17/15.pdf}{“社会网络”综述——CCF YOCSEF学术报告会}


\section{\href{http://history.ccf.org.cn/sites/ccf/jsjtbbd.jsp?contentId=2632927725900}{\textbf{2011年第09期(总第67期)}}}
本期专题是《计算系统虚拟化——原理和技术》。计算系统虚拟化是当今的技术热点。它可灵活地构建满足多种应用需求的计算环境,为用户提供个性化和普适化的计算资源使用环境。本刊邀请了国家973计划“计算系统虚拟化基础理论与方法研究”项目团队的专家,分“原理和技术”与“评测和应用”两个主题,撰文介绍虚拟化技术的研究进展以及该项目在基础理论、关键技术、评测验证和应用示范等方面取得的一系列研究成果。这些文章将分两期呈现给读者。
\subsection{专题}
\href{http://history.ccf.org.cn/resources/1190201776262/2011/09/20/1.pdf}{计算系统虚拟化——原理和技术}

\href{http://history.ccf.org.cn/resources/1190201776262/2011/09/20/2.pdf}{二进制程序的动态优化}

\href{http://history.ccf.org.cn/resources/1190201776262/2011/09/20/3.pdf}{单计算系统资源虚拟化方法}

\href{http://history.ccf.org.cn/resources/1190201776262/2011/09/20/4.pdf}{桌面虚拟化}

\href{http://history.ccf.org.cn/resources/1190201776262/2011/09/20/5.pdf}{虚拟化系统安全}

\subsection{专栏}
\href{http://history.ccf.org.cn/resources/1190201776262/2011/09/21/6.pdf}{从百度事件看诚信和互联网价值——CCF YOCSEF特别论坛}

\href{http://history.ccf.org.cn/resources/1190201776262/2011/09/21/7.pdf}{一名系统研究者的攀登之路}

\href{http://history.ccf.org.cn/resources/1190201776262/2011/09/21/8.pdf}{莱斯利·瓦里安特获2010年ACM图灵奖}

\href{http://history.ccf.org.cn/resources/1190201776262/2011/09/21/9.pdf}{搜索引擎与社交网络的博弈}

\subsection{动态}
\href{http://history.ccf.org.cn/resources/1190201776262/2011/09/21/10.pdf}{人物专访:CCF海外杰出贡献奖获得者赵伟教授}

\href{http://history.ccf.org.cn/resources/1190201776262/2011/09/21/11.pdf}{国际会议介绍:ACM SIGMOD数据管理国际会议}

\href{http://history.ccf.org.cn/resources/1190201776262/2011/09/21/12.pdf}{中国计算机学会推荐国际学术刊物(二)}

\subsection{译文}
\href{http://history.ccf.org.cn/resources/1190201776262/2011/09/21/13.pdf}{新型机器脑}

\subsection{学会论坛}
\href{http://history.ccf.org.cn/resources/1190201776262/2011/09/21/14.pdf}{CCF通过理事会选举条例修订版}


\section{\href{http://history.ccf.org.cn/sites/ccf/jsjtbbd.jsp?contentId=2626706918868}{\textbf{2011年第08期(总第66期)}}}
本期专题是《可穿戴计算》。真实自然地增强人与计算机交互的手段是可穿戴计算需要解决的问题。其中的学问涉及到对人感知能力的理解和对计算机人机接口极限的应用。如何让计算机与人的联系更加紧密,让计算如影随行?本期专题邀请该领域的专家组织了6篇文章,力图全面介绍可穿戴计算的主要研究和应用领域。
\subsection{专题}
\href{http://history.ccf.org.cn/resources/1190201776262/2011/08/16/6.pdf}{面向移动增强现实的视透型头戴显示器}

\href{http://history.ccf.org.cn/resources/1190201776262/2011/08/15/3.pdf}{基于生理计算的人机交互}

\href{http://history.ccf.org.cn/resources/1190201776262/2011/08/15/1.pdf}{可穿戴计算}

\href{http://history.ccf.org.cn/resources/1190201776262/2011/08/15/2.pdf}{可穿戴计算的发展}

\href{http://history.ccf.org.cn/resources/1190201776262/2011/08/15/4.pdf}{可穿戴计算在欧洲——wearIT@work项目}

\href{http://history.ccf.org.cn/resources/1190201776262/2011/08/15/5.pdf}{可穿戴健康监护系统在韩国}

\subsection{专栏}
\href{http://history.ccf.org.cn/resources/1190201776262/2011/08/16/7.pdf}{中国IT如何进行原始创新?——CCF YOCSEF专题论坛}

\href{http://history.ccf.org.cn/resources/1190201776262/2011/08/16/9.pdf}{在美国做博士后的工作体会}

\href{http://history.ccf.org.cn/resources/1190201776262/2011/08/16/10.pdf}{多点思想,少点技巧}

\href{http://history.ccf.org.cn/resources/1190201776262/2011/08/16/8.pdf}{创新就是解决现实问题}

\href{http://history.ccf.org.cn/resources/1190201776262/2011/08/16/12.pdf}{多核体系下的并行程序调试}

\href{http://history.ccf.org.cn/resources/1190201776262/2011/08/16/11.pdf}{内容App化,App社交化}

\subsection{动态}
\href{http://history.ccf.org.cn/resources/1190201776262/2011/08/16/13.pdf}{国际会议介绍:计算机支持协同工作国际会议}

\href{http://history.ccf.org.cn/resources/1190201776262/2011/08/16/15.pdf}{中国计算机学会推荐国际学术刊物(一)}

\href{http://history.ccf.org.cn/resources/1190201776262/2011/08/16/14.pdf}{研究进展报告:2010理论计算机科学年会报告}

\subsection{译文}
\href{http://history.ccf.org.cn/resources/1190201776262/2011/08/16/16.pdf}{利用查询数据探测流感传播}

\subsection{学会论坛}
\href{http://history.ccf.org.cn/resources/1190201776262/2011/08/16/17.pdf}{CCF通过两部规章}

\subsection{特别报道}
\href{http://history.ccf.org.cn/resources/1190201776262/2011/08/17/See-Through Head Worn Displays for Mobile Augmented Reality.pdf}{See-Through Head Worn Displays for Mobile Augmented Reality}

\href{http://history.ccf.org.cn/resources/1190201776262/2011/08/17/Europe's effort in Wearable Computing - wearIT@work.pdf}{Europe's effort in Wearable Computing - wearIT@work.pdf}

\href{http://history.ccf.org.cn/resources/1190201776262/2011/08/17/The Past, Present, and Future of the wearable health care system.pdf}{The Past, Present, and Future of the wearable health care system.pdf}


\section{\href{http://history.ccf.org.cn/sites/ccf/jsjtbbd.jsp?contentId=2621220707876}{\textbf{2011年第07期(总第65期)}}}
本期专题是《数据密集型计算》。数据是宝贵的财富,而从海量数据中获取信息却要让人们花费更大代价,以至于现有的计算机处理能力都不能满足数据处理需求。结合海量数据特征,研究者给出了许多解决方案。本期专题邀请了几位专家介绍了数据密集型计算的挑战、系统结构、编程模型、应用和数据中心的建设。
\subsection{专题}
\href{http://history.ccf.org.cn/resources/1190201776262/2011/07/14/1.pdf}{数据密集型计算}

\href{http://history.ccf.org.cn/resources/1190201776262/2011/07/14/2.pdf}{数据密集型计算系统结构}

\href{http://history.ccf.org.cn/resources/1190201776262/2011/07/14/3.pdf}{数据爆炸时代的技术变革}

\href{http://history.ccf.org.cn/resources/1190201776262/2011/07/14/4.pdf}{数据中心网络}

\href{http://history.ccf.org.cn/resources/1190201776262/2011/07/14/5.pdf}{云计算的编程模型与自适应资源管理}

\href{http://history.ccf.org.cn/resources/1190201776262/2011/07/14/6.pdf}{数据密集型大规模计算系统}

\href{http://history.ccf.org.cn/resources/1190201776262/2011/07/14/7.pdf}{高通量计算机的性能评价}

\subsection{专栏}
\href{http://history.ccf.org.cn/resources/1190201776262/2011/07/14/8.pdf}{呼唤我国的网络空间战略}

\href{http://history.ccf.org.cn/resources/1190201776262/2011/07/14/9.pdf}{普适计算的重定位与探讨}

\href{http://history.ccf.org.cn/resources/1190201776262/2011/07/14/10.pdf}{企业研究院管理框架之问}

\href{http://history.ccf.org.cn/resources/1190201776262/2011/07/14/11.pdf}{论文与项目}

\subsection{动态}
\href{http://history.ccf.org.cn/resources/1190201776262/2011/07/14/13.pdf}{历史回顾:深切怀念软件科学家仲萃豪老师}

\href{http://history.ccf.org.cn/resources/1190201776262/2011/07/14/12.pdf}{历史回顾:千代师表,人生楷模——怀念恩师萨师煊教授}

\subsection{译文}
\href{http://history.ccf.org.cn/resources/1190201776262/2011/07/14/14.pdf}{利用自动化测试发现可用性缺陷}

\subsection{学会论坛}
\href{http://history.ccf.org.cn/resources/1190201776262/2011/07/14/15.pdf}{换届选举是学会的一件大事}


\section{\href{http://history.ccf.org.cn/sites/ccf/jsjtbbd.jsp?contentId=2615820732256}{\textbf{2011年第06期(总第64期)}}}
本期专题是《电子服务与务联网》。“服务”与“计算”的深入融合形成了一个独特的多学科交叉的领域。现代服务业、服务学、电子服务、云计算、物联网、务联网以及服务工程等研究,推动了服务计算领域的发展,随之也产生了一系列新的研究方向和研究问题。本期专题围绕服务计算主题,邀请了几位专家从多个角度撰文介绍了电子服务和务联网的研究进展。
\subsection{专题}
\href{http://history.ccf.org.cn/resources/1190201776262/2011/06/13/1.pdf}{电子服务与务联网}

\href{http://history.ccf.org.cn/resources/1190201776262/2011/06/13/2.pdf}{未来互联网环境下的务联网}

\href{http://history.ccf.org.cn/resources/1190201776262/2011/06/13/3.pdf}{电子服务——从产业到学科}

\href{http://history.ccf.org.cn/resources/1190201776262/2011/06/13/4.pdf}{面向服务的企业}

\href{http://history.ccf.org.cn/resources/1190201776262/2011/06/13/5.pdf}{情境感知服务创造智慧生活}

\href{http://history.ccf.org.cn/resources/1190201776262/2011/06/13/6.pdf}{落地云计算}

\href{http://history.ccf.org.cn/resources/1190201776262/2011/06/13/7.pdf}{基于顾客参与的电子服务管理}

\subsection{专栏}
\href{http://history.ccf.org.cn/resources/1190201776262/2011/06/13/9.pdf}{用云计算重建虚拟世界}

\href{http://history.ccf.org.cn/resources/1190201776262/2011/06/13/8.pdf}{《网络空间国际战略》解读}

\href{http://history.ccf.org.cn/resources/1190201776262/2011/06/13/10.pdf}{互联网的开放与创新}

\href{http://history.ccf.org.cn/resources/1190201776262/2011/06/13/11.pdf}{洪堡奖学金}

\href{http://history.ccf.org.cn/resources/1190201776262/2011/06/13/12.pdf}{一个博士生的留学感受}

\subsection{动态}
\href{http://history.ccf.org.cn/resources/1190201776262/2011/06/13/13.pdf}{人物专访:图灵奖获得者清华大学姚期智教授}

\href{http://history.ccf.org.cn/resources/1190201776262/2011/06/13/14.pdf}{国际组织介绍:ACM空间信息专委会}

\subsection{译文}
\href{http://history.ccf.org.cn/resources/1190201776262/2011/06/13/16.pdf}{计算社会科学}

\href{http://history.ccf.org.cn/resources/1190201776262/2011/06/13/15.pdf}{既是媒介又是信息的微博}

\subsection{学会论坛}
\href{http://history.ccf.org.cn/resources/1190201776262/2011/06/13/17.pdf}{由志愿者双肩撑起的ACM}


\section{\href{http://history.ccf.org.cn/sites/ccf/jsjtbbd.jsp?contentId=2611154939231}{\textbf{2011年第05期(总第63期)}}}
本期专题是《生物特征计算》。生物特征计算旨在实现人体生物特征的自动感知与分析,建立生物特征与人的身份、情感、行为、健康状况以及美学评价等对应关系的可计算模型。发展高效的生物特征计算技术,可在“感知自身”与“感知世界”之间架起一座桥梁,为人与人、人与物理信息世界之间高效、安全、便捷的交互提供关键技术支撑。本期专题邀请了几位专家,从不同角度对生物特征计算进行了介绍和展望。
\subsection{专题}
\href{http://history.ccf.org.cn/resources/1190201776262/2011/05/17/6.pdf}{人脸美学生物特征计算}

\href{http://history.ccf.org.cn/resources/1190201776262/2011/05/17/2.pdf}{传统生物特征识别技术}

\href{http://history.ccf.org.cn/resources/1190201776262/2011/05/17/1.pdf}{生物特征计算(free)}

\href{http://history.ccf.org.cn/resources/1190201776262/2011/05/17/3.pdf}{生物特征识别家族的新成员}

\href{http://history.ccf.org.cn/resources/1190201776262/2011/05/17/4.pdf}{掌纹识别技术}

\href{http://history.ccf.org.cn/resources/1190201776262/2011/05/17/5.pdf}{医学生物特征计算}

\subsection{专栏}
\href{http://history.ccf.org.cn/resources/1190201776262/2011/05/17/7.pdf}{移动平台上的阅读软件开发(free)}

\href{http://history.ccf.org.cn/resources/1190201776262/2011/05/17/8.pdf}{利用虚拟化技术提升系统可用性(free)}

\href{http://history.ccf.org.cn/resources/1190201776262/2011/05/17/9.pdf}{移动对象数据库关键技术(free)}

\href{http://history.ccf.org.cn/resources/1190201776262/2011/05/17/10.pdf}{大学,谁比谁重要(free)}

\href{http://history.ccf.org.cn/resources/1190201776262/2011/05/17/11.pdf}{网络实名制,虚拟与真实的边界(free)}

\subsection{动态}
\href{http://history.ccf.org.cn/resources/1190201776262/2011/05/17/12.pdf}{人物专访:CCF海外杰出贡献奖获得者张晓东教授 丁治明   孟小峰(free)}

\href{http://history.ccf.org.cn/resources/1190201776262/2011/05/17/13.pdf}{国际会议介绍:ACM网络搜索与数据挖掘国际会议(free)}

\subsection{译文}
\href{http://history.ccf.org.cn/resources/1190201776262/2011/05/17/14.pdf}{云计算信任新挑战 凯斯•米勒   杰弗里•伍奥斯   菲尔•拉普兰特(译者  王国军)  (free)}

\subsection{学会论坛}
\href{http://history.ccf.org.cn/resources/1190201776262/2011/05/17/16.pdf}{关于ACM评奖委员会的工作(free)}


\section{\href{http://history.ccf.org.cn/sites/ccf/jsjtbbd.jsp?contentId=2606489336780}{\textbf{2011年第04期(总第62期)}}}
信息时代带来了复杂的海量信息,而人们对其理解和处理的能力却没有得到相应的提升。信息可视化与可视分析技术将数据映射为视觉符号,为人们提供了交互理解高维度、多层次、时空、动态、关系等复杂数据的手段。本期专题邀请了多位专家撰文,从不同角度介绍信息可视化和可视分析的研究现状及未来。
\subsection{专题}
\href{http://history.ccf.org.cn/resources/1190201776262/2011/04/20/1.pdf}{信息可视化与可视分析(free)}

\href{http://history.ccf.org.cn/resources/1190201776262/2011/04/20/7.pdf}{基于移动数据的可视化分析}

\href{http://history.ccf.org.cn/resources/1190201776262/2011/04/20/2.pdf}{树结构与网络结构的混合型可视化}

\href{http://history.ccf.org.cn/resources/1190201776262/2011/04/20/3.pdf}{高维数据可视化}

\href{http://history.ccf.org.cn/resources/1190201776262/2011/04/20/4.pdf}{复杂有序数据的可视化}

\href{http://history.ccf.org.cn/resources/1190201776262/2011/04/20/5.pdf}{可视化分析中的多媒体语义模型}

\href{http://history.ccf.org.cn/resources/1190201776262/2011/04/20/6.pdf}{可视化文本分析}

\href{http://history.ccf.org.cn/resources/1190201776262/2011/04/20/8.pdf}{城市环境可视分析技术}

\subsection{专栏}
\href{http://history.ccf.org.cn/resources/1190201776262/2011/04/21/12.pdf}{云数据管理与NoSQL运动(free)}

\href{http://history.ccf.org.cn/resources/1190201776262/2011/04/20/13.pdf}{方兴未艾的社会计算(free)}

\href{http://history.ccf.org.cn/resources/1190201776262/2011/04/20/14.pdf}{多核时代处理器设计初探(free)}

\href{http://history.ccf.org.cn/resources/1190201776262/2011/04/20/15.pdf}{留学生的科研动力来自哪里(free)}

\href{http://history.ccf.org.cn/resources/1190201776262/2011/04/20/9.pdf}{以创新促转型的2010世界软件产业(free)}

\href{http://history.ccf.org.cn/resources/1190201776262/2011/04/20/10.pdf}{从RSA 2011看明日安全(free)}

\href{http://history.ccf.org.cn/resources/1190201776262/2011/04/20/11.pdf}{计算机技术领域工程硕士培养质量认证方案初探(free)}

\subsection{动态}
\href{http://history.ccf.org.cn/resources/1190201776262/2011/04/20/16.pdf}{人物专访:CCF海外杰出贡献奖获得者倪明选教授(free)}

\href{http://history.ccf.org.cn/resources/1190201776262/2011/04/20/17.pdf}{研究进展报告:第7届全国WEB信息系统及其应用学术会议(free)}

\href{http://history.ccf.org.cn/resources/1190201776262/2011/04/20/18.pdf}{国际会议介绍:IEEE 数据挖掘国际会议(free)}

\subsection{译文}
\href{http://history.ccf.org.cn/resources/1190201776262/2011/04/20/19.pdf}{改变科学数据可视化的方式(free)}

\subsection{学会论坛}
\href{http://history.ccf.org.cn/resources/1190201776262/2011/04/20/20.pdf}{CCF理事会架构改革建议案(free)}


\section{\href{http://history.ccf.org.cn/sites/ccf/jsjtbbd.jsp?contentId=2601260996998}{\textbf{2011年第03期(总第61期)}}}
本期专题是《下一代互联网及其管理》。随着以IPv6为核心的下一代互联网的广泛部署,网络规模将进一步增大,对网络管理能力提出了更高的要求。未来的网络将越来越复杂,也越来越智能,但这种智能并非天然就具备,需要网络管理系统的密切配合。本期专题邀请了我国从事下一代互联网体系结构研究和运行与管理的专家,从多方位探讨了下一代互联网和未来互联网的发展方向和思路。希望对读者了解下一代互联网有所帮助。
\subsection{专题}
\href{http://history.ccf.org.cn/resources/1190201776262/2011/03/22/2.pdf}{下一代互联网及其管理(free)}

\href{http://history.ccf.org.cn/resources/1190201776262/2011/03/23/3.pdf}{未来互联网的自治化网络体系架构}

\href{http://history.ccf.org.cn/resources/1190201776262/2011/03/23/5.pdf}{下一代互联网管理体系结构}

\href{http://history.ccf.org.cn/resources/1190201776262/2011/03/23/4.pdf}{标识网络与管理}

\href{http://history.ccf.org.cn/resources/1190201776262/2011/03/23/6.pdf}{面向服务的网络管理}

\href{http://history.ccf.org.cn/resources/1190201776262/2011/03/23/7.pdf}{未来网络管理的业务感知技术}

\subsection{专栏}
\href{http://history.ccf.org.cn/resources/1190201776262/2011/03/23/9.pdf}{吸引海外高端人才  加快学科高地建设(free)}

\href{http://history.ccf.org.cn/resources/1190201776262/2011/03/23/10.pdf}{数字脚印与“社群智能”(free)}

\href{http://history.ccf.org.cn/resources/1190201776262/2011/03/23/8.pdf}{从应用角度看计算机行业未来发展(free)}

\href{http://history.ccf.org.cn/resources/1190201776262/2011/03/23/11.pdf}{计算技术多学科融合研究(free)}

\href{http://history.ccf.org.cn/resources/1190201776262/2011/03/23/12.pdf}{超启发式算法:跨领域的问题求解模式(free)}

\subsection{动态}
\href{http://history.ccf.org.cn/resources/1190201776262/2011/03/23/14.pdf}{国际组织介绍:ACM信息检索专委会(free)}

\href{http://history.ccf.org.cn/resources/1190201776262/2011/03/23/13.pdf}{研究团队介绍:清华大学EMC讲席教授组(free)}

\subsection{译文}
\href{http://history.ccf.org.cn/resources/1190201776262/2011/03/23/15.pdf}{虚拟化:是福还是祸?(free)}

\subsection{学会论坛}
\href{http://history.ccf.org.cn/resources/1190201776262/2011/03/23/16.pdf}{ACM推出新数字图书馆(free)}

\subsection{特别报道}
\href{http://history.ccf.org.cn/resources/1190201776262/2011/03/22/1.pdf}{心系计算机 情系计算机——我国计算机事业的先驱张效祥先生访谈录(free)}


\section{\href{http://history.ccf.org.cn/sites/ccf/jsjtbbd.jsp?contentId=2590507189137}{\textbf{2011年第02期(总第60期)}}}
2011年1月22日中国计算机学会在京召开了以“责任、创新、奉献”为主题的颁奖大会。会上颁发了2010年度CCF终身成就奖、CCF青年科学家奖和CCF优秀博士学位论文奖。《中国计算机学会通讯》将颁奖大会相关资料整理成文,让更多的读者分享大会盛况,让更多的读者了解大会精神。
本期专题是《媒体计算》。从结绳记事到文字,再到各种媒体的出现,人类记录和传播信息的手段不断向高级发展。网络和通信技术的发展,又给媒体传播带来了前所未有的途径和挑战,使多媒体信息处理技术越来越呈现海量化、语义化、多样化和社会化的特征。本期组织的7篇文章,希望能给读者了解媒体计算的发展趋势提供帮助。
专栏栏目刊登了4篇文章,其中《另类云计算和另类物联网...
\subsection{专题}
\href{http://history.ccf.org.cn/resources/1190201776262/2011/02/16/8.pdf}{数字媒体内容理解}

\href{http://history.ccf.org.cn/resources/1190201776262/2011/02/16/13.pdf}{Web社会网络分析}

\href{http://history.ccf.org.cn/resources/1190201776262/2011/02/16/7.pdf}{媒体计算(free)}

\href{http://history.ccf.org.cn/resources/1190201776262/2011/02/16/9.pdf}{跨媒体分析与检索}

\href{http://history.ccf.org.cn/resources/1190201776262/2011/02/16/10.pdf}{超大规模媒体存储与计算平台}

\href{http://history.ccf.org.cn/resources/1190201776262/2011/02/16/11.pdf}{多媒体内容认证}

\href{http://history.ccf.org.cn/resources/1190201776262/2011/02/16/12.pdf}{空间音频编码技术}

\subsection{特邀报告}
\href{http://history.ccf.org.cn/resources/1190201776262/2011/02/16/2.pdf}{CCF要成为产学研合作的纽带——中国计算机学会理事长}

\href{http://history.ccf.org.cn/resources/1190201776262/2011/02/16/3.pdf}{共同铸造明日的学术之星——微软亚洲研究院常务副院长}

\href{http://history.ccf.org.cn/resources/1190201776262/2011/02/16/1.pdf}{2010中国计算机学会颁奖大会在京隆重举行(free)}

\href{http://history.ccf.org.cn/resources/1190201776262/2011/02/16/6.pdf}{CCF颁发2010 CCF优秀博士学位论文奖}

\href{http://history.ccf.org.cn/resources/1190201776262/2011/02/16/4.pdf}{张效祥夏培肃获首届CCF终身成就奖}

\href{http://history.ccf.org.cn/resources/1190201776262/2011/02/16/5.pdf}{CCF颁发2010 CCF青年科学家奖}

\subsection{专栏}
\href{http://history.ccf.org.cn/resources/1190201776262/2011/02/16/14.pdf}{另类云计算和另类物联网(free)}

\href{http://history.ccf.org.cn/resources/1190201776262/2011/02/16/16.pdf}{查尔斯•萨克尔获2009年图灵奖(free)}

\href{http://history.ccf.org.cn/resources/1190201776262/2011/02/16/17.pdf}{学会做“减法”(free)}

\href{http://history.ccf.org.cn/resources/1190201776262/2011/02/16/15.pdf}{从传感网到物联网:微型操作系统的现状与未来(free)}

\subsection{动态}
\href{http://history.ccf.org.cn/resources/1190201776262/2011/02/16/18.pdf}{第27届CCF中国数据库学术会议(free)}

\href{http://history.ccf.org.cn/resources/1190201776262/2011/02/16/19.pdf}{全国系统虚拟化技术研讨会(free)}

\subsection{译文}
\href{http://history.ccf.org.cn/resources/1190201776262/2011/02/16/20.pdf}{学术会议的健康和重大创意论文的缺乏(free)}

\subsection{学会论坛}
\href{http://history.ccf.org.cn/resources/1190201776262/2011/02/16/23.pdf}{2010 CCF专委评估结果发布(free)}

\href{http://history.ccf.org.cn/resources/1190201776262/2011/02/16/24.pdf}{ACM如何出版期刊(free)}

\subsection{特别报道}
\href{http://history.ccf.org.cn/resources/1190201776262/2011/02/16/21.pdf}{“云计算”时代的国家信息安全战略(free)}

\href{http://history.ccf.org.cn/resources/1190201776262/2011/02/16/22.pdf}{创新型人才是怎样炼成的?(free)}


\section{\href{http://history.ccf.org.cn/sites/ccf/jsjtbbd.jsp?contentId=2590507187982}{\textbf{2011年第01期(总第59期)}}}
本期的专题是《CCF CNCC2010特邀报告》。2010年10月11~12日,中国计算机大会在杭州举行。大会以“网联世界、计算无限”为主题,邀请了张尧学、怀进鹏、约翰•怀特(ACM CEO)、郑纬民、马云、张良杰、张晓东、孙凝晖、吴朝晖9位国内外知名学者和专家,就基础软件、网络计算、服务计算、计算系统面临的挑战等问题作了特邀报告。这些内容涵盖了当今计算技术研究热点和发展趋势,是计算技术领域非常重要的信息。《中国计算机学会通讯》将大会报告整理成文章刊登,让更多读者分享他们的见解和观点。
\subsection{专题}
\href{http://history.ccf.org.cn/resources/1190201776262/2011/01/18/201101-1.pdf}{关于我国基础软件与CPU发展的几点思考(free)}

\href{http://history.ccf.org.cn/resources/1190201776262/2011/01/18/201101-9.pdf}{现代服务业的现状、趋势与战略思考(free)}

\href{http://history.ccf.org.cn/resources/1190201776262/2011/01/18/201101-2.pdf}{网络计算时代的软件技术思考与实践(free)}

\href{http://history.ccf.org.cn/resources/1190201776262/2011/01/18/201101-3.pdf}{ACM在中国(free)}

\href{http://history.ccf.org.cn/resources/1190201776262/2011/01/18/201101-4.pdf}{云计算的挑战与机遇(free)}

\href{http://history.ccf.org.cn/resources/1190201776262/2011/01/18/201101-5.pdf}{网络创造美好生活(free)}

\href{http://history.ccf.org.cn/resources/1190201776262/2011/01/18/201101-6.pdf}{服务计算:服务转型与打造现代服务业的利器(free)}

\href{http://history.ccf.org.cn/resources/1190201776262/2011/01/18/201101-7.pdf}{数据存储管理技术的更新换代(free)}

\href{http://history.ccf.org.cn/resources/1190201776262/2011/01/18/201101-8.pdf}{计算机系统的技术挑战(free)}

\subsection{专栏}
\href{http://history.ccf.org.cn/resources/1190201776262/2011/01/18/201101-10.pdf}{关于物联网工程专业课程体系的思考(free)}

\href{http://history.ccf.org.cn/resources/1190201776262/2011/01/18/201101-11.pdf}{物联网对软件技术的挑战及其对策(free)}

\href{http://history.ccf.org.cn/resources/1190201776262/2011/01/18/201101-12.pdf}{社会网络中的隐私保护(free)}

\href{http://history.ccf.org.cn/resources/1190201776262/2011/01/18/201101-13.pdf}{网络计算时代的工作流技术——过程服务(free)}

\href{http://history.ccf.org.cn/resources/1190201776262/2011/01/18/201101-14.pdf}{关于开源软件进一步发展的动因(free)}

\subsection{动态}
\href{http://history.ccf.org.cn/resources/1190201776262/2011/01/18/201101-15.pdf}{基金委拟大力支持科学仪器研究(free)}

\href{http://history.ccf.org.cn/resources/1190201776262/2011/01/18/201101-16.pdf}{计算分子生物学国际会议(free)}

\subsection{译文}
\href{http://history.ccf.org.cn/resources/1190201776262/2011/01/18/201101-17.pdf}{企业开源软件资产管理(free)}

\subsection{学会论坛}
\href{http://history.ccf.org.cn/resources/1190201776262/2011/01/18/201101-19.pdf}{“CCF走进高校”将是会员与高校互动的平台(free)}

\subsection{特别报道}
\href{http://history.ccf.org.cn/resources/1190201776262/2011/01/18/201101-18.pdf}{IT人的精神家园(free)}


\section{\href{http://history.ccf.org.cn/sites/ccf/jsjtbbd.jsp?contentId=2585984258060}{\textbf{2010年第12期(总第58期)}}}
本期的专题是《协同计算》。协同计算是多学科交叉和支持的研究领域,它将不同学科中共同存在的协同现象抽取出来,作为研究对象,以便更有效地促进社会群体间有目的的交互和协作。经历了15年的发展,协同计算的研究已取得了相当进展,但是在网络计算、情景感知等新环境下,协同计算依然存在诸多挑战,本期专题特别邀请了几位专家从多个角度撰文介绍协同计算的现状和未来。
\subsection{专题}
\href{http://history.ccf.org.cn/resources/1190201776262/2010/12/23/1.pdf}{协同计算}

\href{http://history.ccf.org.cn/resources/1190201776262/2010/12/23/2.pdf}{情感协同计算}

\href{http://history.ccf.org.cn/resources/1190201776262/2010/12/23/6.pdf}{流程工厂的协同设计与管理}

\href{http://history.ccf.org.cn/resources/1190201776262/2010/12/23/3.pdf}{多种协同方式集成的一致性维护}

\href{http://history.ccf.org.cn/resources/1190201776262/2010/12/23/5.pdf}{支持创新设计的协同进化设计系统}

\href{http://history.ccf.org.cn/resources/1190201776262/2010/12/23/4.pdf}{万维网信息服务的协同软件}

\subsection{专栏}
\href{http://history.ccf.org.cn/resources/1190201776262/2010/12/23/7.pdf}{百花齐放尚可 百家争鸣不足——专栏编委谢幕作}

\href{http://history.ccf.org.cn/resources/1190201776262/2010/12/23/8.pdf}{为国人同行的成就喝彩——兼谈对CCF王选奖评选机制的理解}

\href{http://history.ccf.org.cn/resources/1190201776262/2010/12/23/9.pdf}{如何评价学术会议的质量}

\href{http://history.ccf.org.cn/resources/1190201776262/2010/12/23/13.pdf}{转变}

\href{http://history.ccf.org.cn/resources/1190201776262/2010/12/23/10.pdf}{忘年之交 亦师亦友——纪念麦中凡教授}

\href{http://history.ccf.org.cn/resources/1190201776262/2010/12/23/12.pdf}{结合产学研 可以以利为利}

\href{http://history.ccf.org.cn/resources/1190201776262/2010/12/23/11.pdf}{研究生培养模式的探讨}

\subsection{动态}
\href{http://history.ccf.org.cn/resources/1190201776262/2010/12/23/15.pdf}{中国计算机学会推荐国际学术会议(三)}

\href{http://history.ccf.org.cn/resources/1190201776262/2010/12/23/14.pdf}{IEEE ICDCS 2010会议介绍}

\subsection{译文}
\href{http://history.ccf.org.cn/resources/1190201776262/2010/12/23/16.pdf}{智能服务机}

\subsection{学会论坛}
\href{http://history.ccf.org.cn/resources/1190201776262/2010/12/23/19.pdf}{构建学会基础架构 提供多层次服务}

\subsection{特别报道}
\href{http://history.ccf.org.cn/resources/1190201776262/2010/12/23/17.pdf}{中国IT创新发展的助跑者——读《中国计算机学会通讯》有感}

\href{http://history.ccf.org.cn/resources/1190201776262/2010/12/23/18.pdf}{CCF YOCSEF联动特别论坛“我的桌面谁做主?”}


\section{\href{http://history.ccf.org.cn/sites/ccf/jsjtbbd.jsp?contentId=2578397185803}{\textbf{2010年第11期(总第57期)}}}
本期的专题是《普适计算技术》。学术界已经对普适计算技术进行了多年的研究,即将到来的云计算和物联网时代又给这个领域带来了新的挑战和机遇。基于对大量用户和设备的物理世界情境知识的积累和挖掘,可以研发更为创新的互联网服务。本期专题邀请了几位专家对普适计算技术提出自己的观点及展望,同广大读者共同思考普适计算的未来。
\subsection{专题}
\href{http://history.ccf.org.cn/resources/1190201776262/2010/11/17/9.pdf}{挖掘物理世界中的智能  谢幸}

\href{http://history.ccf.org.cn/resources/1190201776262/2010/11/17/10.pdf}{流式软件结构  张尧学 周悦芝 陈渝 史元春}

\href{http://history.ccf.org.cn/resources/1190201776262/2010/11/17/6.pdf}{普适计算技术:创建人机物和谐环境  特邀编辑 史元春 谢幸(免费)}

\href{http://history.ccf.org.cn/resources/1190201776262/2010/11/17/7.pdf}{普适计算如何融入日常生活  游创文 沈而立 张诗平 陈予涵 朱浩华}

\href{http://history.ccf.org.cn/resources/1190201776262/2010/11/17/8.pdf}{基于传感器数据的行为识别  杨强 郑文琛 胡昊}

\subsection{专栏}
\href{http://history.ccf.org.cn/resources/1190201776262/2010/11/17/12.pdf}{软件服务及应用是当前云计算发展的关键  韩燕波 王千祥(免费)}

\href{http://history.ccf.org.cn/resources/1190201776262/2010/11/17/13.pdf}{物联网之路在何方  卜佳俊 何道敬}

\href{http://history.ccf.org.cn/resources/1190201776262/2010/11/17/14.pdf}{新一代搜索引擎的未来  赖荣凤 孙立峰}

\href{http://history.ccf.org.cn/resources/1190201776262/2010/11/17/11.pdf}{CCF CNCC2010分论坛专题(免费)}

\href{http://history.ccf.org.cn/resources/1190201776262/2010/11/17/15.pdf}{规划下一代数据中心  郑纬民}

\href{http://history.ccf.org.cn/resources/1190201776262/2010/11/17/16.pdf}{百花齐放,百家争鸣  蒋涛}

\href{http://history.ccf.org.cn/resources/1190201776262/2010/11/17/17.pdf}{计算机系统发展的必然趋势  薛一波 王海霞}

\href{http://history.ccf.org.cn/resources/1190201776262/2010/11/17/18.pdf}{新计算、新网络的危.机  潘柱廷}

\subsection{动态}
\href{http://history.ccf.org.cn/resources/1190201776262/2010/11/30/23.pdf}{中国计算机学会推荐国际学术会议(二)  中国计算机学会(免费)}

\href{http://history.ccf.org.cn/resources/1190201776262/2010/11/17/19.pdf}{ACM普适计算国际会议  谢幸}

\subsection{译文}
\href{http://history.ccf.org.cn/resources/1190201776262/2010/11/17/21.pdf}{创感时代的计算机科学  迈克尔.吉达(译者:姚登峰 李超)}

\href{http://history.ccf.org.cn/resources/1190201776262/2010/11/17/20.pdf}{全球化IT管理规模、响应与创新的构建  SIEW KIEN SIA 克里斯蒂娜 彼德.威尔(译者 华丽水)}

\subsection{学会论坛}
\href{http://history.ccf.org.cn/resources/1190201776262/2010/11/17/22.pdf}{从V1.0到V2.0:对YOCSEF未来发展的设想  周宏桥(免费)}

\subsection{特别报道}
\href{http://history.ccf.org.cn/resources/1190201776262/2010/11/17/3.pdf}{贺信}

\href{http://history.ccf.org.cn/resources/1190201776262/2010/11/17/4.pdf}{大会报告摘要(免费)}

\href{http://history.ccf.org.cn/resources/1190201776262/2010/11/17/5.pdf}{大会十大亮点(免费)}

\href{http://history.ccf.org.cn/resources/1190201776262/2010/11/15/3.pdf}{2010中国计算机大会开幕词(免费)}

\href{http://history.ccf.org.cn/resources/1190201776262/2010/11/15/1.pdf}{2010中国计算机大会成功举行(免费)}


\section{\href{http://history.ccf.org.cn/sites/ccf/jsjtbbd.jsp?contentId=2573051627543}{\textbf{2010年第10期(总第56期)}}}
艾级超级计算 2008年,随着美国的走鹃和美洲豹超级计算机的峰值和Linpack 性能双双突破千万亿次大关,人类终于突破这一看似不可逾越的速度障碍,世界各国的目光都转向超级计算机速度的下一个皇冠——Exascale(艾级,百亿亿次)超级计算机。针对艾级计算能耗、存储、并发和局部性、可靠性等问题,本刊邀请了国内主流超级计算机硬件和软件研制单位的专家从多方位探讨当前业界面临的艾级超级计算机挑战及应对策略。
\subsection{专题}
\href{http://history.ccf.org.cn/resources/1190201776262/2010/10/20/zhuanlan-4.pdf}{艾级计算系统若干挑战问题的思考}

\href{http://history.ccf.org.cn/resources/1190201776262/2010/10/20/zhuanti-6.pdf}{艾级专用高性能计算机}

\href{http://history.ccf.org.cn/resources/1190201776262/2010/10/20/zhuanti-3.pdf}{我国如何应对国际艾级计算的竞争}

\href{http://history.ccf.org.cn/resources/1190201776262/2010/10/20/zhuanti-5.pdf}{艾级高性能计算机系统高可用技术}

\href{http://history.ccf.org.cn/resources/1190201776262/2010/10/20/zhuanti-1.pdf}{艾级超级计算的机遇与挑战 (免费)}

\href{http://history.ccf.org.cn/resources/1190201776262/2010/10/20/zhuanti-2.pdf}{高性能计算软件的开发和研制}

\subsection{专栏}
\href{http://history.ccf.org.cn/resources/1190201776262/2010/10/20/zhuanlan-6.pdf}{从图灵奖得主看信息时代研究生创新教育}

\href{http://history.ccf.org.cn/resources/1190201776262/2010/10/20/zhuanlan-1.pdf}{韵河:聆听经典 传承文化 (免费)}

\href{http://history.ccf.org.cn/resources/1190201776262/2010/10/20/zhuanlan-2.pdf}{网络计算仍未超越图灵计算}

\href{http://history.ccf.org.cn/resources/1190201776262/2010/10/20/zhuanlan-5.pdf}{做人与成佛}

\href{http://history.ccf.org.cn/resources/1190201776262/2010/10/20/zhuanlan-3.pdf}{基于显式数据模板的高效能可重构异构多核处理器}

\href{http://history.ccf.org.cn/resources/1190201776262/2010/10/20/zhuanlan-4.pdf}{相变存储器}

\subsection{动态}
\href{http://history.ccf.org.cn/resources/1190201776262/2010/10/20/dongtai-1.pdf}{我国第五届计算机技术领域工程硕士培养工作会议}

\href{http://history.ccf.org.cn/resources/1190201776262/2010/10/20/dongtai-2.pdf}{中国计算机学会推荐国际学术会议(一)}

\subsection{译文}
\href{http://history.ccf.org.cn/resources/1190201776262/2010/10/20/yiwen-2.pdf}{云计算和电力:超越公用服务模型}

\href{http://history.ccf.org.cn/resources/1190201776262/2010/10/20/yiwen-1.pdf}{节能软件(免费)}

\subsection{学会论坛}
\href{http://history.ccf.org.cn/resources/1190201776262/2010/10/20/xuehuiluntan-2.pdf}{CCF是青年人成长的摇篮(免费)}

\href{http://history.ccf.org.cn/resources/1190201776262/2010/10/20/zhengwen-1.pdf}{计算机科研工作者的良师益友——《中国计算机学会通讯》(免费)}

\href{http://history.ccf.org.cn/resources/1190201776262/2010/10/20/xuehuiluntan-1.pdf}{新观念 新思维 新服务 (免费)}


\section{\href{http://history.ccf.org.cn/sites/ccf/jsjtbbd.jsp?contentId=2568871040459}{\textbf{2010年第09期(总第55期)}}}
服务计算 物联网、社会信息网络和云基础设施的发展使软件的构成、系统边界、运营方式、管控原理和使用模式产生质的变化,也推动了服务计算的第二波发展。中国计算机学会于2010年初适时成立了服务计算专业委员会,旨在促进我国服务计算领域的科学技术进步与产业发展。本期专题组织了来自服务计算专业委员会的主题报告,为读者介绍服务计算的进展情况。
\subsection{专题}
\href{http://history.ccf.org.cn/resources/1190201776262/2010/09/14/201009-1.pdf}{互联网环境下的服务计算(free)}

\href{http://history.ccf.org.cn/resources/1190201776262/2010/09/15/201009-2.pdf}{网络信息服务的演进}

\href{http://history.ccf.org.cn/resources/1190201776262/2010/09/15/201009-3.pdf}{从软件工程的发展看面向价值的服务工程}

\href{http://history.ccf.org.cn/resources/1190201776262/2010/09/15/201009-4.pdf}{面向按需服务的软件方法及其标准化研究进展}

\href{http://history.ccf.org.cn/resources/1190201776262/2010/09/15/201009-5 .pdf}{服务网络}

\href{http://history.ccf.org.cn/resources/1190201776262/2010/09/15/201009-6.pdf}{大规模复杂服务计算系统性能分析}

\href{http://history.ccf.org.cn/resources/1190201776262/2010/09/15/201009-7.pdf}{高可伸缩、动态重构——服务计算对软件系统测试的两个挑战}

\subsection{专栏}
\href{http://history.ccf.org.cn/resources/1190201776262/2010/09/15/201009-8-zhuanlan-free.pdf}{物联网数据特性对建模和挖掘的挑战(free)}

\href{http://history.ccf.org.cn/resources/1190201776262/2010/09/15/201009-9.pdf}{社会感知计算}

\href{http://history.ccf.org.cn/resources/1190201776262/2010/09/15/201009-10.pdf}{蚁群优化算法的理论}

\href{http://history.ccf.org.cn/resources/1190201776262/2010/09/15/201009-11-free.pdf}{自适应可循环计算(free)}

\href{http://history.ccf.org.cn/resources/1190201776262/2010/09/15/201009-12.pdf}{面向服务架构的虚拟实验教学}

\subsection{动态}
\href{http://history.ccf.org.cn/resources/1190201776262/2010/09/15/201009-13.pdf}{IEEE国际并行与分布式处理会议}

\subsection{译文}
\href{http://history.ccf.org.cn/resources/1190201776262/2010/09/15/201009-14.pdf}{可用性评估中人员的数量:10±2规则}

\href{http://history.ccf.org.cn/resources/1190201776262/2010/09/15/201009-15-free.pdf}{云计算和SaaS——计算的新平台(free)}

\subsection{学会论坛}
\href{http://history.ccf.org.cn/resources/1190201776262/2010/09/15/201009-18.pdf}{CCF将与ACM开展实质性合作}

\subsection{特别报道}
\href{http://history.ccf.org.cn/resources/1190201776262/2010/09/15/201009-16-free.pdf}{十年坚持  共铸未来——2010年CCF吕梁教育扶贫考察报告(free)}

\href{http://history.ccf.org.cn/resources/1190201776262/2010/09/15/201009-17.pdf}{中国计算机学会YOCSEF专题论坛 “英雄莫问出处真假?”}


\section{\href{http://history.ccf.org.cn/sites/ccf/jsjtbbd.jsp?contentId=2562834412265}{\textbf{2010年第08期(总第54期)}}}
语义万维网与万维网科学 万维网源于人们对网络的小小需求,却发展成为“人类历史上最大的信息系统”。当看到它的潜力后,人们进而希望它变成能理解语义的具有逻辑智慧的网络——语义万维网。它要解决如何表达、存储、查询结构化信息等诸多难题,使网络具有类似人的判断力。本期专题特邀了国内外从事此项研究的学者就以上问题,给读者做一介绍。
\subsection{专题}
\href{http://history.ccf.org.cn/resources/1190201776262/2010/08/11/201008-1.pdf}{语义万维网与万维网科学(free)}

\href{http://history.ccf.org.cn/resources/1190201776262/2010/08/12/201008-2.pdf}{数据万维网的知识管理}

\href{http://history.ccf.org.cn/resources/1190201776262/2010/08/12/201008-3.pdf}{语义万维网搜索}

\href{http://history.ccf.org.cn/resources/1190201776262/2010/08/12/201008-4.pdf}{语义万维网中的大规模推理}

\href{http://history.ccf.org.cn/resources/1190201776262/2010/08/12/201008-5.pdf}{语义社会网络}

\href{http://history.ccf.org.cn/resources/1190201776262/2010/08/12/201008-6.pdf}{语义万维网在医疗和生命科学领域的应用}

\subsection{专栏}
\href{http://history.ccf.org.cn/resources/1190201776262/2010/08/12/201008-7.pdf}{计算机学者发表论文的目的与追求(free)}

\href{http://history.ccf.org.cn/resources/1190201776262/2010/08/12/201008-8.pdf}{基于ACP平行架构的关键字竞价计算}

\href{http://history.ccf.org.cn/resources/1190201776262/2010/08/12/201008-9.pdf}{社区问答架构}

\href{http://history.ccf.org.cn/resources/1190201776262/2010/08/12/201008-10.pdf}{语义数据与终端用户的交互平台}

\subsection{动态}
\href{http://history.ccf.org.cn/resources/1190201776262/2010/08/12/201008-11.pdf}{《信息科学发展战略研究》中的若干热点(free)}

\subsection{译文}
\href{http://history.ccf.org.cn/resources/1190201776262/2010/08/12/201008-12.pdf}{万维网科学:理解万维网的跨学科途径}

\subsection{学会论坛}
\href{http://history.ccf.org.cn/resources/1190201776262/2010/08/12/201008-14.pdf}{专委发展的转折点(free)}

\href{http://history.ccf.org.cn/resources/1190201776262/2010/08/12/201008-15.pdf}{CCF专业委员会组织层次及职能}

\href{http://history.ccf.org.cn/resources/1190201776262/2010/08/12/201008-16.pdf}{CCF奖励委员会成立}

\href{http://history.ccf.org.cn/resources/1190201776262/2010/08/12/201008-17.pdf}{将专委打造成为专业人员的交流平台}

\subsection{特别报道}
\href{http://history.ccf.org.cn/resources/1190201776262/2010/08/12/201008-13.pdf}{“三网融合,我们准备好了吗?”}


\section{\href{http://history.ccf.org.cn/sites/ccf/jsjtbbd.jsp?contentId=2558164128410}{\textbf{2010年第07期(总第53期)}}}
虚拟现实 数字世界是如何走入我们的生活的?虚拟现实技术在其中起了至关重要的作用。从《阿凡达》的幻境到奥运会安保演练,无不得益于虚拟现实技术的发展。本期虚拟现实专题邀请了国内虚拟现实领域的几位专家撰写文章,用较为通俗的语言介绍虚拟现实技术的发展,揭开她的神秘面纱。
\subsection{专题}
\href{http://history.ccf.org.cn/resources/1190201776262/2010/07/15/201007-1.pdf}{揭开虚拟现实的面纱(free)}

\href{http://history.ccf.org.cn/resources/1190201776262/2010/07/15/201007-2.pdf}{虚拟现实中的10个科学技术问题}

\href{http://history.ccf.org.cn/resources/1190201776262/2010/07/15/201007-3.pdf}{虚拟现实技术的现状与发展}

\href{http://history.ccf.org.cn/resources/1190201776262/2010/07/15/201007-4.pdf}{分布式虚拟现实中超大规模数据管理}

\href{http://history.ccf.org.cn/resources/1190201776262/2010/08/17/2010.7.23-28.pdf}{头盔显示技术的发展趋势}

\href{http://history.ccf.org.cn/resources/1190201776262/2010/07/15/201007-6.pdf}{人体行为模拟的应用与展望}

\href{http://history.ccf.org.cn/resources/1190201776262/2010/07/15/201007-7.pdf}{虚拟现实项目(2008~2009年)自然科学基金资助情况}

\subsection{专栏}
\href{http://history.ccf.org.cn/resources/1190201776262/2010/07/15/2010.7 xiao-36.pdf}{第三届中美计算机科学高峰论坛}

\href{http://history.ccf.org.cn/resources/1190201776262/2010/07/15/201007-8.pdf}{老兵新谈:中国教育和科技发展之孔见}

\href{http://history.ccf.org.cn/resources/1190201776262/2010/07/15/201007-9.pdf}{中美大型科研项目巡礼(free)}

\href{http://history.ccf.org.cn/resources/1190201776262/2010/07/15/201007-10.pdf}{更透彻的感知、更全面的互联、更精确的操控}

\href{http://history.ccf.org.cn/resources/1190201776262/2010/07/15/201007-11.pdf}{现代计算科学与技术为人类健康造福}

\href{http://history.ccf.org.cn/resources/1190201776262/2010/07/15/201007-12.pdf}{交叉学科研究与教育的挑战}

\href{http://history.ccf.org.cn/resources/1190201776262/2010/07/15/201007-13.pdf}{海计算:物联网的新型计算模型}

\href{http://history.ccf.org.cn/resources/1190201776262/2010/07/15/201007-14.pdf}{对加州大学抵制英国自然出版集团的解读和思考}

\href{http://history.ccf.org.cn/resources/1190201776262/2010/07/15/201007-15.pdf}{面向超大规模系统的软件工程}

\subsection{动态}
\href{http://history.ccf.org.cn/resources/1190201776262/2010/07/15/201007-16.pdf}{ACL会议概述}

\subsection{译文}
\href{http://history.ccf.org.cn/resources/1190201776262/2010/07/15/201007-17.pdf}{计算实践和教育的未来}

\href{http://history.ccf.org.cn/resources/1190201776262/2010/07/15/2010.7 xiao-18.pdf}{类型论蓬勃发展}

\subsection{学会论坛}
\href{http://history.ccf.org.cn/resources/1190201776262/2010/07/15/201007-20.pdf}{专业委员会要讲服务促改革(free)}

\href{http://history.ccf.org.cn/resources/1190201776262/2010/07/15/201007-21.pdf}{也谈对专委发展的看法(free)}

\href{http://history.ccf.org.cn/resources/1190201776262/2010/07/15/201007-19.pdf}{迁兰变鲍——《中国计算机学会通讯》}


\section{\href{http://history.ccf.org.cn/sites/ccf/jsjtbbd.jsp?contentId=2552628433519}{\textbf{2010年第06期(总第52期)}}}
基于位置的服务 如今信息技术拉近了人与人之间的距离,人们从未像现在这样深刻体会到时间和空间对于自己的重要性,以及它们给自己带来的好处和烦恼。大家对移动、智能、定位和服务这些专业概念充满了好奇和期待,未来的生活会是什么样子?本期专题介绍了如何利用信息技术展开基于位置的服务,也许会触发您对此方面的一些灵感。
\subsection{专题}
\href{http://history.ccf.org.cn/resources/1190201776262/2010/06/12/201006-1.pdf}{基于位置的服务(free)}

\href{http://history.ccf.org.cn/resources/1190201776262/2010/06/12/201006-2.pdf}{基于位置服务的分析与展望}

\href{http://history.ccf.org.cn/resources/1190201776262/2010/06/12/201006-3.pdf}{基于位置服务的隐私保护}

\href{http://history.ccf.org.cn/resources/1190201776262/2010/06/12/201006-4.pdf}{基于用户轨迹挖掘的智能位置服务}

\href{http://history.ccf.org.cn/resources/1190201776262/2010/06/12/201006-5.pdf}{基于位置服务与人类活动的关系和影响}

\href{http://history.ccf.org.cn/resources/1190201776262/2010/06/12/201006-6.pdf}{LBS的数据处理技术}

\subsection{专栏}
\href{http://history.ccf.org.cn/resources/1190201776262/2010/06/12/201006-7.pdf}{浅谈国内大学的使命和追求(free)}

\href{http://history.ccf.org.cn/resources/1190201776262/2010/06/12/201006-8.pdf}{乌云渐散  云朵呈现——2009~2010年世界软件产业回顾与展望}

\href{http://history.ccf.org.cn/resources/1190201776262/2010/06/12/201006-9.pdf}{从PC行业的变迁解读“创新地图”(free)}

\href{http://history.ccf.org.cn/resources/1190201776262/2010/06/12/201006-10.pdf}{面向老年人生活的智能辅助}

\subsection{动态}
\href{http://history.ccf.org.cn/resources/1190201776262/2010/06/12/201006-11.pdf}{ACM信息与知识管理国际会议}

\subsection{译文}
\href{http://history.ccf.org.cn/resources/1190201776262/2010/06/12/201006-12.pdf}{软件体系结构评审实践}

\subsection{学会论坛}
\href{http://history.ccf.org.cn/resources/1190201776262/2010/06/12/201006-14.pdf}{关于专委发展的一些思考(free)}

\subsection{特别报道}
\href{http://history.ccf.org.cn/resources/1190201776262/2010/06/12/201006-13.pdf}{中国IT人的钮带——《中国计算机学会通讯》(free)}


\section{\href{http://history.ccf.org.cn/sites/ccf/jsjtbbd.jsp?contentId=2547160900037}{\textbf{2010年第05期(总第51期)}}}
计算机科研评价的评估 如何发挥科研成果的影响力?美国佛蒙特大学王晓阳教授认为应该建立一个评估计算机科研工作的检验框架。此框架是把新想法、新观念合理地放入“科研社会网”的有效方法。作为学术社团,学会也一直在探索为科研成果交流和评价提供服务的渠道,欢迎读者来稿参与讨论。
\subsection{专题}
\href{http://history.ccf.org.cn/resources/1190201776262/2010/05/12/201005-1zhuanti-free.pdf}{情感计算(free)}

\href{http://history.ccf.org.cn/resources/1190201776262/2010/05/12/201005-2.pdf}{基于PAD三维空间的情感测量}

\href{http://history.ccf.org.cn/resources/1190201776262/2010/05/21/201005-3.pdf}{自然环境下的人脸表情识别}

\href{http://history.ccf.org.cn/resources/1190201776262/2010/05/12/201005-4.pdf}{局部异常行为检测与特殊情感识别}

\href{http://history.ccf.org.cn/resources/1190201776262/2010/05/21/201005-5.pdf}{面向自然语音的情感认知}

\href{http://history.ccf.org.cn/resources/1190201776262/2010/05/21/201005-6.pdf}{基于情感维度的人脸表情生成}

\href{http://history.ccf.org.cn/resources/1190201776262/2010/05/12/201005-7.pdf}{情感语音的合成}

\subsection{专栏}
\href{http://history.ccf.org.cn/resources/1190201776262/2010/05/12/201005-8zhuanlan-free.pdf}{计算机科研评价的评估(free)}

\href{http://history.ccf.org.cn/resources/1190201776262/2010/05/12/201005-9.pdf}{云计算环境的安全威胁和保护}

\href{http://history.ccf.org.cn/resources/1190201776262/2010/05/12/201005-10.pdf}{万维网信息可信性问题}

\href{http://history.ccf.org.cn/resources/1190201776262/2010/05/13/201005-11.pdf}{对等网络流量优化}

\href{http://history.ccf.org.cn/resources/1190201776262/2010/05/13/201005-12.pdf}{异构环境下的移动对象数据管理}

\subsection{动态}
\href{http://history.ccf.org.cn/resources/1190201776262/2010/05/13/201005-13dongtai.pdf}{国际微处理结构大会}

\href{http://history.ccf.org.cn/resources/1190201776262/2010/05/13/201005-14-free.pdf}{国际普适计算会议IEEE PerCom 2010(free)}

\subsection{译文}
\href{http://history.ccf.org.cn/resources/1190201776262/2010/05/13/201005-15yiwen.pdf}{平台思维的演变}

\href{http://history.ccf.org.cn/resources/1190201776262/2010/05/13/201005-16.pdf}{无处不在的计算思维}

\subsection{学会论坛}
\href{http://history.ccf.org.cn/resources/1190201776262/2010/05/25/2010.5-92-93.pdf}{学会应尽快建立完善的会士制度(free)}

\subsection{特别报道}
\href{http://history.ccf.org.cn/resources/1190201776262/2010/05/13/201005-18-free.pdf}{回忆《中国计算机学会通讯》的创办(free)}

\href{http://history.ccf.org.cn/resources/1190201776262/2010/05/13/201005-17tebiebaodao-free.pdf}{读者调查报告(2009.7~12)(free)}


\section{\href{http://history.ccf.org.cn/sites/ccf/jsjtbbd.jsp?contentId=2543828071575}{\textbf{2010年第04期(总第50期)}}}
本期专题:物联网技术初探 “物联网”是近年来人们
关注的热点,但学术界是否已经做好物联网相关技术的准备? 本期专题邀请了研究物
联网相关领域的几位专家撰文,从物联网与CPS(Cyber-Physical System)的关系、物联
网系统及设备、多媒体信息感知、数据管理以及应用等不同视角阐述对物联网的认识。
\subsection{专题}
\href{http://history.ccf.org.cn/resources/1190201776262/2010/04/23/201004-1free.pdf}{物联网技术初探(free)}

\href{http://history.ccf.org.cn/resources/1190201776262/2010/04/23/201004-2.pdf}{信息产业新革命之争:物联网还是CPS?}

\href{http://history.ccf.org.cn/resources/1190201776262/2010/04/23/201004-3.pdf}{物联网系统及核心设备}

\href{http://history.ccf.org.cn/resources/1190201776262/2010/04/23/201004-4.pdf}{无线多媒体传感器网络}

\href{http://history.ccf.org.cn/resources/1190201776262/2010/04/23/201004-5.pdf}{物联网中的数据管理}

\href{http://history.ccf.org.cn/resources/1190201776262/2010/04/23/201004-6free.pdf}{绿野千传:突破自组织传感网大规模应用壁垒(free)}

\href{http://history.ccf.org.cn/resources/1190201776262/2010/04/23/201004-7.pdf}{实时信息系统——物联网的基础}

\subsection{专栏}
\href{http://history.ccf.org.cn/resources/1190201776262/2010/04/23/201004-8.pdf}{提高青年教师实践教学能力的一条有效途径——记计算机教指委与亚思晟公司合作的一次尝试(free)}

\href{http://history.ccf.org.cn/resources/1190201776262/2010/04/23/201004-9.pdf}{软件行为学在游戏角色行为控制中的作用初探}

\href{http://history.ccf.org.cn/resources/1190201776262/2010/04/23/201004-10.pdf}{自主网络特征与模型}

\href{http://history.ccf.org.cn/resources/1190201776262/2010/04/23/201004-11.pdf}{动态计算机的思想方法}

\subsection{动态}
\href{http://history.ccf.org.cn/resources/1190201776262/2010/04/23/201004-12.pdf}{国际百亿亿级计算软件项目}

\subsection{译文}
\href{http://history.ccf.org.cn/resources/1190201776262/2010/04/23/201004-13.pdf}{准备好迎接Web操作系统了吗?}

\href{http://history.ccf.org.cn/resources/1190201776262/2010/04/23/201004-14.pdf}{面向研究的智能信息基础设施——论语义计算及其在科学研究中的作用}

\subsection{学会论坛}
\href{http://history.ccf.org.cn/resources/1190201776262/2010/04/23/201004-16.pdf}{中国计算机学会应设置更多奖项(free)}

\subsection{特别报道}
\href{http://history.ccf.org.cn/resources/1190201776262/2010/04/23/201004-15.pdf}{《中国计算机学会通讯》创刊50期纪念活动}


\section{\href{http://history.ccf.org.cn/sites/ccf/jsjtbbd.jsp?contentId=2543828071403}{\textbf{2010年第03期(总第49期)}}}
本期专题:绿色计算 如今,计算已经深入到人们的生活中,就像依赖水和电一样,须臾不能离开。但是计算也给人们带来重要的问题,那就是计算能耗。这个问题如果不重视、不解决,必将威胁到人类生存的环境。本期组织的绿色计算专题,从软硬件方面综合阐述了计算节能减排的技术发展状况,希望引起广大读者对计算中环保问题的关注。
\subsection{专题}
\href{http://history.ccf.org.cn/resources/1190201776262/2010/04/23/20103-2.pdf}{绿色计算——中国计算机科学发展的新机遇(free)}

\href{http://history.ccf.org.cn/resources/1190201776262/2010/04/23/20103-3.pdf}{绿色计算的概念及内容分析}

\href{http://history.ccf.org.cn/resources/1190201776262/2010/04/23/20103-4.pdf}{虚拟化——“绿化”数据中心的有效途径}

\href{http://history.ccf.org.cn/resources/1190201776262/2010/04/23/20103-5.pdf}{“云”中的绿色计算技术}

\href{http://history.ccf.org.cn/resources/1190201776262/2010/04/23/20103-6.pdf}{绿色计算中的虚拟机节能}

\href{http://history.ccf.org.cn/resources/1190201776262/2010/04/23/20103-7.pdf}{绿色软件技术研究进展}

\href{http://history.ccf.org.cn/resources/1190201776262/2010/04/23/20103-8.pdf}{绿色计算软件前沿问题}

\subsection{专栏}
\href{http://history.ccf.org.cn/resources/1190201776262/2010/04/23/20103-9.pdf}{UCLA领域专用计算中心简介(free)}

\href{http://history.ccf.org.cn/resources/1190201776262/2010/04/23/20103-10.pdf}{计算机科学的基础、结构与问题}

\href{http://history.ccf.org.cn/resources/1190201776262/2010/04/23/20103-11.pdf}{社交网络}

\href{http://history.ccf.org.cn/resources/1190201776262/2010/04/23/20103-12.pdf}{云备份的研究与发展}

\href{http://history.ccf.org.cn/resources/1190201776262/2010/04/23/20103-13.pdf}{中国邮政信息化建设的发展和思考}

\href{http://history.ccf.org.cn/resources/1190201776262/2010/04/23/20103-14.pdf}{下一代互联网体系结构}

\subsection{动态}
\href{http://history.ccf.org.cn/resources/1190201776262/2010/04/23/20103-15.pdf}{IEEE GLOBECOM 2009会议}

\href{http://history.ccf.org.cn/resources/1190201776262/2010/04/23/20103-16.pdf}{密码学三大国际会议}

\subsection{译文}
\href{http://history.ccf.org.cn/resources/1190201776262/2010/04/23/20103-17.pdf}{技术视角——一个编译器的故事}

\href{http://history.ccf.org.cn/resources/1190201776262/2010/04/23/20103-18.pdf}{真实编译器的形式化验证}

\subsection{学会论坛}
\href{http://history.ccf.org.cn/resources/1190201776262/2010/04/23/20103-19.pdf}{完善内部治理结构创建新型科技社团(free)}

\subsection{特别报道}
\href{http://history.ccf.org.cn/resources/1190201776262/2010/04/23/20103-1.pdf}{2009年度CCF优秀博士学位论文奖获奖论文摘要(free)}


\section{\href{http://history.ccf.org.cn/sites/ccf/jsjtbbd.jsp?contentId=2543828071281}{\textbf{2010年第02期(总第48期)}}}
可信计算(Trusted Computing) 随着计算成为人们生活中不可或缺的基础设施,对计算的安全性和可靠性也提出了更高的要求。可信计算(国内有学者建议将Trusted Computing译成信任计算,以区别于DependableComputing,希望广大学者展开讨论,名词委员会再做决定。)作为一种能开发出按照人们预期操作的软硬件技术,被赋予了新的意义,也面临着新的挑战。本期组织的可信计算专题,从其现状和未来进行了阐述,相信会对读者了解这个领域有所帮助.....
\subsection{专题}
\href{http://history.ccf.org.cn/resources/1190201776262/2010/04/23/201002-1.pdf}{可信计算(Trusted Computing)(free)}

\href{http://history.ccf.org.cn/resources/1190201776262/2010/04/23/2010.2-2.pdf}{我国可信计算研究与发展}

\href{http://history.ccf.org.cn/resources/1190201776262/2010/04/23/2010.2-3.pdf}{软件可信性:互联网带来的挑战}

\href{http://history.ccf.org.cn/resources/1190201776262/2010/05/31/2010.2-4.pdf}{网络时代的软件可信演化}

\href{http://history.ccf.org.cn/resources/1190201776262/2010/04/23/2010.2-5.pdf}{可信网络体系结构与关键技术}

\href{http://history.ccf.org.cn/resources/1190201776262/2010/05/31/2010.2-6.pdf}{TCG的可信计算技术}

\subsection{专栏}
\href{http://history.ccf.org.cn/resources/1190201776262/2010/04/23/2010.2-7.pdf}{物联网与云计算(free)}

\href{http://history.ccf.org.cn/resources/1190201776262/2010/04/23/2010.2-8.pdf}{从软件服务组合到软件服务自主协同}

\href{http://history.ccf.org.cn/resources/1190201776262/2010/04/23/2010.2-9.pdf}{“编译原理”课程的思考}

\subsection{动态}
\href{http://history.ccf.org.cn/resources/1190201776262/2010/04/23/2010.2-10.pdf}{第22届高阶定理证明国际会议}

\subsection{译文}
\href{http://history.ccf.org.cn/resources/1190201776262/2010/04/23/2010.2-11.pdf}{可扩展的同步队列(下)}

\subsection{学会论坛}
\href{http://history.ccf.org.cn/resources/1190201776262/2010/04/23/2010.2-12.pdf}{访问英国工程与技术学会报告(free)}

\href{http://history.ccf.org.cn/resources/1190201776262/2010/05/31/2010.2-13.pdf}{英国工程技术学会实行会员导师制}


\section{\href{http://history.ccf.org.cn/sites/ccf/jsjtbbd.jsp?contentId=2542567629084}{\textbf{2010年第01期(总第47期)}}}
CNCC2009特邀报告 2009年10月23 ~ 24日在天津举行的中国计算机大会
(CNCC2009)上,9位国内外知名学者做了大会特邀报告,就计算机科学技术在未来10~20
年的发展、Cyber -PhysicalSystem,以及人才培养战略等
问题进行了演讲、并对计算机
领域所面临的问题和挑战,提
出了非常值得关注的问题。根
据大会记录整理的7位讲者的发
言,相信会对您有所启发……
\subsection{特邀报告}
\href{http://history.ccf.org.cn/resources/1190201776262/2010/04/22/20101-1.pdf}{努力践行“重点跨越”的战略取向(free)}

\href{http://history.ccf.org.cn/resources/1190201776262/2010/04/15/20101-2.pdf}{新时期需要什么样的ICT工程科技人才?——对计算机工程教育改革的思考}

\href{http://history.ccf.org.cn/resources/1190201776262/2010/04/15/20101-3.pdf}{信息技术面向2010年代的发展}

\href{http://history.ccf.org.cn/resources/1190201776262/2010/04/15/20101-4.pdf}{Cyber-physical Systems}

\href{http://history.ccf.org.cn/resources/1190201776262/2010/04/16/20101-5.pdf}{并行计算的第二个春天——机遇与挑战}

\href{http://history.ccf.org.cn/resources/1190201776262/2010/04/22/20101-6.pdf}{一个海外学者对中国计算机科研的回顾与展望(free)}

\href{http://history.ccf.org.cn/resources/1190201776262/2010/04/16/20101-7.pdf}{我国计算机事业的新起点}

\subsection{专栏}
\href{http://history.ccf.org.cn/resources/1190201776262/2010/04/16/20101-8.pdf}{软件工程教育的发展}

\href{http://history.ccf.org.cn/resources/1190201776262/2010/04/16/20101-9.pdf}{基于关键字的XML信息搜索技术}

\href{http://history.ccf.org.cn/resources/1190201776262/2010/04/16/20101-10.pdf}{认知科学的发展}

\href{http://history.ccf.org.cn/resources/1190201776262/2010/04/17/20101-11.pdf}{P2P十年:何去何从?}

\subsection{动态}
\href{http://history.ccf.org.cn/resources/1190201776262/2010/04/21/20101-12.pdf}{投影时序逻辑及其应用}

\subsection{译文}
\href{http://history.ccf.org.cn/resources/1190201776262/2010/04/21/20101-13.pdf}{可扩展的同步队列(上)}

\subsection{学会论坛}
中国计算机学会工作报告(free)


\section{\href{http://history.ccf.org.cn/sites/ccf/jsjtbbd.jsp?contentId=2542567629055}{\textbf{2009年第12期(总第46期)}}}
互联网计算:网络化软件 软件应用于互联网环境,带来了计算模式、开发模式、演化模式和产品形态等方面的改变。网络化软件的新特性有哪些?对传统软件理论和方法提出了那些挑战?本期专题将从不同视角对网络化软件进行介绍,让读者有一个综合了解。
\subsection{专题}
\href{http://history.ccf.org.cn/resources/1190201776262/2010/05/18/200912-1.pdf}{互联网计算:网络化软件(free)}

\href{http://history.ccf.org.cn/resources/1190201776262/2010/04/15/046008.pdf}{超出图灵机的互联网计算(free)}

\href{http://history.ccf.org.cn/resources/1190201776262/2010/04/15/046017.pdf}{按需服务的网络化软件开发}

\href{http://history.ccf.org.cn/resources/1190201776262/2010/04/15/046027.pdf}{互联网对软件演化的挑战}

\href{http://history.ccf.org.cn/resources/1190201776262/2010/04/15/046044.pdf}{网络化软件的协同与互操作}

\href{http://history.ccf.org.cn/resources/1190201776262/2010/04/15/046052.pdf}{开源网络化软件及其开放标准}

\href{http://history.ccf.org.cn/resources/1190201776262/2010/04/15/046036.pdf}{网络化软件的质量观}

\subsection{专栏}
\href{http://history.ccf.org.cn/resources/1190201776262/2010/04/15/046060.pdf}{基于互联网的图像检索与融合(free)}

\href{http://history.ccf.org.cn/resources/1190201776262/2010/04/15/046066.pdf}{从普适计算、CPS到物联网:下一代互联网的视界}

\subsection{动态}
\href{http://history.ccf.org.cn/resources/1190201776262/2010/04/15/046070.pdf}{国际万维网大会}

\subsection{译文}
\href{http://history.ccf.org.cn/resources/1190201776262/2010/04/15/046072.pdf}{移动计算的未来是位置,位置,位置?}

\subsection{学会论坛}
\href{http://history.ccf.org.cn/resources/1190201776262/2010/04/15/046086.pdf}{ACM简介}


\section{\href{http://history.ccf.org.cn/sites/ccf/jsjtbbd.jsp?contentId=2542567629052}{\textbf{2009年第11期(总第45期)}}}
多核时代的机遇与挑战 进入多核时代,计算机系统的软硬件功能分配以及各自的作用和特性都发生了质的变化。研究新型的计算模型、系统软件、新型的通信技术、应用开发平台技术以及和谐的人机交互模式等关键技术具有重要意义。本期专题邀请了国内几位专家撰文,从多核关键技术、操作系统、语言与编译环境、图形处理器和可重构计算等视角阐述了多核处理器技术以及面临的挑战。
\subsection{专题}
\href{http://history.ccf.org.cn/resources/1190201776262/2010/04/15/045010.pdf}{多核时代的机遇与挑战(free)}

\href{http://history.ccf.org.cn/resources/1190201776262/2010/04/15/045012.pdf}{多核/众核处理器的关键技术}

\href{http://history.ccf.org.cn/resources/1190201776262/2010/04/15/045019.pdf}{面向多核环境的操作系统设计}

\href{http://history.ccf.org.cn/resources/1190201776262/2010/04/15/045028.pdf}{并行编程语言与编译技术}

\href{http://history.ccf.org.cn/resources/1190201776262/2010/04/15/045043.pdf}{通用计算中的GPU}

\href{http://history.ccf.org.cn/resources/1190201776262/2010/04/15/045034.pdf}{可重构计算相关问题探讨}

\subsection{专栏}
\href{http://history.ccf.org.cn/resources/1190201776262/2010/04/15/045050.pdf}{芭芭拉·莉斯科芙获ACM图灵奖}

\href{http://history.ccf.org.cn/resources/1190201776262/2010/04/15/045052.pdf}{最重要的是创新和人才(free)}

\href{http://history.ccf.org.cn/resources/1190201776262/2010/04/15/045055.pdf}{网格、云、SaaS、SOA、WebX.0……本质是互联网计算}

\href{http://history.ccf.org.cn/resources/1190201776262/2010/04/15/045059.pdf}{频谱竞价拍卖方式的新思路}

\href{http://history.ccf.org.cn/resources/1190201776262/2010/04/15/045063.pdf}{软件项目管理中如何挖掘用户潜在需求}

\subsection{动态}
\href{http://history.ccf.org.cn/resources/1190201776262/2010/04/15/045068.pdf}{ACM SIGGRAPH会议}

\subsection{译文}
\href{http://history.ccf.org.cn/resources/1190201776262/2010/04/15/045072.pdf}{美国的计算机教育方向错了?}

\subsection{学会论坛}
\href{http://history.ccf.org.cn/resources/1190201776262/2010/04/15/045088.pdf}{要充分发挥学术共同体在学术评价体系中的作用}

\href{http://history.ccf.org.cn/resources/1190201776262/2010/04/15/045090.pdf}{如何把个人想法上升为学会意志?}

\href{http://history.ccf.org.cn/resources/1190201776262/2010/04/15/045092.pdf}{IEEE计算机学会2009年的新举措}

\subsection{特别报道}
\href{http://history.ccf.org.cn/resources/1190201776262/2010/04/15/045006.pdf}{2009中国计算机大会成功举行}


\section{\href{http://history.ccf.org.cn/sites/ccf/jsjtbbd.jsp?contentId=2542567629049}{\textbf{2009年第10期(总第44期)}}}
Petri网理论与应用的蓬勃发展 近50年,Petri网的抽象研究和描述能力不断地向纵横两个方向扩展。它的应用也逐渐广泛起来,成为完成系统的形式化描述、系统的正确性验证、系统性能的评价、系统的目标实现和测试的有效的图形化数学工具。本期专题从扩展理论和应用两方面对Petri网模型进行了介绍。
\subsection{专题}
\href{http://history.ccf.org.cn/resources/1190201776262/2010/04/15/044006.pdf}{Petri网理论与应用的蓬勃发展(free)}

\href{http://history.ccf.org.cn/resources/1190201776262/2010/04/15/044008.pdf}{Petri网和Pi演算模型的表达能力(free)}

\href{http://history.ccf.org.cn/resources/1190201776262/2010/04/15/044012.pdf}{Petri网的活性研究与应用}

\href{http://history.ccf.org.cn/resources/1190201776262/2010/04/15/044016.pdf}{工作流研究——从WF-net到同步网}

\href{http://history.ccf.org.cn/resources/1190201776262/2010/04/15/044021.pdf}{Petri网模型检测概述}

\href{http://history.ccf.org.cn/resources/1190201776262/2010/04/15/044028.pdf}{不可否认协议的Petri网建模与分析}

\href{http://history.ccf.org.cn/resources/1190201776262/2010/04/15/044036.pdf}{Petri网在电子商务系统分析中的应用}

\subsection{专栏}
\href{http://history.ccf.org.cn/resources/1190201776262/2010/04/15/044045.pdf}{云计算与虚拟化}

\href{http://history.ccf.org.cn/resources/1190201776262/2010/04/15/044049.pdf}{云计算——将计算变成水和电}

\href{http://history.ccf.org.cn/resources/1190201776262/2010/04/15/044055.pdf}{互联网用户生成内容的管理与挖掘}

\href{http://history.ccf.org.cn/resources/1190201776262/2010/04/15/044060.pdf}{从木桶理论的两大误区看人才与企业的匹配}

\href{http://history.ccf.org.cn/resources/1190201776262/2010/04/15/044040.pdf}{明日之星的“姚”篮——记清华大学理论计算机科学研究中心(free)}

\subsection{动态}
\href{http://history.ccf.org.cn/resources/1190201776262/2010/04/15/044062.pdf}{超大规模分布式虚拟现实系统}

\subsection{译文}
\href{http://history.ccf.org.cn/resources/1190201776262/2010/04/15/044070.pdf}{再谈计算机界的女性}

\subsection{学会论坛}
\href{http://history.ccf.org.cn/resources/1190201776262/2010/04/15/044092.pdf}{中国计算机学会监事会2008~2009年度工作报告}

\subsection{特别报道}
\href{http://history.ccf.org.cn/resources/1190201776262/2010/04/15/044088.pdf}{本刊读者调查报告}


\section{\href{http://history.ccf.org.cn/sites/ccf/jsjtbbd.jsp?contentId=2542567629046}{\textbf{2009年第09期(总第43期)}}}
仿真科学与技术 20世纪40年代,仿真催生了第一台电子计算机;如今,计算机科学与技术的发展为复杂系统仿真提供了更多可能,也使仿真科学与技术真正成为一门综合性、交叉性学科。仿真科学与技术在系统科学、控制科学、计算机科学交叉融合中实现了跨越和创新,极大地扩展了人类认知世界的能力。经过近一个世纪的发展,仿真科学与技术发展到了何种程度?本期专题组织了一组文章专门为读者介绍仿真科学与技术在几个典型领域中的发展。
\subsection{专题}
\href{http://history.ccf.org.cn/resources/1190201776262/2010/04/15/043006.pdf}{仿真科学与技术——一个新的技术学科向我们走来(free)}

\href{http://history.ccf.org.cn/resources/1190201776262/2010/04/15/043008.pdf}{体系仿真的关键问题与发展趋势}

\href{http://history.ccf.org.cn/resources/1190201776262/2010/04/15/043015.pdf}{海洋自然环境及人工系统仿真技术综述}

\href{http://history.ccf.org.cn/resources/1190201776262/2010/04/15/043022.pdf}{植物代谢系统的建模与仿真}

\href{http://history.ccf.org.cn/resources/1190201776262/2010/04/15/043028.pdf}{复杂产品虚拟样机工程技术}

\href{http://history.ccf.org.cn/resources/1190201776262/2010/04/15/043039.pdf}{分布分层式协同仿真运行支撑环境}

\subsection{专栏}
\href{http://history.ccf.org.cn/resources/1190201776262/2010/04/15/043052.pdf}{中国计算机事业的机遇窗口期只有10~15年}

\href{http://history.ccf.org.cn/resources/1190201776262/2010/04/15/043055.pdf}{博士生的理想与徘徊(free)}

\href{http://history.ccf.org.cn/resources/1190201776262/2010/04/15/043060.pdf}{一段不应被遗忘的历史——DJS100使用的数据库管理系统SKGX}

\subsection{动态}
\href{http://history.ccf.org.cn/resources/1190201776262/2010/04/15/043068.pdf}{国际软件工程大会}

\href{http://history.ccf.org.cn/resources/1190201776262/2010/04/15/043062.pdf}{需求工程——复杂系统的软件工程的基础研究进展}

\subsection{译文}
\href{http://history.ccf.org.cn/resources/1190201776262/2010/04/15/043070.pdf}{信息系统学科的一项新使命(free)}

\href{http://history.ccf.org.cn/resources/1190201776262/2010/04/15/043073.pdf}{计算机科学该进入成年了(free)}

\subsection{学会论坛}
\href{http://history.ccf.org.cn/resources/1190201776262/2010/04/15/043092.pdf}{高水平学术会议商业运作探索}


\section{\href{http://history.ccf.org.cn/sites/ccf/jsjtbbd.jsp?contentId=2542567629043}{\textbf{2009年第08期(总第42期)}}}
机器学习 让机器具有人的智慧是人类由来已久的梦想。美好的愿景和艰巨的难题都激发着科学家的研究热情。机器学习已成为计算机科学技术中最活跃的研究分支之一。 目前,机器学习技术不仅在计算机视觉、自然语言处理、多媒体、图形学、操作系统以及软件工程等计算机科学的众多领域中发挥作用,而且还在生物信息学、计算金融学等交叉学科领域成为重要的支撑技术。本期专题邀请国内知名专家撰写文章进行综述性介绍,希望读者阅读后对机器学习研究进展做一概要了解。
\subsection{专题}
\href{http://history.ccf.org.cn/resources/1190201776262/2010/04/15/042006.pdf}{机器学习(free)}

\href{http://history.ccf.org.cn/resources/1190201776262/2010/04/15/042007.pdf}{统计机器学习——损失函数与优化求解}

\href{http://history.ccf.org.cn/resources/1190201776262/2010/04/15/042015.pdf}{高维数据降维方法}

\href{http://history.ccf.org.cn/resources/1190201776262/2010/04/15/042023.pdf}{聚类分析}

\href{http://history.ccf.org.cn/resources/1190201776262/2010/04/15/042031.pdf}{贝叶斯网络和因果网络}

\href{http://history.ccf.org.cn/resources/1190201776262/2010/04/15/042036.pdf}{分类器网络的Boosting学习}

\href{http://history.ccf.org.cn/resources/1190201776262/2010/04/15/042042.pdf}{强化学习}

\subsection{专栏}
\href{http://history.ccf.org.cn/resources/1190201776262/2010/04/15/042052.pdf}{志在中国计算机科学2020计划(free)}

\subsection{动态}
\href{http://history.ccf.org.cn/resources/1190201776262/2010/04/15/042058.pdf}{对等计算与广域网虚拟平台(free)}

\href{http://history.ccf.org.cn/resources/1190201776262/2010/04/15/042062.pdf}{ACM计算理论年会}

\subsection{译文}
\href{http://history.ccf.org.cn/resources/1190201776262/2010/04/15/042064.pdf}{编译器研究之路:未来50年}

\subsection{学会论坛}
\href{http://history.ccf.org.cn/resources/1190201776262/2010/04/15/042092.pdf}{NOI要培养学生的计算思维能力}

\href{http://history.ccf.org.cn/resources/1190201776262/2010/04/15/042094.pdf}{YOCSEF的未来发展——《YOCSEF》发布暨YOCSEF发展座谈会上的讲话(free)}


\section{\href{http://history.ccf.org.cn/sites/ccf/jsjtbbd.jsp?contentId=2542567629040}{\textbf{2009年第07期(总第41期)}}}
可视媒体的智能处理 人类获取的外部信息有83%来自视觉,而让计算机获取图像、视频和数字几何等可视媒体信息却不轻松,如何实现可视媒体智能处理是目前学者界努力的重要方向。我们邀请多位国内专家将可视媒体智能处理的热点方向和研究进展介绍给读者,希望能够帮助读者对可视媒体智能处理研究有一个概要性的了解。
\subsection{专题}
\href{http://history.ccf.org.cn/resources/1190201776262/2010/04/15/041006.pdf}{可视媒体的智能处理(free)}

\href{http://history.ccf.org.cn/resources/1190201776262/2010/04/15/041007.pdf}{可视媒体中的人体运动分析}

\href{http://history.ccf.org.cn/resources/1190201776262/2010/04/15/041017.pdf}{视觉搜索的认知机理与应用}

\href{http://history.ccf.org.cn/resources/1190201776262/2010/04/15/041023.pdf}{交互式高效动画的智能制作}

\href{http://history.ccf.org.cn/resources/1190201776262/2010/04/15/041030.pdf}{可视媒体的语义挖掘}

\href{http://history.ccf.org.cn/resources/1190201776262/2010/04/15/041037.pdf}{数字音频水印技术}

\href{http://history.ccf.org.cn/resources/1190201776262/2010/04/15/041044.pdf}{数字化城市}

\subsection{专栏}
\href{http://history.ccf.org.cn/resources/1190201776262/2010/04/15/041050.pdf}{数据密集型计算——数据管理技术面临的挑战}

\href{http://history.ccf.org.cn/resources/1190201776262/2010/04/15/041054.pdf}{2008年入侵检测新进展会议一暼}

\href{http://history.ccf.org.cn/resources/1190201776262/2010/04/15/041060.pdf}{学生才是你的财富(free)}

\subsection{动态}
\href{http://history.ccf.org.cn/resources/1190201776262/2010/04/15/041066.pdf}{计算系统虚拟化基础理论与方法项目研究进展}

\href{http://history.ccf.org.cn/resources/1190201776262/2010/04/15/041071.pdf}{面向城市规划的虚拟现实集成环境}

\subsection{译文}
\href{http://history.ccf.org.cn/resources/1190201776262/2010/04/15/041074.pdf}{多核时代的阿姆达尔定律 (free)}

\subsection{学会论坛}
\href{http://history.ccf.org.cn/resources/1190201776262/2010/04/15/041092.pdf}{关于社团发展的认识(下)}


\section{\href{http://history.ccf.org.cn/sites/ccf/jsjtbbd.jsp?contentId=2542567629037}{\textbf{2009年第06期(总第40期)}}}
云计算的大幕已经拉开 随着信息技术的广泛应用和快速发展,云计算作为一种新兴的商业计算模型得到了人们的广关注。2008年,“云计算”的概念因其更清晰的商业模式而受到广泛关注,并得到工业和学术界的普遍认可,成为最热门的词汇。2009年,云计算的大幕已经拉开。本期专题是由国内相关领域的专家、学者和来自谷歌、微软、英特尔、惠普的企业界领袖、学者撰写的文章,他们将从不同的角度,用战略的眼光从结构、功能、管理到应用的各个层面解析什么是云计算。
\subsection{专题}
\href{http://history.ccf.org.cn/resources/1190201776262/2010/04/15/040006.pdf}{云计算的大幕已经拉开(free)}

\href{http://history.ccf.org.cn/resources/1190201776262/2010/04/15/040008.pdf}{透明计算——面向服务共享的计算模式}

\href{http://history.ccf.org.cn/resources/1190201776262/2010/04/15/040016.pdf}{2009——云计算普及年}

\href{http://history.ccf.org.cn/resources/1190201776262/2010/04/15/040018.pdf}{与“云”共舞——再谈云计算}

\href{http://history.ccf.org.cn/resources/1190201776262/2010/04/15/040022.pdf}{漫谈云计算}

\href{http://history.ccf.org.cn/resources/1190201776262/2010/04/15/040026.pdf}{Cells-as-a-Service——一项云计算基础设施服务}

\href{http://history.ccf.org.cn/resources/1190201776262/2010/04/15/040032.pdf}{面向服务的云计算基础设施}

\href{http://history.ccf.org.cn/resources/1190201776262/2010/04/15/040044.pdf}{云存储}

\subsection{专栏}
\href{http://history.ccf.org.cn/resources/1190201776262/2010/04/15/040054.pdf}{P2P存储在云计算时代的新的机遇(free)}

\href{http://history.ccf.org.cn/resources/1190201776262/2010/04/15/040057.pdf}{虚拟器件技术及应用}

\href{http://history.ccf.org.cn/resources/1190201776262/2010/04/15/040062.pdf}{人类思维特征及其人工智能模型探讨}

\subsection{动态}
\href{http://history.ccf.org.cn/resources/1190201776262/2010/04/15/040066.pdf}{计算机软件新技术国家重点实验室的研究工作进展}

\href{http://history.ccf.org.cn/resources/1190201776262/2010/04/15/040070.pdf}{IEEE INFOCOM 2009 会议}

\subsection{译文}
\href{http://history.ccf.org.cn/resources/1190201776262/2010/04/15/040072.pdf}{云计算真的准备好唱主角了吗?}

\href{http://history.ccf.org.cn/resources/1190201776262/2010/04/15/040078.pdf}{编程的技艺}

\subsection{学会论坛}
\href{http://history.ccf.org.cn/resources/1190201776262/2010/04/15/040092.pdf}{关于社团发展的认识(上)}


\section{\href{http://history.ccf.org.cn/sites/ccf/jsjtbbd.jsp?contentId=2542567629034}{\textbf{2009年第05期(总第39期)}}}
一座闪亮的里程碑 我们到底需要什么样的精神和条件才能实现跨越式发展?本期专题以30多年前的“产学研合作”告诉我们……
\subsection{专题}
\href{http://history.ccf.org.cn/resources/1190201776262/2010/04/15/039006.pdf}{一座闪亮的里程碑——DJS100系列(free)}

\href{http://history.ccf.org.cn/resources/1190201776262/2010/04/15/039008.pdf}{联合设计和研制的回顾}

\href{http://history.ccf.org.cn/resources/1190201776262/2010/04/15/039014.pdf}{内外协同 建功立业}

\href{http://history.ccf.org.cn/resources/1190201776262/2010/04/15/039016.pdf}{联合设计成功的启示}

\href{http://history.ccf.org.cn/resources/1190201776262/2010/04/15/039022.pdf}{计算机历史上光辉灿烂的一页}

\href{http://history.ccf.org.cn/resources/1190201776262/2010/04/15/039024.pdf}{保障研制成功 器件工作先行}

\href{http://history.ccf.org.cn/resources/1190201776262/2010/04/15/039027.pdf}{研制小型机的梦想实现了}

\href{http://history.ccf.org.cn/resources/1190201776262/2010/04/15/039030.pdf}{DJS130及其H761的往事回眸}

\href{http://history.ccf.org.cn/resources/1190201776262/2010/04/15/039035.pdf}{我从事计算机科研工作的起点}

\href{http://history.ccf.org.cn/resources/1190201776262/2010/04/15/039038.pdf}{动荡的年代  艰苦的工作}

\href{http://history.ccf.org.cn/resources/1190201776262/2010/04/15/039041.pdf}{北京计算机三厂参加研制纪实}

\href{http://history.ccf.org.cn/resources/1190201776262/2010/04/15/039043.pdf}{系统软件联合设计}

\href{http://history.ccf.org.cn/resources/1190201776262/2010/04/15/039044.pdf}{联合设计的“外围”遭遇战}

\subsection{专栏}
\href{http://history.ccf.org.cn/resources/1190201776262/2010/04/15/039048.pdf}{自由式虚拟现实:一个新的目标?(free)}

\href{http://history.ccf.org.cn/resources/1190201776262/2010/04/15/039054.pdf}{信息安全中五大检测机制}

\href{http://history.ccf.org.cn/resources/1190201776262/2010/04/15/039050.pdf}{我国高清光盘技术研究进展}

\href{http://history.ccf.org.cn/resources/1190201776262/2010/04/15/039052.pdf}{培养高质量计算机学科研究生的挑战}

\subsection{动态}
\href{http://history.ccf.org.cn/resources/1190201776262/2010/04/15/039058.pdf}{数据库研究进展——2008数据库高级研讨会报告}

\href{http://history.ccf.org.cn/resources/1190201776262/2010/04/15/039065.pdf}{生物信息学、系统生物学和智能计算联合国际会议}

\href{http://history.ccf.org.cn/resources/1190201776262/2010/04/15/039066.pdf}{IEEE INFOCOM会议}

\subsection{译文}
\href{http://history.ccf.org.cn/resources/1190201776262/2010/04/15/039068.pdf}{改变世界的10篇网络论文}

\href{http://history.ccf.org.cn/resources/1190201776262/2010/04/15/039072.pdf}{赞美脚本编程——真正的编程实用主义}

\subsection{学会论坛}
\href{http://history.ccf.org.cn/resources/1190201776262/2010/04/15/039092.pdf}{学者的坚果仁(free)}


\section{\href{http://history.ccf.org.cn/sites/ccf/jsjtbbd.jsp?contentId=2542567629031}{\textbf{2009年第04期(总第38期)}}}
不确定性数据管理技术及其应用 除了结构化和半结构化数据,在经济、军事和通信等应用领域还存在着大量不确定性数据。这些数据的存在性未知且属性值存在误差。本期专题介绍不确定性数据研究面临的挑战、当前全球研究机构的进展和国际会议,还介绍了不确定性数据管理技术研究中诸如数据集成、世系分析和数据流分析等各种方法。读者通过这组文章可以全面了解不确定性数据研究的概况。
\subsection{专题}
\href{http://history.ccf.org.cn/resources/1190201776262/2010/04/15/038006.pdf}{不确定性数据管理的要求与挑战(free)}

\href{http://history.ccf.org.cn/resources/1190201776262/2010/04/15/038014.pdf}{不确定性数据的世系管理}

\href{http://history.ccf.org.cn/resources/1190201776262/2010/04/15/038021.pdf}{不确定移动对象管理}

\href{http://history.ccf.org.cn/resources/1190201776262/2010/04/15/038031.pdf}{不确定图数据管理研究现状}

\href{http://history.ccf.org.cn/resources/1190201776262/2010/04/15/038037.pdf}{不确定数据流管理技术}

\href{http://history.ccf.org.cn/resources/1190201776262/2010/04/15/038045.pdf}{无线传感器网络的不确定性}

\href{http://history.ccf.org.cn/resources/1190201776262/2010/04/15/038052.pdf}{概率XML结构模型及查询技术}

\subsection{专栏}
\href{http://history.ccf.org.cn/resources/1190201776262/2010/04/15/038060.pdf}{中国企业技术产品创新中的几个问题分析(free)}

\href{http://history.ccf.org.cn/resources/1190201776262/2010/04/15/038067.pdf}{牛年的安全牛不牛}

\href{http://history.ccf.org.cn/resources/1190201776262/2010/04/15/038070.pdf}{国内计算机专业研究生的质量下降了吗?(free)}

\subsection{动态}
\href{http://history.ccf.org.cn/resources/1190201776262/2010/04/15/038074.pdf}{863计划“高效能计算机及网格服务环境”重大项目进展}

\subsection{译文}
\href{http://history.ccf.org.cn/resources/1190201776262/2010/04/15/038080.pdf}{请给我信息,而不是技术}

\subsection{学会论坛}
\href{http://history.ccf.org.cn/resources/1190201776262/2010/04/15/038094.pdf}{计算对21世纪的K-12 STEM技能教育至关重要}


\section{\href{http://history.ccf.org.cn/sites/ccf/jsjtbbd.jsp?contentId=2542567629028}{\textbf{2009年第03期(总第37期)}}}
2008年中国计算机学会优秀博士学位论文奖专题 2008年有6位博士获得了中国计算机学会优秀博士学位论文奖,4位博士获提名奖。他们的论文涉及理论计算机科学、数据库技术、软件理论、计算机图形学、分布式计算、图像视频与模式识别、计算机网络、人工智能与生物信息学等方面,获得了国内外专家的一致肯定,也反映了国内青年学者的学术关注点。为了让广大会员能够一览他们的研究成果,本期专题邀请上述10位获奖者就获奖论文研究成果作综述介绍,以飨读者。
\subsection{专题}
\href{http://history.ccf.org.cn/resources/1190201776262/2010/04/15/037010.pdf}{中国计算机学会2008年度优秀博士论文评选(free)}

\href{http://history.ccf.org.cn/resources/1190201776262/2010/04/15/037012.pdf}{大尺度几何形变理论与方法(free)}

\href{http://history.ccf.org.cn/resources/1190201776262/2010/04/15/037019.pdf}{非规则计算中的局部性和并行性(free)}

\href{http://history.ccf.org.cn/resources/1190201776262/2010/04/15/037027.pdf}{对等网络流媒体直播}

\href{http://history.ccf.org.cn/resources/1190201776262/2010/04/15/037036.pdf}{多示例学习与多标记学习}

\href{http://history.ccf.org.cn/resources/1190201776262/2010/04/15/037044.pdf}{基于k近邻分类准则的特征变换算法}

\href{http://history.ccf.org.cn/resources/1190201776262/2010/04/15/037052.pdf}{海量多媒体数据库高效查询处理}

\href{http://history.ccf.org.cn/resources/1190201776262/2010/04/15/037061.pdf}{分布式进程通信范型形式化验证}

\href{http://history.ccf.org.cn/resources/1190201776262/2010/04/15/037068.pdf}{计算机图形学中若干基本问题及光线跟踪技术}

\href{http://history.ccf.org.cn/resources/1190201776262/2010/04/15/037077.pdf}{Max-SAT问题的求解方法}

\href{http://history.ccf.org.cn/resources/1190201776262/2010/04/15/037084.pdf}{PCNN及其在指纹系统中的应用}

\subsection{特别报道}
\href{http://history.ccf.org.cn/resources/1190201776262/2010/04/15/037008.pdf}{表彰先进是学会的重要职能}

\href{http://history.ccf.org.cn/resources/1190201776262/2010/04/15/037006.pdf}{青年科学家要承担起科教兴国的重任}

\href{http://history.ccf.org.cn/resources/1190201776262/2010/04/15/037007.pdf}{关心青年人才成长是微软的责任}


\section{\href{http://history.ccf.org.cn/sites/ccf/jsjtbbd.jsp?contentId=2542567629025}{\textbf{2009年第02期(总第36期)}}}
计算机科学的变革 2008年6月29日至7月14日,中国国家自然科学基金委员会组织计算机科学代表团访问了美国加州大学洛杉矶分校等10所大学和美国国家科学基金会。本期专题讨论了中美两国与计算机科学变革相关的研究热点、教师队伍建设、学科建设和招生与培养等问题。
\subsection{专题}
\href{http://history.ccf.org.cn/resources/1190201776262/2010/04/15/036006.pdf}{计算机科学的变革(free)}

\href{http://history.ccf.org.cn/resources/1190201776262/2010/04/15/036010.pdf}{主题报告和分组讨论评述(free)}

\href{http://history.ccf.org.cn/resources/1190201776262/2010/04/15/036015.pdf}{21世纪计算机科学的研究热点(free)}

\href{http://history.ccf.org.cn/resources/1190201776262/2010/04/15/036023.pdf}{美国大学计算机与信息科学的跨学科发展(free)}

\href{http://history.ccf.org.cn/resources/1190201776262/2010/04/15/036028.pdf}{向着星星攀登——访美归来谈教师队伍建设和业界支持(free)}

\href{http://history.ccf.org.cn/resources/1190201776262/2010/04/15/036035.pdf}{关于美国大学本科教学的思考(free)}

\subsection{专栏}
\href{http://history.ccf.org.cn/resources/1190201776262/2010/04/15/036040.pdf}{未来的科技创新}

\href{http://history.ccf.org.cn/resources/1190201776262/2010/04/15/036042.pdf}{专业教育的科学化}

\href{http://history.ccf.org.cn/resources/1190201776262/2010/04/15/036038.pdf}{警钟:美国的创新环境恶化?(free)}

\href{http://history.ccf.org.cn/resources/1190201776262/2010/04/15/036047.pdf}{浅析云计算安全的核心问题}

\href{http://history.ccf.org.cn/resources/1190201776262/2010/04/15/036049.pdf}{数据质量}

\href{http://history.ccf.org.cn/resources/1190201776262/2010/04/15/036052.pdf}{中国电信业借对等网络挑战移动核心网国际标准}

\subsection{动态}
\href{http://history.ccf.org.cn/resources/1190201776262/2010/04/15/036056.pdf}{构建协同共享的可信软件生产环境}

\href{http://history.ccf.org.cn/resources/1190201776262/2010/04/15/036062.pdf}{50个重要的计算机领域国际会议}

\subsection{译文}
\href{http://history.ccf.org.cn/resources/1190201776262/2010/04/15/036064.pdf}{提高安全性的自动代码审查工具}

\href{http://history.ccf.org.cn/resources/1190201776262/2010/04/15/036069.pdf}{有机用户界面}

\subsection{学会论坛}
\href{http://history.ccf.org.cn/resources/1190201776262/2010/04/15/036086.pdf}{计算机工程教育认证情况介绍}


\section{\href{http://history.ccf.org.cn/sites/ccf/jsjtbbd.jsp?contentId=2542567629022}{\textbf{2009年第01期(总第35期)}}}
2008中国计算机大会特邀报告 由中国计算机学会(CCF)和西安市人民政府主办,西北工业大学和西安市科技局承办的2008中国计算机大会(China National Computer Conference,CNCC)于2008年9月25日~27日在中国西安举行。大会以“突出科技创新主旋律,展现计算技术新发展”为主题,邀请了两院院士、国内外知名专家和著名企业代表共11位介绍了当前计算机科学与技术领域的最新动态,并探讨了未来的发展趋势,报告主题涉及高性能计算、软件工程、虚拟现实、普适计算、分布式计算与服务计算等多个热点问题。
\subsection{特邀报告}
\href{http://history.ccf.org.cn/resources/1190201776262/2010/04/15/035006.pdf}{2008中国计算机大会开幕词}

\href{http://history.ccf.org.cn/resources/1190201776262/2010/04/15/035007.pdf}{网络时代的软件工程}

\href{http://history.ccf.org.cn/resources/1190201776262/2010/04/15/035013.pdf}{软件确保(free)}

\href{http://history.ccf.org.cn/resources/1190201776262/2010/04/15/035017.pdf}{虚拟现实中的科学技术问题(free)}

\href{http://history.ccf.org.cn/resources/1190201776262/2010/04/15/035022.pdf}{数字化城市与应急管理(free)}

\href{http://history.ccf.org.cn/resources/1190201776262/2010/04/15/035025.pdf}{高性能计算机体系结构的发展(free)}

\href{http://history.ccf.org.cn/resources/1190201776262/2010/04/15/035031.pdf}{对软件方法学基本问题的思考(free)}

\href{http://history.ccf.org.cn/resources/1190201776262/2010/04/15/035035.pdf}{冗余数据删除存储系统}

\href{http://history.ccf.org.cn/resources/1190201776262/2010/04/15/035039.pdf}{提高健康、安全和生活质量的模式识别}

\href{http://history.ccf.org.cn/resources/1190201776262/2010/04/15/035043.pdf}{云计算和惠普}

\href{http://history.ccf.org.cn/resources/1190201776262/2010/04/15/035046.pdf}{处理器未来二十年}

\href{http://history.ccf.org.cn/resources/1190201776262/2010/04/15/035049.pdf}{抗恶劣环境计算机创新与研发}

\subsection{译文}
\href{http://history.ccf.org.cn/resources/1190201776262/2010/04/15/035054.pdf}{软件测试的原则}

\href{http://history.ccf.org.cn/resources/1190201776262/2010/04/15/035058.pdf}{计算之声}

\subsection{学会论坛}
\href{http://history.ccf.org.cn/resources/1190201776262/2010/04/15/035082.pdf}{学术社团也是推动教育变革的重要力量}

\href{http://history.ccf.org.cn/resources/1190201776262/2010/04/15/035078.pdf}{一个学会的民主改革之路}


\section{\href{http://history.ccf.org.cn/sites/ccf/jsjtbbd.jsp?contentId=2542567629019}{\textbf{2008年第12期(总第34期)}}}
生物计算机时代即将来临 生物计算是近十几年发展起来的一个重要研究领域,其重要科学和技术问题的解决涉及计算机科学、生命科学、化学、数学等诸多传统学科,不但会产生具有重要优越性的计算系统,而且还会对生物科学领域的发展产生深远影响。
\subsection{专题}
\href{http://history.ccf.org.cn/resources/1190201776262/2010/04/15/034012.pdf}{生物计算机的时代即将来临(free)}

\href{http://history.ccf.org.cn/resources/1190201776262/2010/04/15/034014.pdf}{基于DNA计算的信息安全发展(free)}

\href{http://history.ccf.org.cn/resources/1190201776262/2010/04/15/034023.pdf}{分子计算的过去、现在和将来}

\href{http://history.ccf.org.cn/resources/1190201776262/2010/04/15/034030.pdf}{基于生物传感器技术的DNA计算(free)}

\href{http://history.ccf.org.cn/resources/1190201776262/2010/04/15/034037.pdf}{DNA计算中的分子图灵机与分子自动机}

\href{http://history.ccf.org.cn/resources/1190201776262/2010/04/15/034044.pdf}{成功与挑战}

\subsection{专栏}
\href{http://history.ccf.org.cn/resources/1190201776262/2010/04/15/034048.pdf}{对计算教育的七大挑战}

\href{http://history.ccf.org.cn/resources/1190201776262/2010/04/15/034066.pdf}{全球兴起开放教育资源热}


\section{\href{http://history.ccf.org.cn/sites/ccf/jsjtbbd.jsp?contentId=2542567629016}{\textbf{2008年第11期(总第33期)}}}
网络存储技术 现在,全球每年新产生的数字化信息正呈几何级数增长,对存储的需求也在日益增多。存储系统不再是计算机系统的附属设备,而成为互联网中与计算、传输设施同等重要的三大基石之一。可以预计,信息技术正从以计算设备为核心的计算时代和以交换机为中心的网络时代进入到以存储为核心的存储时代。
\subsection{专题}
\href{http://history.ccf.org.cn/resources/1190201776262/2010/04/15/033016.pdf}{网络存储技术(free)}

\href{http://history.ccf.org.cn/resources/1190201776262/2010/04/15/033018.pdf}{进化海量存储系统关键技术与实现方法(free)}

\href{http://history.ccf.org.cn/resources/1190201776262/2010/04/15/033027.pdf}{主动存储技术及其在对象存储中的实现(free)}

\href{http://history.ccf.org.cn/resources/1190201776262/2010/04/15/033034.pdf}{一种高性能和高可靠性的网络虚拟存储系统(free)}

\href{http://history.ccf.org.cn/resources/1190201776262/2010/04/15/033044.pdf}{开放式对等网络存储系统所面临的挑战}

\href{http://history.ccf.org.cn/resources/1190201776262/2010/04/15/033050.pdf}{基于万兆局域网的存储系统发展现状与趋势}

\href{http://history.ccf.org.cn/resources/1190201776262/2010/04/15/033059.pdf}{互联网的寄生存储潜力}

\href{http://history.ccf.org.cn/resources/1190201776262/2010/04/15/033063.pdf}{持续数据保护技术研究}

\href{http://history.ccf.org.cn/resources/1190201776262/2010/04/15/033068.pdf}{网络存储技术在地理信息系统中的应用研究}

\subsection{专栏}
\href{http://history.ccf.org.cn/resources/1190201776262/2010/04/15/033080.pdf}{CCF会成为一朵云吗?}

\href{http://history.ccf.org.cn/resources/1190201776262/2010/04/15/033074.pdf}{在创建可持续发展世界中的若干计算机科学问题}


\section{\href{http://history.ccf.org.cn/sites/ccf/jsjtbbd.jsp?contentId=2542567629013}{\textbf{2008年第10期(总第32期)}}}
互联网服务——让计算以人为本 架构在互联网上的各类互联网信息服务是由全世界人民共同建设出来的一项伟大工程。毫无疑问,大部分互联网服务的成功得益于其所提供服务的简洁性和用户参与的低门槛。然而,为了追求简洁高效的公众服务能力,现有的计算机与网络正变得越来越复杂。这种复杂的能力与简洁化的价值之间的矛盾,让人们开始思考是否存在一种让系统、人与社会更加和谐,让计算与服务同时简洁的网络世界。
\subsection{专题}
\href{http://history.ccf.org.cn/resources/1190201776262/2010/04/15/032012.pdf}{互联网服务——让计算以人为本(free)}

\href{http://history.ccf.org.cn/resources/1190201776262/2010/04/15/032015.pdf}{为人民计算的三个问题(free)}

\href{http://history.ccf.org.cn/resources/1190201776262/2010/04/15/032022.pdf}{Web2.0下的社会标注(free)}

\href{http://history.ccf.org.cn/resources/1190201776262/2010/04/15/032029.pdf}{面向语义网的互联网多媒体服务}

\href{http://history.ccf.org.cn/resources/1190201776262/2010/04/15/032035.pdf}{对等网络服务的现状与展望}

\href{http://history.ccf.org.cn/resources/1190201776262/2010/04/15/032043.pdf}{基于地理信息的用户行为理解}

\href{http://history.ccf.org.cn/resources/1190201776262/2010/04/15/032050.pdf}{以搜索引擎为核心整合互联网服务}

\subsection{专栏}
\href{http://history.ccf.org.cn/resources/1190201776262/2010/04/15/032056.pdf}{从2008年虚拟执行环境会议看系统级虚拟化研究方向}

\href{http://history.ccf.org.cn/resources/1190201776262/2010/04/15/032078.pdf}{软件即服务}

\href{http://history.ccf.org.cn/resources/1190201776262/2010/04/15/032083.pdf}{云计算}

\href{http://history.ccf.org.cn/resources/1190201776262/2010/04/15/032066.pdf}{我们能从编程中解放出来使其成为历史吗?}


\section{\href{http://history.ccf.org.cn/sites/ccf/jsjtbbd.jsp?contentId=2542567629010}{\textbf{2008年第09期(总第31期)}}}
永恒的操作系统多彩旋律 操作系统在硬件技术和应用需求的驱动下不断向前发展。各类计算机都需要有与之相适应的操作系统为其提供支持。我们不但能看到支持桌面应用的Windows XP和Linux等操作系统,或支持移动便携式应用的Linux DA O/S、Palm OS和WindowsCE等操作系统,也能看到支持超级计算应用的UNICOS等操作系统;不但能看到支持分时应用的UNIX等操作系统,也能看到支持实时应用的QNX等操作系统。
\subsection{专题}
\href{http://history.ccf.org.cn/resources/1190201776262/2010/04/15/031012.pdf}{永恒的操作系统多彩旋律(free)}

\href{http://history.ccf.org.cn/resources/1190201776262/2010/04/15/031015.pdf}{现代操作系统的发展(free)}

\href{http://history.ccf.org.cn/resources/1190201776262/2010/04/15/031023.pdf}{集群操作系统的发展与展望}

\href{http://history.ccf.org.cn/resources/1190201776262/2010/04/15/031031.pdf}{操作系统安全性研究的演进}

\href{http://history.ccf.org.cn/resources/1190201776262/2010/04/15/031040.pdf}{Windows操作系统的发展}

\href{http://history.ccf.org.cn/resources/1190201776262/2010/04/15/031050.pdf}{UNIX类操作系统的发展}

\href{http://history.ccf.org.cn/resources/1190201776262/2010/04/15/031060.pdf}{我们能使操作系统安全可靠吗?}

\subsection{专栏}
\href{http://history.ccf.org.cn/resources/1190201776262/2010/04/15/031070.pdf}{近十年我国信息人才培养的发展状况}

\href{http://history.ccf.org.cn/resources/1190201776262/2010/04/15/031072.pdf}{未来计算技术研究展望}

\href{http://history.ccf.org.cn/resources/1190201776262/2010/04/15/031076.pdf}{可从隐式反馈中进行学习的搜索引擎}


\section{\href{http://history.ccf.org.cn/sites/ccf/jsjtbbd.jsp?contentId=2542567629007}{\textbf{2008年第08期(总第30期)}}}
人工智能在中国 办这一期专题的想法源于2005年《IEEE 智能系统》(IEEE Intelligent Systems)杂志的一次编委会议。当时讨论的主题是《人工智能50年》(AI 50)专刊的准备工作。在会上,作者提出专刊内容中没有中国的人工智能研究不合理,可否再出一期《人工智能在中国》(AI in China)的专刊?这一建议得到了主编吉姆 • 亨德勒(Jim Hendler)和许多编委的响应... ...
\subsection{专题}
\href{http://history.ccf.org.cn/resources/1190201776262/2010/04/15/030012.pdf}{人工智能在中国(free)}

\href{http://history.ccf.org.cn/resources/1190201776262/2010/04/15/030015.pdf}{人工智能发展近况(free)}

\href{http://history.ccf.org.cn/resources/1190201776262/2010/04/15/030028.pdf}{中国图像识别50年}

\href{http://history.ccf.org.cn/resources/1190201776262/2010/04/15/030039.pdf}{中国语言技术进展}

\href{http://history.ccf.org.cn/resources/1190201776262/2010/04/15/030049.pdf}{机器学习研究:现状与分析}

\href{http://history.ccf.org.cn/resources/1190201776262/2013/09/29/030057.pdf}{书法创作过程与形象思维模拟}

\href{http://history.ccf.org.cn/resources/1190201776262/2010/04/15/030064.pdf}{知识网格环境}

\subsection{专栏}
\href{http://history.ccf.org.cn/resources/1190201776262/2010/04/15/030074.pdf}{五大军事电子信息技术现状与展望}

\href{http://history.ccf.org.cn/resources/1190201776262/2010/04/15/030082.pdf}{从计算智能到网络智能}

\href{http://history.ccf.org.cn/resources/1190201776262/2010/04/15/030086.pdf}{游戏智能}


\section{\href{http://history.ccf.org.cn/sites/ccf/jsjtbbd.jsp?contentId=2542567629004}{\textbf{2008年第07期(总第29期)}}}
量子计算与量子通信 正如足球迷关注世界杯的“战事”那样,许多科技工作者正在期待另外一场科技赛事。2007年2月13日,加拿大D-Wave公司宣布研制成功16量子比特的量子计算机——“猎户座”(Orion)... ...
\subsection{专题}
\href{http://history.ccf.org.cn/resources/1190201776262/2010/04/15/029012.pdf}{量子计算与量子通信(free)}

\href{http://history.ccf.org.cn/resources/1190201776262/2010/04/15/029014.pdf}{量子信息基础与量子算法(free)}

\href{http://history.ccf.org.cn/resources/1190201776262/2010/04/15/029025.pdf}{量子计算机的硬件进展(free)}

\href{http://history.ccf.org.cn/resources/1190201776262/2010/04/15/029044.pdf}{量子程序理论及相关问题研究}

\href{http://history.ccf.org.cn/resources/1190201776262/2010/04/15/029057.pdf}{连续变量量子信息}

\href{http://history.ccf.org.cn/resources/1190201776262/2010/04/15/029065.pdf}{量子控制及其在量子计算中的应用}

\href{http://history.ccf.org.cn/resources/1190201776262/2010/04/15/029072.pdf}{量子保密通信}


\section{\href{http://history.ccf.org.cn/sites/ccf/jsjtbbd.jsp?contentId=2542567629001}{\textbf{2008年第06期(总第28期)}}}
超级计算中心——高性能计算引领创新的基地 随着计算技术的发展,高性能计算已经成为与理论研究和科学实验并列的第三种科学研究方法,成为一项促进重大科学发现和经济发展的战略支撑技术,在诸多场合可以取代真实实验以节省昂贵费用,以及模拟在人工条件下难以实现的物理条件......
\subsection{专题}
\href{http://history.ccf.org.cn/resources/1190201776262/2010/04/15/028026.pdf}{超级计算中心——高性能计算引领创新的基地(free)}

\href{http://history.ccf.org.cn/resources/1190201776262/2010/04/15/028028.pdf}{高性能计算公共服务平台的特点与展望(free)}

\href{http://history.ccf.org.cn/resources/1190201776262/2010/04/15/028032.pdf}{世界各地超级计算中心的发展概况(free)}

\href{http://history.ccf.org.cn/resources/1190201776262/2010/04/15/028043.pdf}{芬兰高性能计算体系——高水平设施和服务促进研究创新}

\href{http://history.ccf.org.cn/resources/1190201776262/2010/04/15/028051.pdf}{高性能计算——基础科学研究的利器}

\href{http://history.ccf.org.cn/resources/1190201776262/2010/04/15/028057.pdf}{加速我国高性能计算发展}

\href{http://history.ccf.org.cn/resources/1190201776262/2010/04/15/028062.pdf}{汽车安全性分析中的高性能计算技术概述}

\subsection{专栏}
\href{http://history.ccf.org.cn/resources/1190201776262/2010/04/15/028066.pdf}{求解油气方程}

\href{http://history.ccf.org.cn/resources/1190201776262/2010/04/15/028071.pdf}{斯普林特的宽带赌博}

\href{http://history.ccf.org.cn/resources/1190201776262/2010/04/15/028076.pdf}{仍在等待中的纳米管存储芯片}

\href{http://history.ccf.org.cn/resources/1190201776262/2010/04/15/028079.pdf}{世界信息产业演进规律}

\subsection{特别报道}
\href{http://history.ccf.org.cn/resources/1190201776262/2010/04/15/028015.pdf}{YOCSEF十周年纪念征文选登}

\href{http://history.ccf.org.cn/resources/1190201776262/2010/04/15/028013.pdf}{在YOCSEF十年特别论坛上的发言}

\href{http://history.ccf.org.cn/resources/1190201776262/2010/04/15/028008.pdf}{十年激情  十年成长——纪念YOCSEF创建十周年特别论坛侧记}


\section{\href{http://history.ccf.org.cn/sites/ccf/jsjtbbd.jsp?contentId=2542567628998}{\textbf{2008年第05期(总第27期)}}}
发展汽车计算平台——工业化与信息化融合的战略选择 国家汽车计算平台工程是集汽车电子控制、信息处理、应用服务为一体的创新工程,其实施目标是开发汽车集成计算平台与智能计算新技术......
\subsection{专题}
\href{http://history.ccf.org.cn/resources/1190201776262/2010/04/15/027012.pdf}{发展汽车计算平台(free)}

\href{http://history.ccf.org.cn/resources/1190201776262/2010/04/15/027014.pdf}{世界汽车计算平台体系结构研究}

\href{http://history.ccf.org.cn/resources/1190201776262/2010/04/15/027022.pdf}{汽车中央计算机及软件系统发展现状与趋势}

\href{http://history.ccf.org.cn/resources/1190201776262/2010/04/15/027027.pdf}{汽车网络发展现状与趋势}

\href{http://history.ccf.org.cn/resources/1190201776262/2010/04/15/027033.pdf}{汽车电子控制系统发展现状及趋势(free)}

\href{http://history.ccf.org.cn/resources/1190201776262/2010/04/15/027038.pdf}{汽车车载电子系统发展现状与趋势}

\href{http://history.ccf.org.cn/resources/1190201776262/2010/04/15/027042.pdf}{汽车用电子元器件发展现状与趋势}

\href{http://history.ccf.org.cn/resources/1190201776262/2010/04/15/027046.pdf}{汽车计算平台领域专利技术分析(free)}

\subsection{专栏}
\href{http://history.ccf.org.cn/resources/1190201776262/2010/04/15/027054.pdf}{世界软件产业现状及趋势分析}

\href{http://history.ccf.org.cn/resources/1190201776262/2010/04/15/027060.pdf}{最好的介质是真空}

\href{http://history.ccf.org.cn/resources/1190201776262/2010/04/15/027064.pdf}{“足球妈咪”们讨厌的卫星电视}

\href{http://history.ccf.org.cn/resources/1190201776262/2010/04/15/027066.pdf}{使用OLIVE建立你自己的虚拟世界}

\href{http://history.ccf.org.cn/resources/1190201776262/2010/04/15/027070.pdf}{世界信息技术与产品发展态势与特点}

\href{http://history.ccf.org.cn/resources/1190201776262/2010/04/15/027076.pdf}{SoftUDC:基于软件的面向能力计算的数据中心}


\section{\href{http://history.ccf.org.cn/sites/ccf/jsjtbbd.jsp?contentId=2542567628995}{\textbf{2008年第04期(总第26期)}}}
虚拟化的复兴 “虚拟化(Virtualization)技术”的产生可以追溯到上世纪60年代。“虚拟”一词最早来源于光学,用于描述镜子里的物体。之所以使用“虚拟”的概念定义计算机领域的一项技术,是因为人们希望虚拟机看起来和工作起来都和真正的机器一摸一样。这同时也意味着......
\subsection{专题}
\href{http://history.ccf.org.cn/resources/1190201776262/2010/04/15/026012.pdf}{虚拟化的复兴(free)}

\href{http://history.ccf.org.cn/resources/1190201776262/2010/04/15/026015.pdf}{计算系统虚拟化:体系结构领域的重要挑战}

\href{http://history.ccf.org.cn/resources/1190201776262/2010/04/15/026024.pdf}{虚拟环境下计算机系统的可靠性增强技术}

\href{http://history.ccf.org.cn/resources/1190201776262/2010/04/15/026034.pdf}{基于虚拟化技术的服务计算开放软件模型与协同机理}

\href{http://history.ccf.org.cn/resources/1190201776262/2010/04/15/026042.pdf}{虚拟机监视器的运行层次与实现技术}

\href{http://history.ccf.org.cn/resources/1190201776262/2010/04/15/026050.pdf}{虚拟计算的系统安全}

\href{http://history.ccf.org.cn/resources/1190201776262/2010/04/15/026056.pdf}{面向领域的业务服务建模和服务虚拟化}

\href{http://history.ccf.org.cn/resources/1190201776262/2010/04/15/026064.pdf}{系统级多机虚拟化:现状与挑战}

\href{http://history.ccf.org.cn/resources/1190201776262/2010/04/15/026076.pdf}{共享底层结构下的分布式虚拟环境}

\subsection{专栏}
\href{http://history.ccf.org.cn/resources/1190201776262/2010/04/15/026080.pdf}{2007年高性能计算机排行榜对比分析(续)}

\href{http://history.ccf.org.cn/resources/1190201776262/2010/04/15/026086.pdf}{中国软件外包面临的难题、挑战与机遇(free)}


\section{\href{http://history.ccf.org.cn/sites/ccf/jsjtbbd.jsp?contentId=2542567628992}{\textbf{2008年第03期(总第25期)}}}

\subsection{专题}
\href{http://history.ccf.org.cn/resources/1190201776262/2010/04/15/025016.pdf}{中国计算机学会2007年度优秀博士论文评选综述(free)}

\href{http://history.ccf.org.cn/resources/1190201776262/2010/04/15/025018.pdf}{量子可分辨性与量子纠缠}

\href{http://history.ccf.org.cn/resources/1190201776262/2010/04/15/025023.pdf}{基于特征的纹理图像分割技术}

\href{http://history.ccf.org.cn/resources/1190201776262/2010/04/15/025030.pdf}{数字电路测试压缩方法研究}

\href{http://history.ccf.org.cn/resources/1190201776262/2010/04/15/025036.pdf}{面向特征的领域建模技术研究}

\href{http://history.ccf.org.cn/resources/1190201776262/2010/04/15/025046.pdf}{关系数据库关键词检索性能优化技术研究}

\href{http://history.ccf.org.cn/resources/1190201776262/2010/04/15/025054.pdf}{共享主存多SIMD结构编译优化及结构研究}

\href{http://history.ccf.org.cn/resources/1190201776262/2010/04/15/025062.pdf}{无线传感器网络数据汇聚路由问题的研究}

\href{http://history.ccf.org.cn/resources/1190201776262/2010/04/15/025070.pdf}{神经网络的动力学行为研究与发展趋势}

\href{http://history.ccf.org.cn/resources/1190201776262/2010/04/15/025074.pdf}{当对等网络遇到网络编码:多播吞吐率还能提高吗?}

\href{http://history.ccf.org.cn/resources/1190201776262/2010/04/15/025079.pdf}{代理密码学研究现状}

\href{http://history.ccf.org.cn/resources/1190201776262/2010/04/15/025084.pdf}{交叉点缓存交换结构低层调度及确保时限高层调度}

\subsection{特别报道}
\href{http://history.ccf.org.cn/resources/1190201776262/2010/04/15/025008.pdf}{专家精英聚京城  青年才俊展风采(free)}

\href{http://history.ccf.org.cn/resources/1190201776262/2010/04/15/025010.pdf}{重视基础研究  提高学术水平(free)}

\href{http://history.ccf.org.cn/resources/1190201776262/2010/04/15/025012.pdf}{长期合作  共创未来(free)}

\href{http://history.ccf.org.cn/resources/1190201776262/2010/04/15/025013.pdf}{锐意进取  再创辉煌(free)}

\href{http://history.ccf.org.cn/resources/1190201776262/2010/04/15/025014.pdf}{建立科学的人才培养和评价体系(free)}


\section{\href{http://history.ccf.org.cn/sites/ccf/jsjtbbd.jsp?contentId=2542567628989}{\textbf{2008年第02期(总第24期)}}}
中文信息处理奇葩绽放 中文信息处理是我国计算机领域中的一朵奇葩,是计算机技术与语言学、心理学、数学、控制论、信息论、声学、自动化技术等相互交叉融合而形成的一个学科......
\subsection{专题}
\href{http://history.ccf.org.cn/resources/1190201776262/2010/04/15/024012.pdf}{中文信息处理“奇葩绽放”(free)}

\href{http://history.ccf.org.cn/resources/1190201776262/2010/04/15/024015.pdf}{汉语语言资源建设的理论基础与发展规划(free)}

\href{http://history.ccf.org.cn/resources/1190201776262/2010/04/15/024021.pdf}{中文信息处理技术评测综述(free)}

\href{http://history.ccf.org.cn/resources/1190201776262/2010/04/15/024029.pdf}{中文语义处理}

\href{http://history.ccf.org.cn/resources/1190201776262/2010/04/15/024036.pdf}{机器翻译技术及应用}

\href{http://history.ccf.org.cn/resources/1190201776262/2010/04/15/024042.pdf}{大海捞针亦有道——中文信息检索技术的现状与挑战}

\href{http://history.ccf.org.cn/resources/1190201776262/2010/04/15/024047.pdf}{中文文本情感倾向性分析}

\href{http://history.ccf.org.cn/resources/1190201776262/2010/04/15/024054.pdf}{语音识别技术与应用的发展趋势}

\subsection{专栏}
\href{http://history.ccf.org.cn/resources/1190201776262/2010/04/15/024070.pdf}{科研活动信息化——实现科学技术现代化的必由之路}

\href{http://history.ccf.org.cn/resources/1190201776262/2010/04/15/024058.pdf}{2007年高性能计算机排行榜对比分析}

\href{http://history.ccf.org.cn/resources/1190201776262/2010/04/15/024065.pdf}{基于自然语言处理的入侵检测研究}


\section{\href{http://history.ccf.org.cn/sites/ccf/jsjtbbd.jsp?contentId=2542567628986}{\textbf{2008年第01期(总第23期)}}}
CNCC 2007特邀报告
\subsection{特邀报告}
\href{http://history.ccf.org.cn/resources/1190201776262/2010/04/15/023012.pdf}{2007中国计算机大会开幕词}

\href{http://history.ccf.org.cn/resources/1190201776262/2010/04/15/023013.pdf}{演算之美(free)}

\href{http://history.ccf.org.cn/resources/1190201776262/2010/04/15/023018.pdf}{信息化助力苏州第三次战略转型}

\href{http://history.ccf.org.cn/resources/1190201776262/2010/04/15/023020.pdf}{网络化的计算机系统:挑战与机遇(free)}

\href{http://history.ccf.org.cn/resources/1190201776262/2010/04/15/023024.pdf}{超微(AMD)全新科技引领CPU走向革新}

\href{http://history.ccf.org.cn/resources/1190201776262/2010/04/15/023029.pdf}{对未来网络化软件技术的几点认识}

\href{http://history.ccf.org.cn/resources/1190201776262/2010/04/15/023036.pdf}{纳入973计划的无线传感网络研究}

\href{http://history.ccf.org.cn/resources/1190201776262/2010/04/15/023043.pdf}{推进文档格式国家标准UOF}

\href{http://history.ccf.org.cn/resources/1190201776262/2010/04/15/023048.pdf}{下一代互联网的研究与思考}

\href{http://history.ccf.org.cn/resources/1190201776262/2010/04/15/023052.pdf}{龙芯之路}

\href{http://history.ccf.org.cn/resources/1190201776262/2010/04/15/023059.pdf}{视频编码技术的演进及其AVS视频标准}

\href{http://history.ccf.org.cn/resources/1190201776262/2010/04/15/023065.pdf}{“星光中国芯工程”创新之路}

\href{http://history.ccf.org.cn/resources/1190201776262/2010/04/15/023068.pdf}{视频处理及其与图形的融合}

\href{http://history.ccf.org.cn/resources/1190201776262/2010/04/15/023073.pdf}{C*Core技术创新与产业化}


\section{\href{http://history.ccf.org.cn/sites/ccf/jsjtbbd.jsp?contentId=2542567628983}{\textbf{2007年第12期(总第22期)}}}
智能科学:实现人工智能长期目标的途径 史忠植(CCF高级会员,CCF理事,中国科学院计算技术研究所研究员)  
目前,科学技术发展趋势3个特点:(1)学科结构重心从物理科学转移到生命科学;(2)跨学科综合研究滋生新科学;(3)社会对科学的需求更加迫切。信息技术在近20年内仍处在蓬勃发展期,仍然是科技发展的第一技术......
\subsection{专题}
\href{http://history.ccf.org.cn/resources/1190201776262/2010/04/15/022010.pdf}{智能科学:实现人工智能长期目标的途径(free)}

\href{http://history.ccf.org.cn/resources/1190201776262/2010/04/15/022013.pdf}{数学机械化研究回顾与展望}

\href{http://history.ccf.org.cn/resources/1190201776262/2010/04/15/022018.pdf}{仿生模式识别的研究进展(free)}

\href{http://history.ccf.org.cn/resources/1190201776262/2010/04/15/022026.pdf}{主体计算}

\href{http://history.ccf.org.cn/resources/1190201776262/2010/04/15/022035.pdf}{机器学习与数据挖掘}

\href{http://history.ccf.org.cn/resources/1190201776262/2010/04/15/022045.pdf}{模式识别研究进展}

\href{http://history.ccf.org.cn/resources/1190201776262/2010/04/15/022053.pdf}{自主机器人研究的若干挑战}

\href{http://history.ccf.org.cn/resources/1190201776262/2010/04/15/022066.pdf}{语义万维网服务}

\href{http://history.ccf.org.cn/resources/1190201776262/2010/04/15/022073.pdf}{知识工程研究新思路}

\subsection{专栏}
\href{http://history.ccf.org.cn/resources/1190201776262/2010/04/15/022078.pdf}{对计算研究的九大挑战(下)}


\section{\href{http://history.ccf.org.cn/sites/ccf/jsjtbbd.jsp?contentId=2542567628980}{\textbf{2007年第11期(总第21期)}}}
软件工程:面向21世纪 软件工程(Software Engineering)的定义五花八门,比较公认的是美国电气与电子工程师协会(IEEE)给出的:“(1)将系统化的、严格约束的、可量化的方法应用于软件的开发、运行和维护,即将工程化应用于软件......
\subsection{专题}
\href{http://history.ccf.org.cn/resources/1190201776262/2010/04/15/021014.pdf}{软件工程:面向21世纪(free)}

\href{http://history.ccf.org.cn/resources/1190201776262/2010/04/15/021017.pdf}{领域工程——实现软件复用的有效途径(free)}

\href{http://history.ccf.org.cn/resources/1190201776262/2010/04/15/021026.pdf}{开放环境下基于信任管理的软件可信保障(free)}

\href{http://history.ccf.org.cn/resources/1190201776262/2010/04/15/021035.pdf}{软件需求工程:部分研究工作进展}

\href{http://history.ccf.org.cn/resources/1190201776262/2010/04/15/021044.pdf}{软件过程技术——解决软件质量问题的有效途径}

\subsection{专栏}
\href{http://history.ccf.org.cn/resources/1190201776262/2010/04/15/021056.pdf}{对计算研究的九大挑战(上)}

\href{http://history.ccf.org.cn/resources/1190201776262/2010/04/15/021066.pdf}{关于需求工程发展历程的讨论(下)}

\href{http://history.ccf.org.cn/resources/1190201776262/2010/04/15/021078.pdf}{从计算思维到计算文化}

\href{http://history.ccf.org.cn/resources/1190201776262/2010/04/15/021083.pdf}{计算思维}


\section{\href{http://history.ccf.org.cn/sites/ccf/jsjtbbd.jsp?contentId=2542567628977}{\textbf{2007年第10期(总第20期)}}}
网络计算 计算机领域的特点之一就是变化快,而且新名词不断出现。但是,如果从计算机学科的知识体(Body of Knowledge)审视,则这个学科还是比较具有规律性、稳定性和连续性的......
\subsection{专题}
\href{http://history.ccf.org.cn/resources/1190201776262/2010/04/15/020014.pdf}{网络计算(free)}

\href{http://history.ccf.org.cn/resources/1190201776262/2010/04/15/020016.pdf}{关于网格的访谈}

\href{http://history.ccf.org.cn/resources/1190201776262/2010/04/15/020018.pdf}{中国国家网格的研究与实践(free)}

\href{http://history.ccf.org.cn/resources/1190201776262/2010/04/15/020028.pdf}{中国教育科研网格ChinaGrid(free)}

\href{http://history.ccf.org.cn/resources/1190201776262/2010/04/15/020036.pdf}{互联网资源空间模型}

\href{http://history.ccf.org.cn/resources/1190201776262/2010/04/15/020041.pdf}{为搜索引擎学习最优的排序模型}

\href{http://history.ccf.org.cn/resources/1190201776262/2010/04/15/020046.pdf}{构建互联网服务的创新平台}

\href{http://history.ccf.org.cn/resources/1190201776262/2010/04/15/020052.pdf}{当我们一天能搜集一千万网页后……}

\href{http://history.ccf.org.cn/resources/1190201776262/2010/04/15/020058.pdf}{什么是服务科学中的“服务”?}

\href{http://history.ccf.org.cn/resources/1190201776262/2010/04/15/020063.pdf}{网格研究上的理性回归}

\subsection{专栏}
\href{http://history.ccf.org.cn/resources/1190201776262/2010/04/15/020066.pdf}{关于需求工程发展历程的讨论(上)}

\href{http://history.ccf.org.cn/resources/1190201776262/2010/04/15/020074.pdf}{四核处理器推动应用发展}


\section{\href{http://history.ccf.org.cn/sites/ccf/jsjtbbd.jsp?contentId=2542567628974}{\textbf{2007年第09期(总第19期)}}}
密码学:信息安全的基石 密码学主要由密码编码学和密码分析学两个分支组成。密码编码学的主要任务是寻求产生安全性高的有效密码算法和密码协议,以满足对消息进行加密或认证的要求。密码分析学的主要任务是破译密码算法和密码协议或伪造认证信息,实现窃取机密信息或进行诈骗破坏活动......
\subsection{专题}
\href{http://history.ccf.org.cn/resources/1190201776262/2010/04/15/019016.pdf}{密码学:信息安全的基石(free)}

\href{http://history.ccf.org.cn/resources/1190201776262/2010/04/15/019018.pdf}{序列密码发展现状(free)}

\href{http://history.ccf.org.cn/resources/1190201776262/2010/04/15/019026.pdf}{AES后分组密码的研究现状及发展趋势(free)}

\href{http://history.ccf.org.cn/resources/1190201776262/2010/04/15/019033.pdf}{公钥密码学进展}

\href{http://history.ccf.org.cn/resources/1190201776262/2010/04/15/019044.pdf}{密钥管理体系}

\href{http://history.ccf.org.cn/resources/1190201776262/2010/04/15/019053.pdf}{计算困难的数学问题——公钥密码算法的基石}

\href{http://history.ccf.org.cn/resources/1190201776262/2010/04/15/019060.pdf}{椭圆曲线密码研究进展}

\href{http://history.ccf.org.cn/resources/1190201776262/2010/04/15/019064.pdf}{量子密码研究进展与我国量子密码产业化分析}

\href{http://history.ccf.org.cn/resources/1190201776262/2010/04/15/019076.pdf}{演化密码的研究进展}


\section{\href{http://history.ccf.org.cn/sites/ccf/jsjtbbd.jsp?contentId=2542567628971}{\textbf{2007年第08期(总第18期)}}}
数据管理新技术 数据管理是指对数据进行分类、组织、编码、存储、检索和维护。数据管理技术的起源可以追溯到20世纪50年代中期......
\subsection{专题}
\href{http://history.ccf.org.cn/resources/1190201776262/2010/04/15/018020.pdf}{数据管理新技术研究(free)}

\href{http://history.ccf.org.cn/resources/1190201776262/2010/04/15/018024.pdf}{数据空间——数据管理新概念(free)}

\href{http://history.ccf.org.cn/resources/1190201776262/2010/04/15/018030.pdf}{基于闪存的数据库技术(free)}

\href{http://history.ccf.org.cn/resources/1190201776262/2010/04/15/018036.pdf}{场合感应中的数据管理问题}

\href{http://history.ccf.org.cn/resources/1190201776262/2010/04/15/018042.pdf}{可信数据管理研究}

\href{http://history.ccf.org.cn/resources/1190201776262/2010/04/15/018050.pdf}{RFID数据管理的研究进展}

\href{http://history.ccf.org.cn/resources/1190201776262/2010/04/15/018059.pdf}{图结构数据搜索的概念、问题与进展}

\href{http://history.ccf.org.cn/resources/1190201776262/2010/04/15/018066.pdf}{Mashup技术——为数据空间提供便利的应用构建技术}

\subsection{专栏}
\href{http://history.ccf.org.cn/resources/1190201776262/2010/04/15/018070.pdf}{全球化和未来标准}

\href{http://history.ccf.org.cn/resources/1190201776262/2010/04/15/018074.pdf}{构建Web 2.0}

\href{http://history.ccf.org.cn/resources/1190201776262/2010/04/15/018077.pdf}{从“汉芯”事件反省中国专家体系}


\section{\href{http://history.ccf.org.cn/sites/ccf/jsjtbbd.jsp?contentId=2542567628968}{\textbf{2007年第07期(总第17期)}}}
系统测试:为“防微杜渐”,须“吹毛求疵” 测试是质量保证的重要手段。其基本目的就是发现问题(“挑毛病”)。“不吹毛而求小疵,不洗垢而察之难”(《韩非子·大体》),因而测试用于防(质量隐)患于未然......
\subsection{专题}
\href{http://history.ccf.org.cn/resources/1190201776262/2010/04/15/017016.pdf}{系统测试:为“防微杜渐”,须“吹毛求疵”}

\href{http://history.ccf.org.cn/resources/1190201776262/2010/04/15/017018.pdf}{片上系统调试技术}

\href{http://history.ccf.org.cn/resources/1190201776262/2010/04/15/017025.pdf}{面向故障模型的软件测试技术}

\href{http://history.ccf.org.cn/resources/1190201776262/2010/04/15/017032.pdf}{故障注入技术及其应用}

\href{http://history.ccf.org.cn/resources/1190201776262/2010/04/15/017042.pdf}{纳电子器件及电路测试方法}

\href{http://history.ccf.org.cn/resources/1190201776262/2010/04/15/017050.pdf}{集成电路测试中的过度测试问题}

\href{http://history.ccf.org.cn/resources/1190201776262/2010/04/15/017057.pdf}{数字电路的测试压缩方法}

\subsection{专栏}
\href{http://history.ccf.org.cn/resources/1190201776262/2010/04/15/017066.pdf}{如何寻找研究课题——我的半个世纪研究工作体会}


\section{\href{http://history.ccf.org.cn/sites/ccf/jsjtbbd.jsp?contentId=2542567628965}{\textbf{2007年第06期(总第16期)}}}
无线局域网技术和应用“方兴未艾” 为了使读者对无线局域网技术有一个概要性的了解,我撰写了《无线局域网概述》一文。文章在一小段引子之后概述了无线局域网的基本概念。在专门查找了一些资料后,简要介绍了IEEE 802.11系列的标准,希望通过从后缀a到s(当然,中间字母并不是连续的)这一系列标准的描述,给读者一个整体的感觉......
\subsection{专题}
\href{http://history.ccf.org.cn/resources/1190201776262/2010/04/15/016016.pdf}{无线局域网技术和应用“方兴未艾”}

\href{http://history.ccf.org.cn/resources/1190201776262/2010/04/15/016019.pdf}{无线局域网概述}

\href{http://history.ccf.org.cn/resources/1190201776262/2010/04/15/016028.pdf}{IEEE 802.11无线局域网结点切换}

\href{http://history.ccf.org.cn/resources/1190201776262/2010/04/15/016035.pdf}{IEEE 802.11的服务质量}

\href{http://history.ccf.org.cn/resources/1190201776262/2010/04/15/016042.pdf}{移动自组网关键技术}

\href{http://history.ccf.org.cn/resources/1190201776262/2010/04/15/016060.pdf}{无线局域网安全}

\href{http://history.ccf.org.cn/resources/1190201776262/2010/04/15/016067.pdf}{无线局域网与个人网、城域网和广域网之间的共存关系}

\subsection{专栏}
\href{http://history.ccf.org.cn/resources/1190201776262/2010/04/15/016080.pdf}{Wi-Fi—笔记本电脑里的精灵音乐家}


\section{\href{http://history.ccf.org.cn/sites/ccf/jsjtbbd.jsp?contentId=2542567628962}{\textbf{2007年第05期(总第15期)}}}
虚拟现实的使命与挑战 人们研究虚拟现实技术的目的是创建逼真的虚拟环境。这种虚拟环境不仅具有数字化表示,可以进行演算,而且是“看”得见,“摸”得着,可以“感受”和“操作”的。因为在虚拟现实技术中,既可以使用计算机信息处理技术和相应设备,将真实世界中的对象和事件转换为数字化表示,也可以将数字化内容转换为人类可以感知的视觉、触觉,乃至力觉。
\subsection{专题}
\href{http://history.ccf.org.cn/resources/1190201776262/2010/04/15/015016.pdf}{虚拟现实的使命与挑战}

\href{http://history.ccf.org.cn/resources/1190201776262/2010/04/15/015018.pdf}{我对虚拟化和虚拟现实的浅见}

\href{http://history.ccf.org.cn/resources/1190201776262/2010/04/15/015022.pdf}{虚拟现实技术研究进展}

\href{http://history.ccf.org.cn/resources/1190201776262/2010/04/15/015032.pdf}{虚拟现实技术——研究大型公共设施安全问题的月光宝盒}

\href{http://history.ccf.org.cn/resources/1190201776262/2010/04/15/015038.pdf}{现代虚拟现实技术的军事应用}

\href{http://history.ccf.org.cn/resources/1190201776262/2010/04/15/015045.pdf}{漫谈虚拟现实}

\href{http://history.ccf.org.cn/resources/1190201776262/2010/04/15/015058.pdf}{增强现实技术研究}

\subsection{专栏}
\href{http://history.ccf.org.cn/resources/1190201776262/2010/04/15/015066.pdf}{让虚拟现实更容易被接受}

\href{http://history.ccf.org.cn/resources/1190201776262/2010/04/15/015071.pdf}{放眼世界——以地点做标签,从地图看照片}

\href{http://history.ccf.org.cn/resources/1190201776262/2010/04/15/015076.pdf}{服务科学的兴起是社会经济发展的必然}

\href{http://history.ccf.org.cn/resources/1190201776262/2010/04/15/015079.pdf}{服务科学:今日经济的新领域}

\href{http://history.ccf.org.cn/resources/1190201776262/2010/04/15/015083.pdf}{从DVD事件看企业知识产权战略}


\section{\href{http://history.ccf.org.cn/sites/ccf/jsjtbbd.jsp?contentId=2542567628959}{\textbf{2007年第04期(总第14期)}}}
让“搜索”比“检索词-->网页列表”更好 2006年冬,《中国计算机学会通讯》编辑部邀请我组织一期关于“搜索技术”的专题文章,经过作者和编辑的一番努力,终于完成了。本期搜索技术文章共5篇,分别来自北京大学的张岩教授、微软亚洲研究院的聂再清博士等、百度公司的洪涛博士、搜狐公司的王小川博士和香港城市大学的祝建华博士与我。
\subsection{专题}
\href{http://history.ccf.org.cn/resources/1190201776262/2010/04/15/014016.pdf}{让“搜索”比“检索词→网页列表”更好}

\href{http://history.ccf.org.cn/resources/1190201776262/2010/04/15/014018.pdf}{一场无休止的竞赛——搜索引擎与垃圾信息制造者之间的战争}

\href{http://history.ccf.org.cn/resources/1190201776262/2010/04/15/014024.pdf}{对象级别的互联网垂直搜索}

\href{http://history.ccf.org.cn/resources/1190201776262/2010/04/15/014029.pdf}{社区化搜索——满足用户需求的有效途径}

\href{http://history.ccf.org.cn/resources/1190201776262/2010/04/15/014034.pdf}{用互联网搜索实现计算机智能}

\href{http://history.ccf.org.cn/resources/1190201776262/2010/04/15/014039.pdf}{一个易用廉价的社会科学研究工具——“易猫”}

\subsection{专栏}
\href{http://history.ccf.org.cn/resources/1190201776262/2010/04/15/014084.pdf}{嵌入式平台软件的未来之路}

\href{http://history.ccf.org.cn/resources/1190201776262/2010/04/15/014044.pdf}{研究人员使得网络搜索更智能化}

\href{http://history.ccf.org.cn/resources/1190201776262/2010/04/15/014056.pdf}{网络搜索引擎技术:下篇}

\href{http://history.ccf.org.cn/resources/1190201776262/2010/04/15/014060.pdf}{让商务智能更有用}

\href{http://history.ccf.org.cn/resources/1190201776262/2010/04/15/014052.pdf}{网络搜索引擎技术:上篇}

\href{http://history.ccf.org.cn/resources/1190201776262/2010/04/15/014048.pdf}{付费搜索}

\href{http://history.ccf.org.cn/resources/1190201776262/2010/04/15/014064.pdf}{IEEE 802.16安全子层分析与研究}

\href{http://history.ccf.org.cn/resources/1190201776262/2010/04/15/014073.pdf}{工业控制计算机技术及其趋势}


\section{\href{http://history.ccf.org.cn/sites/ccf/jsjtbbd.jsp?contentId=2542567628956}{\textbf{2007年第03期(总第13期)}}}
对等计算研究转变:从理论到实践 对等计算对于一般人并不算新鲜,虽然有些人还太清楚其具体含义,但是他们已经使用过诸如 BT 或电驴( eMule )(更确切地应该译成电骡)下载视频或歌曲。但是,对等计算的应用只是单调吗?对于这一技术,学术界都做了哪些方面的研究工作,将会给我们带来什么样的惊喜呢?
\subsection{专题}
\href{http://history.ccf.org.cn/resources/1190201776262/2010/04/15/013006.pdf}{对等计算研究转变:从理论到实践}

\href{http://history.ccf.org.cn/resources/1190201776262/2010/04/15/013008.pdf}{广域网DHT存储系统中副本可靠性的维护}

\href{http://history.ccf.org.cn/resources/1190201776262/2010/04/15/013021.pdf}{片段存储系统:一种基于P2P的网络存储服务}

\href{http://history.ccf.org.cn/resources/1190201776262/2010/04/15/013031.pdf}{P2P网络信任机制综述}

\href{http://history.ccf.org.cn/resources/1190201776262/2010/04/15/013041.pdf}{基于对等网络的应用层组播结构研究}

\href{http://history.ccf.org.cn/resources/1190201776262/2010/04/15/013051.pdf}{对等计算技术在大规模流媒体服务中的应用}


\section{\href{http://history.ccf.org.cn/sites/ccf/jsjtbbd.jsp?contentId=2542567628953}{\textbf{2007年第02期(总第12期)}}}
编译技术:应对片上多处理器的挑战 如今,计算机处理器不仅速度大大提高,而且已经迈入多核时代。在这种情况下,编译系统面临更加严重的挑战。
\subsection{专题}
\href{http://history.ccf.org.cn/resources/1190201776262/2010/04/15/012016.pdf}{编译技术:应对片上多处理器的挑战}

\href{http://history.ccf.org.cn/resources/1190201776262/2010/04/15/012018.pdf}{多核处理器编译研究的若干问题}

\href{http://history.ccf.org.cn/resources/1190201776262/2010/04/15/012023.pdf}{片上并行系统的程序设计模型和语言}

\href{http://history.ccf.org.cn/resources/1190201776262/2010/04/15/012032.pdf}{异构多核处理器的编程模型和编译技术}

\href{http://history.ccf.org.cn/resources/1190201776262/2010/04/15/012040.pdf}{SIMD编译优化技术研究概述}

\href{http://history.ccf.org.cn/resources/1190201776262/2010/04/15/012049.pdf}{验证编译器初探}

\href{http://history.ccf.org.cn/resources/1190201776262/2010/04/15/012057.pdf}{编译技术在检测程序缓冲区溢出漏洞方面的应用}

\subsection{专栏}
\href{http://history.ccf.org.cn/resources/1190201776262/2010/04/15/012080.pdf}{期刊:仅凭几个数字就可定论吗?}

\href{http://history.ccf.org.cn/resources/1190201776262/2010/04/15/012074.pdf}{在数据挖掘发展中引发的两大核心问题}

\href{http://history.ccf.org.cn/resources/1190201776262/2010/04/15/012086.pdf}{从Microsoft到Google,业界人士热议IT行业赢法则}


\section{\href{http://history.ccf.org.cn/sites/ccf/jsjtbbd.jsp?contentId=2542567628950}{\textbf{2007年第01期(总第11期)}}}

\subsection{专题}
\href{http://history.ccf.org.cn/resources/1190201776262/2010/04/15/011064.pdf}{HPP:千万亿次高性能计算机的体系结构}

\href{http://history.ccf.org.cn/resources/1190201776262/2010/04/15/011065.pdf}{高效能计算机研制中的权衡取舍问题}

\href{http://history.ccf.org.cn/resources/1190201776262/2010/04/15/011066.pdf}{对高性能计算机发展的思考}

\href{http://history.ccf.org.cn/resources/1190201776262/2010/04/15/011067.pdf}{面向高端关键商业应用的高效能计算——机遇与挑战}

\href{http://history.ccf.org.cn/resources/1190201776262/2010/04/15/011068.pdf}{透明计算——一种扩展的冯诺依曼结构及其实现}

\href{http://history.ccf.org.cn/resources/1190201776262/2010/04/15/011069.pdf}{中间件的发展与普适计算中间件}

\href{http://history.ccf.org.cn/resources/1190201776262/2010/04/15/011070.pdf}{下一代中间件}

\href{http://history.ccf.org.cn/resources/1190201776262/2010/04/15/011071.pdf}{信息安全的模型}

\href{http://history.ccf.org.cn/resources/1190201776262/2010/04/15/011072.pdf}{中国信息安全科学技术研究进展和展望}

\href{http://history.ccf.org.cn/resources/1190201776262/2010/04/15/011073.pdf}{杂凑密码算法的研究进展}

\href{http://history.ccf.org.cn/resources/1190201776262/2010/04/15/011074.pdf}{从问题型到合规性、从风险管理到对标管理}

\href{http://history.ccf.org.cn/resources/1190201776262/2010/04/15/011075.pdf}{三维几何建模与形状表示}

\href{http://history.ccf.org.cn/resources/1190201776262/2010/04/15/011076.pdf}{多媒体传感器网络的问题与挑战}

\href{http://history.ccf.org.cn/resources/1190201776262/2010/04/15/011077.pdf}{生活中的计算机视觉}

\href{http://history.ccf.org.cn/resources/1190201776262/2010/04/15/011078.pdf}{人机交互和新的计算技术}

\href{http://history.ccf.org.cn/resources/1190201776262/2010/04/15/011079.pdf}{大规模场景的实时图形绘制技术与引擎系统}

\href{http://history.ccf.org.cn/resources/1190201776262/2010/04/15/011080.pdf}{普适计算模式下的人机交互}

\href{http://history.ccf.org.cn/resources/1190201776262/2010/04/15/011081.pdf}{龙芯3号多核处理器设计及其挑战}

\href{http://history.ccf.org.cn/resources/1190201776262/2010/04/15/011082.pdf}{多核CPU与系统芯片}

\href{http://history.ccf.org.cn/resources/1190201776262/2010/04/15/011083.pdf}{多核处理器的软件挑战问题}

\href{http://history.ccf.org.cn/resources/1190201776262/2010/04/15/011084.pdf}{搜索技术与机器智能}

\href{http://history.ccf.org.cn/resources/1190201776262/2010/04/15/011085.pdf}{百度中文搜索引擎简介}

\href{http://history.ccf.org.cn/resources/1190201776262/2010/04/15/011086.pdf}{互联网搜索——建立全球基础设施和平台的根本}

\href{http://history.ccf.org.cn/resources/1190201776262/2010/04/15/011087.pdf}{融合是信息生产力发展的必然要求}

\href{http://history.ccf.org.cn/resources/1190201776262/2010/04/15/011088.pdf}{构建高效能、安全、融合的下一代IPv6网络}

\subsection{特邀报告}
\href{http://history.ccf.org.cn/resources/1190201776262/2010/04/15/011010.pdf}{中国计算机事业五十年回顾与展望}

\href{http://history.ccf.org.cn/resources/1190201776262/2010/04/15/011016.pdf}{对计算机技术创新发展的思考}

\href{http://history.ccf.org.cn/resources/1190201776262/2010/04/15/011021.pdf}{共同成长  不变的主题}

\href{http://history.ccf.org.cn/resources/1190201776262/2010/04/15/011023.pdf}{与中国计算机事业共同成长}

\href{http://history.ccf.org.cn/resources/1190201776262/2010/04/15/011028.pdf}{未来软件技术的发展趋势}

\href{http://history.ccf.org.cn/resources/1190201776262/2010/04/15/011031.pdf}{中国软件工程二十六年}

\href{http://history.ccf.org.cn/resources/1190201776262/2010/04/15/011039.pdf}{数字智能无所不在}

\href{http://history.ccf.org.cn/resources/1190201776262/2010/04/15/011042.pdf}{计算机和网络系统研究的挑战与机遇}

\href{http://history.ccf.org.cn/resources/1190201776262/2010/04/15/011047.pdf}{创新与发展——中国计算机事业五十周年与Acer成立三十周年}

\href{http://history.ccf.org.cn/resources/1190201776262/2010/04/15/011048.pdf}{IT服务如何创造客户价值}

\href{http://history.ccf.org.cn/resources/1190201776262/2010/04/15/011051.pdf}{软件技术的发展及未来方向}

\href{http://history.ccf.org.cn/resources/1190201776262/2010/04/15/011055.pdf}{中国计算机事业50年历程中的长城电脑}

\href{http://history.ccf.org.cn/resources/1190201776262/2010/04/15/011058.pdf}{对未来计算技术的几点思考}


\section{\href{http://history.ccf.org.cn/sites/ccf/jsjtbbd.jsp?contentId=2542567628947}{\textbf{2006年第06期(总第10期)}}}
文明与野蛮的较量——信息系统安全 网络是人类文明的最新产物,也是各种先进技术大融合的舞台。但是,到目前为止,如果从安全角度来看,网络世界干脆就是一个充满野蛮和原始的世界。因此,广大信息安全工作者的责任和义务是让网络世界回归文明,使得网络中的野蛮行为受到有效扼制。
\subsection{封面报道}
\href{http://history.ccf.org.cn/resources/1190201776262/2010/04/15/010016.pdf}{文明与野蛮的较量}

\href{http://history.ccf.org.cn/resources/1190201776262/2010/04/15/010018.pdf}{信息系统安全对抗理论与技术}

\href{http://history.ccf.org.cn/resources/1190201776262/2010/04/15/010027.pdf}{计算机系统内核层入侵检测技术}

\href{http://history.ccf.org.cn/resources/1190201776262/2010/04/15/010034.pdf}{无线传感器网络安全}

\href{http://history.ccf.org.cn/resources/1190201776262/2010/04/15/010043.pdf}{数字内容与安全}

\href{http://history.ccf.org.cn/resources/1190201776262/2010/04/15/010048.pdf}{网络协议信息隐藏}

\href{http://history.ccf.org.cn/resources/1190201776262/2010/04/15/010055.pdf}{自组织网络中的蛀洞攻击及防范}

\href{http://history.ccf.org.cn/resources/1190201776262/2010/04/15/010060.pdf}{融合网络安全综论}

\href{http://history.ccf.org.cn/resources/1190201776262/2010/04/15/010066.pdf}{信息系统安全应用(1)我国可信计算产业的发展与挑战}

\href{http://history.ccf.org.cn/resources/1190201776262/2010/04/15/010068.pdf}{信息系统安全应用(2)电子政务的认证授权与审计}

\subsection{专栏}
\href{http://history.ccf.org.cn/resources/1190201776262/2010/04/15/010070.pdf}{多(跨)媒体信息检索技术介绍}

\href{http://history.ccf.org.cn/resources/1190201776262/2010/04/15/010079.pdf}{基于主体的计算经济学:理论与应用}


\section{\href{http://history.ccf.org.cn/sites/ccf/jsjtbbd.jsp?contentId=2542567628944}{\textbf{2006年第05期(总第09期)}}}

\subsection{封面报道}
\href{http://history.ccf.org.cn/resources/1190201776262/2010/04/15/009016.pdf}{无线传感器网络:未来之星}

\href{http://history.ccf.org.cn/resources/1190201776262/2010/04/15/009017.pdf}{无线传感器网络的节点技术}

\href{http://history.ccf.org.cn/resources/1190201776262/2010/04/15/009025.pdf}{无线传感器网络操作系统研究进展}

\href{http://history.ccf.org.cn/resources/1190201776262/2010/04/15/009032.pdf}{无线传感器网络体系结构}

\href{http://history.ccf.org.cn/resources/1190201776262/2010/04/15/009043.pdf}{无线传感器网络的通信协议}

\href{http://history.ccf.org.cn/resources/1190201776262/2010/04/15/009053.pdf}{无线传感器网络监测质量及其覆盖控制理论}

\href{http://history.ccf.org.cn/resources/1190201776262/2010/04/15/009062.pdf}{国外无线传感器网络研究见闻}

\href{http://history.ccf.org.cn/resources/1190201776262/2010/04/15/009070.pdf}{无线传感器网络应用集锦(1)在精准农业中的应用}

\href{http://history.ccf.org.cn/resources/1190201776262/2010/04/15/009072.pdf}{无线传感器网络应用集锦(2)在故宫环境监测中的应用}

\href{http://history.ccf.org.cn/resources/1190201776262/2010/04/15/009075.pdf}{无线传感器网络应用集锦(3)在冶金设备远程监测系统中的应用}

\href{http://history.ccf.org.cn/resources/1190201776262/2010/04/15/009078.pdf}{无线传感器网络应用集锦(4)在停车场管理中的应用}

\subsection{专栏}
\href{http://history.ccf.org.cn/resources/1190201776262/2010/04/15/009080.pdf}{从ACM会议论文数量看差距}

\href{http://history.ccf.org.cn/resources/1190201776262/2010/04/15/009088.pdf}{要高度重视在顶级国际学术会议上发表论文}

\href{http://history.ccf.org.cn/resources/1190201776262/2010/04/15/009089.pdf}{如何保障学术诚信}


\section{\href{http://history.ccf.org.cn/sites/ccf/jsjtbbd.jsp?contentId=2542567628941}{\textbf{2006年第04期(总第08期)}}}
数据库:更高、更强、更广 随着计算机系统硬件技术的进步以及互联网和万维网技术的发展,数据库系统所管理的数据以及应用环境发生了很大的变化。其表现为数据种类越来越多、越来越复杂、数据量剧增、应用领域越来越广泛,可以说数据管理无处不需无处不在,从而为数据库技术不断带来新的需求、新的挑战和发展机遇。
\subsection{封面报道}
\href{http://history.ccf.org.cn/resources/1190201776262/2010/04/15/008016.pdf}{数据库:更高、更强、更广}

\href{http://history.ccf.org.cn/resources/1190201776262/2010/04/15/008018.pdf}{数据库和信息检索技术的融合}

\href{http://history.ccf.org.cn/resources/1190201776262/2010/04/15/008025.pdf}{万维网数据库技术}

\href{http://history.ccf.org.cn/resources/1190201776262/2010/04/15/008039.pdf}{海量数字资源据管理技术}

\href{http://history.ccf.org.cn/resources/1190201776262/2010/04/15/008047.pdf}{基于基因表达式编程的数据挖掘研究进展}

\href{http://history.ccf.org.cn/resources/1190201776262/2010/04/15/008055.pdf}{数据流分析与挖掘技术现状与发展}

\href{http://history.ccf.org.cn/resources/1190201776262/2010/04/15/008065.pdf}{传感器网络数据管理}

\href{http://history.ccf.org.cn/resources/1190201776262/2010/04/15/008072.pdf}{看得见的进步:国产数据库进展}

\subsection{专栏}
\href{http://history.ccf.org.cn/resources/1190201776262/2010/04/15/008078.pdf}{数学:搜索引擎中的引擎}


\section{\href{http://history.ccf.org.cn/sites/ccf/jsjtbbd.jsp?contentId=2542567628938}{\textbf{2006年第03期(总第07期)}}}
生物信息学:运用信息科学和计算技术探询生物领域的奥秘 生物信息学是生物学、数学和计算机科学交叉所形成的一门学科,该学科力图运用信息科学和计算技术的手段,通过数据分析和处理,揭示海量数据间的内在联系和生物学含义,并进而提炼有用的生物学知识。
\subsection{封面报道}
\href{http://history.ccf.org.cn/resources/1190201776262/2010/04/15/007016.pdf}{生物信息学:运用信息科学和计算技术探询生物领域的奥秘}

\href{http://history.ccf.org.cn/resources/1190201776262/2010/04/15/007017.pdf}{基因芯片的数据处理与分析}

\href{http://history.ccf.org.cn/resources/1190201776262/2010/04/15/007028.pdf}{高性能计算在基因组数据分析中的应用}

\href{http://history.ccf.org.cn/resources/1190201776262/2010/04/15/007034.pdf}{分子进化与生物信息学}

\href{http://history.ccf.org.cn/resources/1190201776262/2010/04/15/007044.pdf}{生物学文献挖掘}

\href{http://history.ccf.org.cn/resources/1190201776262/2010/04/15/007057.pdf}{基于Linux的生物信息操作环境}

\href{http://history.ccf.org.cn/resources/1190201776262/2010/04/15/007062.pdf}{一个新兴的交叉学科:系统生物学}

\subsection{专栏}
\href{http://history.ccf.org.cn/resources/1190201776262/2010/04/15/007070.pdf}{Web 2.0及其影响}

\href{http://history.ccf.org.cn/resources/1190201776262/2010/04/15/007077.pdf}{面向服务的企业建模和工作流管理研究}

\href{http://history.ccf.org.cn/resources/1190201776262/2010/04/15/007080.pdf}{从《国家中长期科学和技术发展规划纲要》看我国科技发展的自主创新}


\section{\href{http://history.ccf.org.cn/sites/ccf/jsjtbbd.jsp?contentId=2542567628935}{\textbf{2006年第02期(总第06期)}}}
促进信息科学和社会科学的交叉研究 人类已经进入了信息社会,时代要求更多的学者关注信息科学和社会科学的交叉研究,开拓信息科学和社会科学研究的新局面。
\subsection{封面报道}
\href{http://history.ccf.org.cn/resources/1190201776262/2010/04/15/006022.pdf}{促进信息科学和社会科学的交叉研究}

\href{http://history.ccf.org.cn/resources/1190201776262/2010/04/15/006023.pdf}{关于网络社会宏观信息学研究的一些思考}

\href{http://history.ccf.org.cn/resources/1190201776262/2010/04/15/006028.pdf}{社会计算的意义及其展望}

\href{http://history.ccf.org.cn/resources/1190201776262/2010/04/15/006036.pdf}{社会信息的网络化分析初探}

\href{http://history.ccf.org.cn/resources/1190201776262/2010/04/15/006043.pdf}{让社会科学插上信息技术的翅膀}

\subsection{专栏}
\href{http://history.ccf.org.cn/resources/1190201776262/2010/04/15/006048.pdf}{计算机音乐漫谈}

\href{http://history.ccf.org.cn/resources/1190201776262/2010/04/15/006051.pdf}{势函数与贪婪算法}

\href{http://history.ccf.org.cn/resources/1190201776262/2010/04/15/006053.pdf}{可信安全计算平台技术的研究现状与发展趋势}

\href{http://history.ccf.org.cn/resources/1190201776262/2010/04/15/006054.pdf}{国家863计算机主题发展与服务网格CROWN}

\href{http://history.ccf.org.cn/resources/1190201776262/2010/04/15/006056.pdf}{资讯科学的发展:一个小兵回头看历史}

\href{http://history.ccf.org.cn/resources/1190201776262/2010/04/15/006057.pdf}{虚拟计算环境:挑战与对策}

\href{http://history.ccf.org.cn/resources/1190201776262/2010/04/15/006058.pdf}{移动Ad-hoc网络与无线传感器网络}

\href{http://history.ccf.org.cn/resources/1190201776262/2010/04/15/006059.pdf}{英特尔与高性能计算}

\href{http://history.ccf.org.cn/resources/1190201776262/2010/04/15/006060.pdf}{互联网的成功经验和挑战}

\href{http://history.ccf.org.cn/resources/1190201776262/2010/04/15/006061.pdf}{信息科技与服务经济}

\href{http://history.ccf.org.cn/resources/1190201776262/2010/04/15/006062.pdf}{加强计算机理论基础教育及研究}

\href{http://history.ccf.org.cn/resources/1190201776262/2010/04/15/006063.pdf}{空间信息网格(SIG)技术的发展与应用}

\href{http://history.ccf.org.cn/resources/1190201776262/2010/04/15/006064.pdf}{高密度平台架构——下一代高效节能系统的创新}

\href{http://history.ccf.org.cn/resources/1190201776262/2010/04/15/006065.pdf}{认知信息学和视觉计算与表达}

\href{http://history.ccf.org.cn/resources/1190201776262/2010/04/15/006066.pdf}{时段演算概述}

\href{http://history.ccf.org.cn/resources/1190201776262/2010/04/15/006067.pdf}{计算机创新趋势——来自国际计算机创新大会的报告}

\href{http://history.ccf.org.cn/resources/1190201776262/2010/04/15/006075.pdf}{计算机教育圆桌会印象}

\href{http://history.ccf.org.cn/resources/1190201776262/2010/04/15/006076.pdf}{从SCI反思中国的学术评价体制——中国计算机学会YOCSEF论坛综述}

\href{http://history.ccf.org.cn/resources/1190201776262/2010/04/15/006079.pdf}{学术黑色幽默}


\section{\href{http://history.ccf.org.cn/sites/ccf/jsjtbbd.jsp?contentId=2542567628932}{\textbf{2006年第01期(总第05期)}}}
高性能计算:科学的第三种手段 高性能计算机犹如计算科学的大型天文望远镜,它是关系到国家安全和国民经济重要部门的关键技术装备,国民经济和社会的可持续发展越来越离不开高性能计算机。在国家安全、提高产业的自主设计能力和核心竞争力等方面,高性能计算机起着不可替代的作用,是各国必争的战略制高点。
\subsection{封面报道}
\href{http://history.ccf.org.cn/resources/1190201776262/2010/04/15/005020.pdf}{高性能计算:科学的第三种手段}

\href{http://history.ccf.org.cn/resources/1190201776262/2010/04/15/005022.pdf}{千万亿次超级计算机展望}

\href{http://history.ccf.org.cn/resources/1190201776262/2010/04/15/005027.pdf}{高性能算法与软件研究漫谈}

\href{http://history.ccf.org.cn/resources/1190201776262/2010/04/15/005035.pdf}{高性能计算应用与超级计算中心}

\href{http://history.ccf.org.cn/resources/1190201776262/2010/04/15/005041.pdf}{能力服务器:低成本办公信息化技术}

\href{http://history.ccf.org.cn/resources/1190201776262/2010/04/15/005050.pdf}{个人高性能计算机}

\subsection{专栏}
\href{http://history.ccf.org.cn/resources/1190201776262/2010/04/15/005054.pdf}{跨媒体海量信息检索:搜索引擎未来之方向}

\href{http://history.ccf.org.cn/resources/1190201776262/2010/04/15/005058.pdf}{人工智能在新军事变革中大显身手}

\href{http://history.ccf.org.cn/resources/1190201776262/2010/04/15/005063.pdf}{即时通讯的现状、未来走向和研究挑战}

\href{http://history.ccf.org.cn/resources/1190201776262/2010/04/15/005067.pdf}{用于科学研究的高性能、能源感知分布式计算}

\href{http://history.ccf.org.cn/resources/1190201776262/2010/04/15/005074.pdf}{Eclipse在Java IDE中崭露头角}

\href{http://history.ccf.org.cn/resources/1190201776262/2010/04/15/005078.pdf}{对计算机科学的反思}

\href{http://history.ccf.org.cn/resources/1190201776262/2010/04/15/005083.pdf}{高校改革路在何方?——中国计算机学会YOCSEF论坛综述}

\href{http://history.ccf.org.cn/resources/1190201776262/2010/04/15/005086.pdf}{商务网格:服务与技术的完美融合}


\section{\href{http://history.ccf.org.cn/sites/ccf/jsjtbbd.jsp?contentId=2542567628929}{\textbf{2005年第04期(总第04期)}}}
中间件:互联网时代一种重要的基础软件 目前,中间件已经成为网络时代最主要、也是最活跃的软件形态之一。令人兴奋的是,我国中间件研发成果在国家信息化进程中正在发挥越来越实质性的作用,中间件为中国软件产业的发展提供了新机遇。本专题通过以下5篇文章,全面展示了中间件技术的现状和发展。
\subsection{封面报道}
\href{http://history.ccf.org.cn/resources/1190201776262/2010/04/15/004016.pdf}{中间件:互联网时代一种重要的基础软件}

\href{http://history.ccf.org.cn/resources/1190201776262/2010/04/15/004017.pdf}{软件中间件技术现状及发展}

\href{http://history.ccf.org.cn/resources/1190201776262/2010/04/15/004029.pdf}{分布对象中间件技术发展——以CORBA标准为背景}

\href{http://history.ccf.org.cn/resources/1190201776262/2010/04/15/004041.pdf}{J2EE技术剖析与评价}

\href{http://history.ccf.org.cn/resources/1190201776262/2010/04/15/004049.pdf}{Web服务技术与面向服务的协议计算}

\href{http://history.ccf.org.cn/resources/1190201776262/2010/04/15/004054.pdf}{开放环境下的软件中间件新技术——途径与进展}

\subsection{专栏}
\href{http://history.ccf.org.cn/resources/1190201776262/2010/04/15/004064.pdf}{下一代互联网搜索发展趋势}

\href{http://history.ccf.org.cn/resources/1190201776262/2010/04/15/004068.pdf}{LAMP给企业级系统开发减负}

\href{http://history.ccf.org.cn/resources/1190201776262/2010/04/15/004071.pdf}{三维模型分割(下)}

\href{http://history.ccf.org.cn/resources/1190201776262/2010/04/15/004080.pdf}{涉及计算机程序的发明专利保护}

\href{http://history.ccf.org.cn/resources/1190201776262/2010/04/15/004086.pdf}{只有自主创新才能获得技术转让}

\href{http://history.ccf.org.cn/resources/1190201776262/2010/04/15/004087.pdf}{因特网≠互联网——Internet与internet是两个不同名词}


\section{\href{http://history.ccf.org.cn/sites/ccf/jsjtbbd.jsp?contentId=2542567628926}{\textbf{2005年第03期(总第03期)}}}
普适计算:营造以人为本的信息服务新环境 普适计算力图实现计算技术从以机器为中心到以人为中心的转变。这将极大地促进信息技术在全社会的普遍应用,具有重要的战略意义。普适计算是全新计算模式的探索,具有鲜明的交叉学科的特点........
\subsection{封面报道}
\href{http://history.ccf.org.cn/resources/1190201776262/2010/04/15/003018.pdf}{普适计算:营造以人为本的信息服务新环境}

\href{http://history.ccf.org.cn/resources/1190201776262/2010/04/15/003021.pdf}{未来信息社会的脑与手}

\href{http://history.ccf.org.cn/resources/1190201776262/2010/04/15/003024.pdf}{泛印技术}

\href{http://history.ccf.org.cn/resources/1190201776262/2010/04/15/003026.pdf}{智能手机}

\href{http://history.ccf.org.cn/resources/1190201776262/2010/04/15/003029.pdf}{智能空间}

\href{http://history.ccf.org.cn/resources/1190201776262/2010/04/15/003033.pdf}{透明计算}

\href{http://history.ccf.org.cn/resources/1190201776262/2010/04/15/003035.pdf}{智能网络应用协议}

\href{http://history.ccf.org.cn/resources/1190201776262/2010/04/15/003037.pdf}{笔式用户界面}

\href{http://history.ccf.org.cn/resources/1190201776262/2010/04/15/003043.pdf}{闪联标准}

\subsection{专栏}
\href{http://history.ccf.org.cn/resources/1190201776262/2010/04/15/003050.pdf}{未来互联环境}

\href{http://history.ccf.org.cn/resources/1190201776262/2010/04/15/003058.pdf}{三维模型分割}

\href{http://history.ccf.org.cn/resources/1190201776262/2010/04/15/003066.pdf}{从曙光十年的发展看自主创新}

\href{http://history.ccf.org.cn/resources/1190201776262/2010/04/15/003068.pdf}{体系结构设计的折中误区}

\href{http://history.ccf.org.cn/resources/1190201776262/2010/04/15/003075.pdf}{即时通信:黑客的新目标}


\section{\href{http://history.ccf.org.cn/sites/ccf/jsjtbbd.jsp?contentId=2542567628923}{\textbf{2005年第02期(总第02期)}}}
并行算法研究进展 并行算法为并行计算的发展奠定理论基础,是我国高性能计算技术发展的关键之一。目前我们需要建立完整的学科研究体系,形成一体化的研究方法,重视培养人才。
\subsection{封面报道}
\href{http://history.ccf.org.cn/resources/1190201776262/2010/04/15/002018.pdf}{并行算法研究进展}

\href{http://history.ccf.org.cn/resources/1190201776262/2010/04/15/002022.pdf}{模态逻辑与模型检测}

\href{http://history.ccf.org.cn/resources/1190201776262/2010/04/15/002027.pdf}{计算复杂性理论部分进展简述}

\subsection{专栏}
\href{http://history.ccf.org.cn/resources/1190201776262/2010/04/15/002038.pdf}{对等计算研究概论}

\href{http://history.ccf.org.cn/resources/1190201776262/2010/04/15/002054.pdf}{网络存储系统与技术的现状与发展趋势}

\href{http://history.ccf.org.cn/resources/1190201776262/2010/04/15/002065.pdf}{室件(Roomware)——计算机的“消失”与交互方式的演变}

\href{http://history.ccf.org.cn/resources/1190201776262/2010/04/15/002073.pdf}{资源虚拟的复兴}

\href{http://history.ccf.org.cn/resources/1190201776262/2010/04/15/002078.pdf}{提高自主创新能力必须端正认识}


\section{\href{http://history.ccf.org.cn/sites/ccf/jsjtbbd.jsp?contentId=2542567628885}{\textbf{2005年第01期(总第01期)}}}

\subsection{封面报道}
\href{http://history.ccf.org.cn/resources/1190201776262/2010/04/15/001010.pdf}{高性能通用微处理器研发现状及发展策略}

\href{http://history.ccf.org.cn/resources/1190201776262/2010/04/15/001023.pdf}{片上多处理器发展趋势}

\href{http://history.ccf.org.cn/resources/1190201776262/2010/04/15/001031.pdf}{高性能多核微处理器研究动态与发展思路}

\subsection{专栏}
\href{http://history.ccf.org.cn/resources/1190201776262/2010/04/15/001038.pdf}{能量效率导向的无线传感器网络协议的研究}

\href{http://history.ccf.org.cn/resources/1190201776262/2010/04/15/001047.pdf}{从面向对象到面向方面}

\href{http://history.ccf.org.cn/resources/1190201776262/2010/04/15/001050.pdf}{奇妙的微电源——核微型电池研发现状}

\href{http://history.ccf.org.cn/resources/1190201776262/2010/04/15/001056.pdf}{姿势识别技术及其未来}

\href{http://history.ccf.org.cn/resources/1190201776262/2010/04/15/001060.pdf}{移动软件运行平台}

\href{http://history.ccf.org.cn/resources/1190201776262/2010/04/15/001065.pdf}{界面的设计空间}

\href{http://history.ccf.org.cn/resources/1190201776262/2010/04/15/001068.pdf}{机器智能需要神经科学}

\href{http://history.ccf.org.cn/resources/1190201776262/2010/04/15/001072.pdf}{想说、敢说、会说——中国标准之路如何走}


\end{document}